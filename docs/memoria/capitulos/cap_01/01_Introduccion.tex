
\chapter{Introducción}

% --------------------------------------------------------------------------------------------------------------------------

\section{Descripción del problema}

\subsection{Antropología forense}

La antropología es la ciencia que estudia la humanidad en todas sus dimensiones: biológica, cultural, lingüística o
arqueológica \cite{AAA2022AnthropologyDefinition}, a lo largo del tiempo y en distintas partes del mundo. La antropología 
biológica o física se centra en el estudio de la anatomía, el crecimiento, la adaptación y la evolución del cuerpo humano 
\cite{nawrocki1996OutlineFA}. 

Dentro de este campo, la \textbf{antropología forense (AF)} es el subcampo especializado que aplica métodos y técnicas 
antropológicas para resolver cuestiones médico-legales \cite{nawrocki1996OutlineFA}, empleando conocimientos de 
antropología física, aunque a veces también de la arqueología, para la correcta recuperación y análisis de la evidencia 
forense.

Tradicionalmente, los antropólogos forenses han tenido cinco principales objetivos en su trabajo \cite{byers2023}:

\begin{enumerate}

    \item Determinar el \textbf{perfil biológico} de un individuo (es decir, sexo, edad, estatura y ascendencia) 
    cuando los tejidos blandos se han deteriorado hasta el punto de que estas características no pueden determinarse
    mediante inspección visual. 

    \item Identificar la naturaleza de lesiones traumáticas (como heridas de bala, puñaladas o fracturas) en huesos humanos, 
    así como sus causantes, con el objetivo de recopilar información sobre la causa y circunstancias de la muerte.

    \item Estimar el intervalo \textit{post mortem}, es decir, el tiempo transcurrido desde la muerte, gracias a su 
    conocimiento sobre los procesos de descomposición corporal.
    
    \item Asistir en la localización, recuperación y conservación de los restos (superficiales o enterrados) aplicando 
    técnicas arqueológicas, garantizando la recolección de toda la evidencia forense relevante.

    \item Proporcionar información clave para la \textbf{identificación} de los fallecidos, basándose en las características
    distintivas de los esqueletos.

\end{enumerate}

Además de estos roles, en la actualidad los antropólogos desempeñan otros trabajos que no están relacionados con el ámbito 
criminalístico. Entre ellos, uno de sus campos de acción más relevantes es la \textbf{identificación de víctimas en contextos 
de catástrofes masivas} \cite{deBoer2019, prinz2007, beauthier2009}, como accidentes aéreos, ataques terroristas o desastres 
naturales, donde los restos suelen estar mutilados o desfigurados.

Su labor también es fundamental en la \textbf{recuperación e identificación de violaciones sistemáticas de
derechos humanos}, como exterminios, persecuciones políticas y represiones dictatoriales \cite{skinner2003}.
Casos como la Guerra Civil Española y la Dictadura Franquista \cite{sanchisgimeno2024, baeta2015}, así como 
las múltiples dictaduras en el Cono Sur de América \cite{ataliva2024}, han requerido la intervención de 
equipos forenses para esclarecer la verdad histórica y restituir la identidad de las víctimas a sus 
familiares, contribuyendo al proceso de memoria, justicia y reparación para las familias afectadas.
Esta vinculación con la justicia trasciende lo nacional: la ciencia forense es clave en la \textbf{investigación 
de crímenes de guerra contra poblaciones civiles}. Organizaciones como Médicos por los Derechos Humanos y la 
ONU financian equipos especializados que documentan estos crímenes, proporcionando pruebas esenciales para 
tribunales internacionales \cite{tanaka2020}.

Y por último, también son fundamentales para \textbf{estimar la edad de personas vivas en casos legales}, 
especialmente cuando no existen registros confiables. Esto ocurre, por ejemplo, en casos de solicitudes 
de asilo, adopciones internacionales o procesos judiciales donde es necesario determinar si una persona 
es menor o mayor de edad, lo cual puede tener importantes implicaciones legales. Según el tipo de 
procedimiento, se puede requerir tanto la estimación de la edad mínima como la edad más probable del 
individuo, con el fin de priorizar la protección de los menores, evitando que queden expuestos a 
violaciones de sus derechos.


\subsection{Problemas de la antropología forense}

Como hemos visto, la \textbf{identificación humana (ID)} es el problema principal abordado por la AF.
Consiste en la determinación y verificación de la identidad de una persona en base a \cite{thompson2006}: 
evidencias circunstanciales (hora y lugar del descubrimiento del cuerpo, efectos personales, confirmación 
visual por parte de familiares y amigos); y evidencias físicas, obtenidas a través de examinación externa
de características como el sexo, color de piel, tatuajes, o huellas dactilares, o, cuando estas no estén
disponibles, mediante examinación interna con técnicas médico-científicas, donde se aplican técnicas de 
antropología y genética forense.

Cabe destacar que, aunque los análisis dactilares y genéticos superan en precisión identificativa a los 
métodos antropológicos, su aplicabilidad enfrenta limitaciones técnicas significativas que condicionan 
su uso en contextos forenses reales.
Las huellas dactilares requieren de: tejido blando preservado, lo que es común en cadáveres frescos, 
pero se pierde con la descomposición o la carbonización; y una base de datos que incluya la huella del 
individuo en vida (registros \textit{ante mortem}).

Por otro lado, en cuanto al análisis genético, este puede verse comprometido por una mala conservación del 
ADN que puede deberse a su degradación o contaminación. La concentración presente en un cadáver se reduce 
drásticamente en los primeros 8 meses \textit{post mortem} \cite{higgins2015}, y factores como las 
altas temperaturas, la exposición a humedad ambiental o la presencia de aguas subterraneas y entornos 
ricos en oxígeno, que fomentan la presencia microbiana, perjudican la conservación del ADN \cite{latham2018}. 
Y, aun extraída una secuencia válida de ADN, se necesita de muestras con las que compararla, a ser posible de 
familiares de primer grado, para establecer una identificación concluyente. 

Por tanto, la AF contribuye al problema de identificación en dos escenarios \cite{swganth2010}:  

\begin{enumerate}

    \item Cuando los otros métodos no son viables, dado que las pruebas no se puedan recoger o no sean válidas, o 
    no haya registros con los que compararlas.
    
    \item Como apoyo a otras técnicas de identificación. Por ejemplo, las técnicas de estimación del perfil 
    biológico pueden reducir el grupo de posibles coincidencias en bases de datos genéticos, facilitando el 
    cotejo de secuencias genéticas y reduciendo el coste del proceso.  

\end{enumerate}

La \textbf{estimación del perfil biológico (PB)} es, por tanto, un proceso fundamental de la AF, en el cual
se determinan características biológicas clave de un individuo \cite{byers2023}: 

\begin{itemize}
    \item \textbf{sexo}, mediante el análisis morfológico y métrico de rasgos sexuales en el esqueleto, 
    especialmente en la pelvis y el cráneo;
    \item \textbf{edad}, estimada a partir de cambios morfológicos y de desarrollo en el esqueleto, pudiendo 
    referirse tanto a la \textbf{edad al momento de la muerte} en restos óseos, como a la \textbf{edad cronológica} 
    \footnote{La edad cronológica es la edad real de una persona desde su nacimiento, mientras que la edad
    biológica refleja la condición fisiológica del cuerpo \cite{marcante2025}.}
    en personas vivas en contextos forenses o humanitarios;
    \item \textbf{estatura}, mediante la estimación de la talla a partir de longitudes óseas, particularmente 
    de los huesos largos; y
    \item \textbf{ascendencia}, analizando variaciones craneométricas y morfológicas asociadas a poblaciones 
    o grupos geográficos (actualmente en revisión \cite{ross2021a, ross2021b, flouri2022}).
\end{itemize}

En los problemas de ID, cuando estas características biológicas coinciden con los registros \textit{ante mortem}, 
se fortalece la hipótesis de identificación; en cambio, si existen una o más discrepancias ---especialmente de 
alguna característica firme como múltiples epífisis no fusionadas, que no pueden ocurrir en un adulto mayor---, 
el individuo es excluido como posible coincidencia \cite{byers2023}. 

En la Figura \ref{fig:SFI_pipeline} podemos observar que la estimación del PB es uno de los primeros pasos en 
el proceso de ID forense. 

\begin{figure}[h]
    \centering
    \includegraphics[width=\textwidth]{capitulos/cap_01/imagenes/SFI_pipeline.png}
    \caption{Procedimiento secuencial para la identificación forense basada en el esqueleto humano 
            (\textit{skeleton-based forensic identification}, SFI) \cite{mesejo2020}.} 
    \label{fig:SFI_pipeline}
\end{figure}

La estimación del perfil biológico en restos humanos es una tarea compleja, especialmente cuando se estima la edad
en el momento de la muerte, ya que hay diferentes métodos a aplicar dependiendo de la fase de desarrollo del individuo. 
Las variaciones en la morfología de los huesos son bien conocidas, pero estas no siempre ocurren al mismo tiempo en 
diferentes individuos, ya que no están expuestos a las mismos condiciones genéticas y del entorno.

Además, como se ha mencionado anteriormente, la estimación de edad también se realiza sobre personas vivas
en casos legales donde la edad es un factor determinante \cite{schmeling2016}, por ejemplo, con menores migrantes  
no acompañados. En estos casos no se tiene acceso a 
los huesos de la persona de forma directa, por lo que el análisis se realiza sobre imágenes médicas.

% sin registros confiables, 
% en casos legales donde la edad es un factor determinante \cite{schmeling2016}. En estos casos no se tiene acceso a 
% los huesos de la persona de forma directa, por lo que el análisis se realiza sobre imágenes médicas. Dependiendo 
% de la cuestión legal, se requerirá la estimación de la edad mínima del individuo o su edad más probable.

% en casos legales donde la edad es un factor determinante \cite{schmeling2016}, como en solicitudes de asilo, adopciones 
% internacionales o juicios donde se debe establecer si un individuo es menor o mayor de edad, lo que puede tener 
% consecuencias legales significativas. En estos casos no se tiene acceso a los huesos de la persona de forma 
% directa, por lo que el análisis se realiza sobre imágenes médicas. Dependiendo de la cuestión legal, se 
% requerirá la estimación de la edad mínima del individuo o su edad más probable.

% --------------------------------------------------------------------------------------------------------------------------
% --------------------------------------------------------------------------------------------------------------------------
% --------------------------------------------------------------------------------------------------------------------------

\section{Motivación}

Los métodos de estimación del perfil biológico se basan en la evaluación visual y en el análisis morfométrico de rasgos
esqueléticos, que requieren de conocimiento especializado. Sin embargo, su aplicación puede presentar ambigüedades en 
su formulación que den lugar a interepretaciones variables ---muchas veces fruto de sesgos cognitivos
\cite{nakhaeizadeh2014, cooper2019}--- y están sujetos a posibles errores de medición \cite{langley2018}.
Además, la gran variabilidad genética y ambiental entre individuos, que afecta la morfología del esqueleto y genera 
diferencias significativas entre poblaciones de distintas regiones \cite{ubelaker2017}, hace que muchos de estos métodos 
---basados en muestras de referencia limitadas o no representativas de la diversidad humana global--- pierdan precisión. 
Esto puede introducir sesgos al estimar el perfil biológico de individuos de grupos poco estudiados o con características 
atípicas.

Frente a estas limitaciones, recientes avances en inteligencia artificial (IA) y machine learning (ML) han demostrado 
el potencial de mejorar la exactitud y objetividad de estimación del perfil biológico, tanto para la estimación de sexo 
\cite{curate2017, darmawan2015, pinto2016} como de edad \cite{kim2017, larson2018, lee2017}. 

Estos modelos, que emplean imágenes médicas con algoritmos de visión por computador, siguen dos principales enfoques.
En el primer enfoque, parten de un método de AF clásico e intentan automatizarlo y/o mejorarlo 
\cite{stern2014, ajafernandez2004} (véase la Figura \ref{fig:MRI_pipeline}). 
Para ello, es necesario especificar:

\begin{enumerate}
    \item cómo extraer las características relevantes de las imágenes médicas, mediante técnicas de procesamiento de 
    imágenes o morfometría tradicional; y
    \item un modelo de clasificación o regresión (como SVM, redes neuronales simples o árboles de decisión) que opere 
    sobre estas características predefinidas.
\end{enumerate}

\begin{figure}[h]
    \centering
    \includegraphics[width=\textwidth]{capitulos/cap_01/imagenes/MRI_pipeline.png}
    \caption{Procedimiento secuencial para el método propuesto en \cite{stern2014}.} 
    \label{fig:MRI_pipeline}
\end{figure}

En cambio, en el enfoque más novedoso, el \textit{end-to-end}, el modelo aprende automáticamente 
tanto la extracción de características como la clasificación/regresión a partir de los datos en bruto. Este enfoque 
es posible gracias a las redes neuronales convolucionales, que eliminan la dependencia de criterios 
antropológicos preestablecidos y permite al modelo extraer por sí mismo las características más relevantes para la
estimación de sexo, edad, etc. 

Este enfoque se ha visto potenciado por el auge del Deep Learning, permitiendo a las CNNs aprender patrones complejos 
que podrían pasar inadvertidos por el ser humano, y mejorando la precisión de las predicciones \cite{stern2019, venema2022}. 
Sin embargo, aún mejorando la exactitud de las predicciones, los modelos siguen mostrando carencias respecto a la 
cuantificación de incertidumbre, pues no todas las predicciones tienen el mismo nivel de confianza o fiabilidad.
Ya en \cite{ferrante2009} se introducía no solo la necesidad de identificar el método adecuado para estimar la edad a 
partir de los elementos disponibles, sino también de evaluar su confiabilidad y realizar un estudio del error arrojado por 
las predicciones del método. 
Estos generalmente se han basado en la estadística frecuentista \cite{verma2020, stepanovsky2024, heinrich2024}
\footnote{La estadística frecuentista es la corriente estadística que desarrolla a partir de los conceptos de probabilidad
y que se centra en el cálculo de probabilidades y el contraste de hipótesis.}.
Un ejemplo de este tipo de análisis se ilustra en la Figura \ref{fig:regression_lentibia_stature}, donde se examina la 
distribución probabilística del error residual arrojado por el modelo de regresión propuesto en \cite{verma2020}.

\begin{figure}[h]
    \centering
    \includegraphics[width=0.7\textwidth]{capitulos/cap_01/imagenes/regression_line_lentibia_stature.png}
    \caption{Línea de regresión del modelo de regresión propuesto en \cite{verma2020} que predice la estatura
    a partir de la longitud de la tibia. En rojo, la línea de regresión; en verde, la línea de los intervalos de confianza 
    del 95\%; y en naranja, la línea de los intervalos de predicción al 95\% de confianza.
    } 
    \label{fig:regression_lentibia_stature}
\end{figure}

Aunque existen métricas para evaluar el error cuando se dispone de \textit{ground truth}, la mayoría de los modelos 
actuales se limitan a ofrecer predicciones puntuales en regresión \cite{park2024, imaizumi2021, stepanovsky2024} o 
etiquetas únicas en clasificación \cite{venema2022, park2024}, sin cuantificar la incertidumbre asociada a cada predicción.

Con lo anterior se expone la motivación de la aplicación de ML a la AF, así como de la necesidad de cuantificar la
incertidumbre en las predicciones, para ofrecer garantías de confiabilidad estadística que aspiren a sustentar la 
validez legal en contextos judiciales. Algunos datos que magnifican la necesidad de técnicas de AF confiables 
actualmente son:

% 1/3 de los muertos del 11S sin identificar

\begin{itemize}

    \item En los últimos años, ha aumentado significativamente el número de cadáveres hallados en el 
    territorio español, como podemos apreciar en la Figura \ref{fig:evolucion_hallazgosID_cadaveres} 
    \cite{interior2025desaparecidos}. 
    En 2024 se ha alcanzado una cifra record, ---en gran parte debido a las inundaciones de la DANA 
    Valencia---, de 531 cadáveres en 2024, de los cuales se pudo identificar a 323.

    \begin{figure}[h]
        \centering
        \includegraphics[width=\textwidth]{capitulos/cap_01/imagenes/hallazgos_cadaveres.png}
        \caption{Evolución de hallazgos/identificación de cadáveres en España (2010-2024) \cite{interior2025desaparecidos}.} 
        \label{fig:evolucion_hallazgosID_cadaveres}
    \end{figure}

    \item En 2020, de las 2.457 fosas totales documentadas de la Guerra Civil y el franquismo, aún 1.221 
    seguían sin ser intervenidas y se estimaba que ``con una intervención oficial del Estado podrían 
    recuperarse unos 20 a 25.000 individuos'' e identificar ``entre 5 y 7.000 de ellos'', estimándose 
    necesario contar con unos 40-50 profesionales de la antropología forense \cite{etxeberria2020}. 

    % \item De acuerdo con UNICEF \cite{unicef2013}, en 2012 cerca de 230 millones de niños menores de 
    % cinco años no contaban con un registro oficial de nacimiento. Las regiones con las tasas más bajas 
    % de registro incluyen África subsahariana (44\%) y el sur de Asia (39\%). Esta situación se agrava 
    % aún más, ya que muchos niños registrados no poseen un certificado de nacimiento, y los documentos 
    % existentes suelen perderse durante procesos de migración.

    \item En España, se ha registrado en la última década (2013-2023) un aumento significativo en la llegada
    de Menores Extranjeros No Acompañados \cite{fge2024,fge2019,fge2016,fge2013}, que ha disparado 
    consigo el número de diligencias abiertas para la determinación de su edad, como se ve reflejado 
    en la Figura \ref{fig:evolucion_DPDE}.

    \begin{figure}[h]
        \centering
        \includegraphics[width=\textwidth]{capitulos/cap_01/imagenes/dpde_España.png}
        \caption{Evolución del número de Diligencias Preprocesales de Determinación de Edad abiertas en España(2011–2023). 
                 Elaboración propia a partir de \cite{fge2013,fge2016,fge2019, fge2024}.} 
        \label{fig:evolucion_DPDE}
    \end{figure}

    \item La relevancia de la ciencia forense en la identificación de víctimas y la protección de la dignidad humana ha convertido 
    su aplicación en un pilar fundamental de los derechos humanos y la justicia internacional, naciendo así la  
    \textbf{acción forense humanitaria} \cite{cordner2017}. Esta disciplina emplea la ciencia forense con un propósito 
    exclusivamente humanitario, con los objetivos de: identificar a las personas fallecidas, gestionar dignamente sus restos y 
    aliviar el sufrimiento de sus familias en situaciones de conflicto, migración y desastres naturales \cite{tidballbinz2021}. 

\end{itemize}

% --------------------------------------------------------------------------------------------------------------------------

\section{Objetivos}

La \textbf{\textit{Conformal Prediction}} emerge como un marco teórico robusto para generar intervalos de predicción con 
garantías estadísticas sólidas, independientemente de la distribución subyacente de los datos. A diferencia de los 
enfoques tradicionales, este método no solo ofrece predicciones puntuales, sino que cuantifica la incertidumbre asociada
a cada estimación mediante intervalos adaptativos o conjuntos de predicción que reflejan la confiabilidad de la 
predicción en cada caso particular.

% A revisar próximo párrafo: ¿se usarán finalmente datos tabulares?

Este proyecto tiene un doble objetivo: por un lado, desde un prisma teórico, estudiar las ventajas y costes asociados a 
las diversas técnicas de inferencia conformal actuales; y, por otro, aplicarlo a un contexto práctico como es es el 
problema de estimación del perfil biológico, centrándonos en la estimación de edad y de sexo a partir de datos biológicos
e imágenes médicas. De esta forma, cuando estemos ante datos biológicos ambiguos, la conformal prediction podrá devolver 
conjuntos de predicciones con más de una etiqueta predicha (p.ej., \{masculino, femenino\}) en problemas de clasificación, 
o intervalos de predicción más amplios (p.ej., edad$\in$[17,20]) en problemas de regresión, en ambos casos para un 
nivel de confianza determinado.

Por tanto, ponemos desgranar los objetivos en:

\begin{itemize}

    \item Estudiar de forma exhaustiva la bibliografía sobre \textit{conformal prediction} y sus diversas variantes, así
          como de la estimación de sexo y edad, centrando nuestra atención en el estado del arte.

    \item Implementar, entrenar y validar modelos de regresión ---en problemas de estimación de edad--- y clasificación 
    ---tanto en problema de estimación de sexo como edad legal--- a los que aplicar la inferencia conformal.

    \item Comparar los intervalos y conjuntos de predicciones generados para evaluar su calibración empírica, robustez 
    ante datos ambiguos y utilidad forense, contrastándolos con métodos tradicionales (p.ej., intervalos de confianza 
    clásicos).  

    \item Realiza una primera aproximación a un marco interpretable y con garantías estadísticas para la estimación del 
    perfil biológico, donde la incertidumbre cuantificada pueda integrarse en informes periciales bajo estándares jurídicos.

\end{itemize}


En resumen, este trabajo no pretende estudiar las fuentes de incertidumbre en las técnicas de AF, ni desarrollar un 
modelo con mejor rendimiento que los actuales, sino explorar la integración de marcos probabilísticos en la práctica 
forense, y facilitar el uso de la inferencia conformal. 
Este enfoque proporciona estimaciones calibradas de incertidumbre, con garantías estadísticas de cobertura válidas 
bajo supuestos mínimos, útiles para la toma de decisiones fundamentadas en contextos prácticos 
donde la interpretabilidad y robustez son críticas.


