\chapter{Conclusiones y trabajos futuros}

\section{Conclusiones}

A la luz de los resultados obtenidos, se puede concluir que el empleo de métodos de predicción conformal constituye una herramienta de gran utilidad, ya que ofrece beneficios significativos en términos de cuantificación de la incertidumbre a un coste computacional muy reducido. Esto resulta especialmente relevante en contextos sensibles, donde la toma de decisiones derivada de estas estimaciones (p. ej., en procedimientos de asilo o investigaciones forenses) incide directamente sobre los derechos fundamentales de las personas.

% Puntos en común entre ambos tipos de predicción

En primer lugar, se observa que tanto las predicciones puntuales como las conformales mejoran en sus métricas a medida que aumenta el desempeño del modelo subyacente. Es decir, cuanto más preciso resulta el modelo al estimar el valor esperado o clase verdadera de cada instancia, mayor es también la calidad de los intervalos o conjuntos conformales asociados. Por tanto, \textbf{el objetivo de mejorar la precisión del modelo base y el de obtener mejores intervalos conformales es común y está alineado}. Esta sinergia entre el modelo y el método conformal es también crucial a la hora de su implementación.

En términos prácticos, \textbf{la mayoría de las técnicas de predicción conformal presentan la ventaja de no requerir un reentrenamiento completo del modelo}, siempre que se disponga de suficientes ejemplos para la calibración, distintos de los empleados en el entrenamiento o la validación. En caso de no contar con este volumen de datos, resulta necesario reentrenar el modelo tras realizar una nueva partición del conjunto de datos que reserve un subconjunto específico para la calibración.
Existen, no obstante, métodos que sí implican modificaciones en la arquitectura y reentrenamiento o, como la \textit{Conformalized Quantile Regression}. Este enfoque requiere incorporar nuevas salidas ---de la \textit{Quantile Regression}--- y entrenar nuevamente el modelo bajo una función de pérdida adaptada. Sin embargo, cabe destacar que este proceso de ajuste resulta relativamente poco costoso, dado que el modelo suele converger en pocas épocas a partir de uno base ya entrenado con una única salida.


% ¿Qué aporta la predicción conformal?

La principal contribución de la predicción conformal reside en \textbf{proporcionar una medida rigurosa de incertidumbre a través del conjunto predicho}. Algunas técnicas miden la incertidumbre del conjunto completo, de modo que todos los ejemplos reciben conjuntos de predicción del mismo tamaño. En estos casos, la incertidumbre no se refleja en la variabilidad del tamaño del conjunto, sino en la frecuencia con que dicho conjunto contiene o no el valor o clase verdadero. Así, los tamaños de los conjuntos conformales son constantes, lo que no deja de ser una aproximación muy cercana al análisis del error tradicional: se garantiza que, en promedio, el error se mantenga bajo un umbral prefijado.
Sin embargo, \textbf{los métodos conformales adaptativos entrañan un mayor potencial, ya que ajustan dinámicamente el tamaño del conjunto de predicción en función de la dificultad de predecir cada instancia}. De este modo, ejemplos en los que el modelo está más seguro tienden a recibir conjuntos más pequeños, mientras que en aquellos en los que la predicción es más incierta, los conjuntos se amplían para mantener la garantía de cobertura. 
Esta adaptatividad permite capturar mejor tanto la incertidumbre epistémica (p.ej., los intervalos de edad más amplios por escasez de datos en individuos de edad avanzada) como la estocástica (p.ej., los intervalos amplios por la mayor variabilidad fisiológica en edades avanzadas), reflejando así de manera más fiel la heterogeneidad de los datos. Esta adaptatividad no siempre consigue conjuntos más pequeños, pero aproxima más a una cobertura condicional.


% ¿Cuál es el coste de la predicción conformal?

En cuanto a los \textbf{costes de implementar la predicción conformal}, estos se concentran en dos aspectos principales:

\begin{itemize}

    \item \textbf{Reserva de datos para calibración}: destinar una fracción del conjunto de entrenamiento \textbf{puede degradar ligeramente el rendimiento del modelo en la predicción puntual}. No obstante, en el caso analizado, el volumen de datos fue suficiente para que la retención del 20\% apenas afectara los resultados: en el problema de estimación de edad, el error absolute medio apenas se resentía un 3.5\% (de 1.16 en métodos no conformales a 1.20 en ICP); o en el problema de estimación de la mayoría de edad, la exactitud solo variaba un 0.63\% de media (de 87.63\% en métodos no conformales al 87\% en los conformales). En contextos con conjuntos de datos reducidos, este aspecto puede volverse más problemático, por lo que resultaría recomendable explorar estrategias alternativas de predicción conformal como \textit{Jackknife+}. 
    
    \item \textbf{Incorporación del proceso de calibración e inferencia conformal}: la calibración supone una fase adicional, y en algunos casos (con los métodos APS, RAPS y SAPS) la inferencia conformal también un mayor coste en tiempo que la inferencia puntual. Aun así, en el problema de estimación de edad, los tiempos de calibración han supuesto en media menos de un 6\% de los tiempos totales de modelado, lo que evidencia que la sobrecarga computacional introducida por este método es mínima comparada con el coste global del proceso, especialmente cuando se emplean modelos de ML complejos. 

\end{itemize}


% Diferencias entre métodos de predicción conformal 

Cabe destacar que, si bien todos los métodos conformales garantizan teóricamente la cobertura marginal, en la práctica exhiben diferencias sustanciales:

\begin{itemize}

    \item Las \textbf{garantías de cobertura estadística no garantizan la cobertura empírica al nivel deseado sobre el conjunto de datos nuevo. La elección de la función de no conformidad es crucial}, pues determina la eficiencia de los intervalos: funciones mal calibradas pueden producir intervalos excesivamente amplios o poco informativos, mientras que elecciones adecuadas permiten intervalos más ajustados sin perder la validez. Nuestros resultados para el problema de clasificación de edad, en igualdad de condiciones (mismo dataset de calibración y modelo), exhiben métodos que logran desde un 94.29\% de cobertura hasta un 95.42\% para una cobertura global deseada del 95\%. 
    
    \item \textbf{Algunas variantes tienden a generar intervalos o conjuntos inestables, sobrecubriendo ciertos grupos e infracubriendo otros, mientras que otros ofrecen un mayor equilibrio}, reduciendo la sobrecobertura en los primeros en favor de una cobertura más homogénea entre subpoblaciones.
    %\todo{Completar}
     
\end{itemize}

En resumen, y atendiendo a los métodos analizados en este trabajo, se pueden establecer algunas conclusiones claras sobre estos:

\begin{itemize}
    
    \item Para problemas de regresión, CQR se consolida como la opción más robusta. Sus intervalos muestran una cobertura muy cercana a la nominal y mantienen tasas de cobertura empírica consistentes tanto en diferentes tamaños muestrales como en subpoblaciones definidas por edad cronológica y sexo. Esto lo convierte en un método confiable y estable en diversos escenarios.
    
    \item En problemas de clasificación, no es tan claro:
    \begin{itemize}
        \item LAC es la opción predeterminada: fácil de implementar y eficiente, logra valores muy cercanos a la cobertura marginal con conjuntos de predicción de tamaño muy moderado.
        \item MCM es una buena opción en casos con pocas clases, donde es prioritario lograr la cobertura requerida en todas ellas.
        \item SAPS es la mejor opción para problemas con muchas clases, poniéndo énfasis en la adaptatividad, logra tasas de cobertura marginal muy sólidas y conjuntos de predicción de tamaño muy estable. 
    \end{itemize}

\end{itemize}

En consecuencia, \textbf{del mismo modo que las predicciones puntuales exigen un análisis de error, las conformales requieren una evaluación sistemática de la cobertura empírica y la comparación entre métodos}. Este análisis debe identificar discrepancias entre la cobertura nominal y la real, detectar subpoblaciones sistemáticamente infracubiertas o sobrecubiertas, y examinar el tamaño de los intervalos. Unos intervalos excesivamente amplios carecen de utilidad práctica; unos demasiado estrechos, comprometen las garantías de cobertura. \textbf{El desafío inmediato reside en avanzar hacia métodos que se aproximen a la cobertura condicional, cerrando la brecha entre teoría y práctica.}

% ------------------------------------------------------------------------------------------------------------
% ------------------------------------------------------------------------------------------------------------

\section{Valoración del trabajo realizado}

% Objetivos planteados y en qué grado se han logrado

En relación con los objetivos planteados en la introducción, todos han sido cumplidos de manera satisfactoria. Se ha realizado un análisis detallado de las distintas técnicas de predicción conformal y del estado del arte en la cuantificación de la incertidumbre aplicada a problemas de estimación de edad. Asimismo, se implementaron con éxito diversas variantes de predicción conformal inductiva (disponibles en el repositorio \href{https://github.com/esdavide2910/tfg-bioprofile-uncertainty}{esdavide2910/tfg-bioprofile-uncertainty}), tanto para tareas de regresión como de clasificación, aplicándolas a un caso de estimación del perfil biológico. Para garantizar una comparación justa, dichas técnicas se contrastaron con aproximaciones heurísticas de predicción interválica y basada en conjuntos. Los resultados obtenidos han permitido evidenciar tanto el potencial como las limitaciones de estos métodos en el contexto específico de la estimación de edad a partir de radiografías maxilofaciales.  

% Evidencia de competencias logradas

A lo largo del desarrollo de este trabajo se ha evidenciado la adquisición y consolidación de competencias clave en el ámbito académico. En primer lugar, se ha fortalecido la gestión de recursos académicos, desde la búsqueda de fuentes fiables y actualizadas hasta la correcta aplicación de normas de citación, junto con la comprensión crítica de documentos especializados. Cientos de referencias a trabajos del ámbito del aprendizaje automático y de la antropología forense han servido de base para construir un marco teórico sólido y fundamentar adecuadamente las decisiones metodológicas adoptadas.
Esta competencia, unida a los conocimientos básicos de programación y al manejo de modelos avanzados de \textit{Machine Learning}, ha resultado esencial para la implementación de las técnicas de predicción conformal.
Asimismo, he reforzado la capacidad de elaborar y maquetar la memoria del proyecto con un formato claro, estructurado y profesional.
Por último, destaco la competencia desarrollada en la gestión de un proyecto individual, que ha requerido planificación, organización del tiempo y toma de decisiones autónomas ---aunque en algunos casos guiadas por los tutores--- orientadas a alcanzar los objetivos planteados.


% ------------------------------------------------------------------------------------------------------------
% ------------------------------------------------------------------------------------------------------------

\section{Trabajos futuros}


Aún queda por explorar un \textbf{análisis de estas herramientas con otros conjuntos de datos}, preferiblemente más equilibrados y con mayor diversidad en los rangos de edad, para un análisis más rico.

Una de las virtudes más destacables de la predicción conformal es su inherente flexibilidad, que permite mejorar sus capacidades mediante su integración sinérgica con otros marcos metodológicos. Esta versatilidad abre la vía para el desarrollo de sistemas de \textit{Machine Learning} más robustos y confiables. En concreto, su potencial se puede ampliar en varias direcciones:

\begin{itemize}

    \item \textbf{Integración con otros paradigmas de cuantificación de la incertidumbre}: La predicción conformal puede combinarse con métodos como \textit{Monte Carlo Dropout} (como en \cite{bethell2024}) o la regresión cuantílica con modelos \textit{ensembles} para generar intervalos conformales que no solo garantizan una cobertura marginal, sino que también se benefician de una estimación de incertidumbre más afinada. 
    
    \item \textbf{Combinación con técnicas de detección de datos fuera de distribución (\textit{Out-of-Distribution}, OOD) y explicabilidad (\textit{Explainable Artificial Intelligence}, XAI)}: La unión de estas áreas es fundamental para construir sistemas confiables. 
    La detección de OOD es relevante dado que la suposición más importante realizada por la predicción conformal es que los datos son intercambiables y, por tanto, pertenecientes a una misma distribución subyacente. Cuando esta premisa fundamental se viola, las garantías de cobertura estadística dejan de ser válidas. Este mecanismo actuaría como un sistema de alerta temprana, identificando ejemplos para los cuales las predicciones, y sus intervalos de incertidumbre asociados, deben ser tratados con precaución.
    
    Por su parte, las técnicas de explicabilidad (XAI) aportan transparencia al proceso, permitiendo comprender las razones detrás de la incertidumbre cuantificada. Por ejemplo, XAI puede revelar si un intervalo amplio se debe a la escasez de datos de entrenamiento en una región específica (incertidumbre epistémica) o a una alta variabilidad inherente en la característica objetivo (incertidumbre aleatoria). Esta distinción es crucial para tomar decisiones informadas: en el primer caso, se podría resolver recopilando más datos específicos, mientras que en el segundo, la incertidumbre es inherente al problema. Así, la combinación de predicción conformal, detección de OOD y XAI no solo identifica cuándo una predicción es incierta, sino también por qué lo es, permitiendo a los expertos (p.ej., antropólogos forenses) evaluar la credibilidad de los intervalos conformales en contextos de toma de decisiones críticas.

\end{itemize}

Por último, el \textbf{aprovechamiento de la información experta del dominio para mejorar el análisis de la cobertura} representa una dirección fundamental. De esta forma, al potencial técnico de la herramienta le acompañe una interpretación significativa y rigurosa, permitiendo a los antropólogos forenses realizar un análisis mucho más rico y preciso, analizando tendencias y certidumbres por subpoblaciones, y trasladando así los resultados abstractos del modelo a conclusiones prácticas y accionables en contextos forenses reales.