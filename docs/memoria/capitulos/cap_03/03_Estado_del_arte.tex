\chapter{Estado del arte}

Analicemos

% ------------------------------------------------------------------------------------------------------------

\section{Estimación de la edad en antropología forense}

En ausencia de documentación escrita confiable, los métodos más precisos para estimar la edad se basan en el 
análisis del estado de los huesos del cuerpo humano. Los huesos experimentan cambios continuos a lo largo de 
la vida, y estas transformaciones progresivas permiten determinar la \textit{edad fisiológica} o biológica de 
un individuo. Esta edad refleja la etapa de desarrollo en la que se encuentra el esqueleto dentro del proceso 
de cambios que ocurren desde el nacimiento hasta la vejez \cite{byers2023}.

Cabe destacar que la edad biológica no siempre coincide con la edad cronológica ---el tiempo transcurrido 
desde el nacimiento---, pero ambas guardan una correlación significativa, lo que permite aproximaciones 
razonables en contextos forenses, antropológicos o médicos.

Las técnicas de estimación de edad presentan diferencias significativas en individuos maduros e inmaduros
\cite{ubelaker2019}. La diferencia radica en el grado de desarrollo esquelético y dental: en inmaduros, el 
esqueleto y la dentición no están completamente formados, por lo que los métodos se basan en patrones de 
crecimiento y osificación; en contraste, en maduros (con desarrollo completo), las técnicas se enfocan en 
cambios degenerativos, como el deterioro articular o la pérdida ósea.
% Además, las técnicas también dependen de si hay accesibilidad directa a los huesos del individuo, o si, por
% defecto, se deben recurrir a técnicas no invasivas como imágenes médicas (radiografías, resonancias magnéticas
% o tomografías computerizadas). 

% ------------------------------------------------------------------------------------------------------------

La estimación en cuerpos subadultos se basan en: el desarrollo y erupción dental
\footnote{
    La erupción dental es el proceso natural mediante el cual los dientes se desplazan desde el interior del 
    hueso maxilar o mandibular hasta alcanzar su posición definitiva en la boca, atravesando las encías.
}, 
los tiempo de aparición y cambios en la morfología de centros de osificación
\footnote{
    La osificación es el proceso natural mediante el cual el cartílago o tejido conectivo se convierte en 
    hueso. Los centros de osificación son regiones específicas del esqueleto donde comienza el proceso de 
    formación ósea durante el desarrollo embrionario, fetal, infantil y adolescente.
},
y los tiempos de fusión de los centros primarios (también denominados diáfisis) y secundarios (epífisis) 
\cite{scheuer2000, adserias2019}. 
Los métodos de mayor precisión se basan en el desarrollo dental, dado que estos, para una determinada
edad cronológica, muestran menor variabilidad que el esqueleto \cite{bowman1992}. En ausencia de estos, 
se recurre a epífisis, cuya formación y fusión son clave para la estimación de la edad esquelética 
\cite{adserias2019}.

La valoración en adultos es más compleja, dado que el desarrollo de la dentadura se ha completado, 
así como el crecimiento del esqueleto ha cesado \cite{byers2023}, por lo que los indicadores 
se basan más en características del deterioro óseo; pero la variabilidad de estas aumenta con la 
edad debido al efecto acumulativo de las influencias ambientales \cite{ubelaker2018, scheuer2004} 
\footnote{
    Por ejemplo, artículos como \cite{merritt2015,wescott2015} indican que la obesidad puede causar que se 
    sobreestime la edad del cuerpo, mientras que personas con una complexión más ligera o bajo peso corporal 
    tienden a presentar una infraestimación de la edad.
}. 
Actualmente, se analiza el proceso degenerativo de sínfisis púbica combinado con el análisis de transparencia 
de raíces caninas (de acuerdo a ), siguiendo el proceso de dos pasos propuesto en ...

Sin embargo, cuando la estimación de edad se realiza a una personas vivas, no se tiene acceso a sus 
huesos de forma directa. En estos casos, que en su mayoría van dirigidos a determinar la mayoría o menoría de 
edad de una persona, el análisis se realiza sobre imágenes médicas, como radiografías de manos, radiografías 
panorámicas de los maxilares o tomografías computerizadas de cortes finos de la epífisis mediales de las 
clavículas \cite{schmeling2016}. Se suelen combinar múltiples métodos para una mayor exactitud en la 
predicción. Dependiendo de los asuntos legales, se requerirá la estimación de la edad mínima del individuo o 
su edad más probable.




La Unión Europea (UE) recomienda, en caso de individuos vivos, realizar una entrevista previa que pueda 
esclarecer futuras hipótesis


% ------------------------------------------------------------------------------------------------------------

\section{Estimación de la edad en antropología forense usando \textit{machine learning}}

Los métodos manuales de estimación del perfil biológico se basan en la evaluación visual y en el análisis 
morfométrico de rasgos esqueléticos. Sin embargo, su aplicación demanda conocimiento especializado, pueden 
presentar ambigüedades en su formulación que den lugar a interpretaciones variables \cite{berst2001}, y están 
sujetos a posibles errores de medición \cite{langley2018}, sesgando el proceso y reduciendo su fiabilidad.

Estas restricciones impulsaron el desarrollo de métodos automatizados, como BoneXpert  
\cite{thodberg2008} que, empleando técnicas de ML tradicionales, mostraba un buen desempeño en individuos de 
distintos grupos geográficos y con distintas configuraciones clínicas para la toma de imágenes
\cite{van2009, martin2010, thodberg2010}.

Sin embargo, el paradigma del aprendizaje extremo a extremo (\textit{end-to-end learning}), impulsado por el
auge del \textit{deep learning}, ha permitido a los modelos aprender automáticamente tanto la extracción de 
características como la regresión de la edad a partir de los datos en bruto (\textit{raw data}). Este avance 
no solo facilita mayores niveles de abstracción, sino también predicciones más precisas en la estimación 
de edad \cite{kim2017, larson2018, lee2017}. 


% ------------------------------------------------------------------------------------------------------------

\section{Cuantificación de incertidumbre en estimación del perfil biológico}

Las técnicas de ML son más adecuadas para evaluar métodos, ya que, a diferencia de los manuales, su 
variabilidad puede controlarse mejor cuando se garantizan condiciones estandarizadas y un proceso riguroso 
en la captación de datos e imágenes.


