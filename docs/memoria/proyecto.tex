\documentclass[a4paper,11pt,openany]{book}
\setlength{\parskip}{0.5em plus 0.2em minus 0.1em}
% \usepackage[a4paper, top=2.7cm, bottom=4.7cm, inner=4.6cm, outer=3.1cm]{geometry}
\usepackage[a4paper, top=2.3cm, bottom=3cm, inner=3.7cm, outer=2.6cm]{geometry}
%\documentclass[a4paper,twoside,11pt,titlepage]{book}
\usepackage{listings}
\usepackage{enumitem}
\usepackage[utf8x]{inputenc}
\usepackage[spanish]{babel}

%\usepackage[style=list, number=none]{glossary} %
%\usepackage{titlesec}
%\usepackage{pailatino}

\decimalpoint
\usepackage{dcolumn}
\newcolumntype{.}{D{.}{\esperiod}{-1}}
\makeatletter
\addto\shorthandsspanish{\let\esperiod\es@period@code}
\makeatother


%\usepackage[chapter]{algorithm}
\RequirePackage{verbatim}
%\RequirePackage[Glenn]{fncychap}
\usepackage{fancyhdr}
\usepackage{graphicx}
\usepackage{afterpage}
\usepackage{longtable}



% 
\usepackage[style=ieee, citestyle=numeric-comp, backend=biber, backref=true, maxnames=99]{biblatex}
\DefineBibliographyStrings{spanish}{
   backrefpage = {Citado en pág.},
   backrefpages = {Citado en págs.},
   url = {URL},
}

%
\usepackage{xpatch}
\DeclareFieldFormat{backrefparens}{\addperiod\hspace*{1pt} [#1]}
\xpatchbibmacro{pageref}{parens}{backrefparens}{}{}

%
% \appto{\bibsetup}{\raggedright}
\appto{\bibsetup}{\sloppy}
\addbibresource{bibliografia.bib}

% Para definir colores
\usepackage{xcolor} 

%
\usepackage[colorlinks=true, linkcolor=blue, citecolor=blue, pdfborder={0 0 0}]{hyperref}

% Para permitir romper URLs largas
\usepackage{xurl}

% Para 
\usepackage{tikz}

% Para incluir fuentes tipográficas para ecuaciones
\usepackage{amsfonts}

% Para permitir introducir subfiguras dentro de figura
\usepackage{subcaption}

%
\usepackage[colorinlistoftodos]{todonotes}

%
\usepackage{booktabs}
\usepackage{multirow}
\usepackage[normalem]{ulem}
\useunder{\uline}{\ul}{}
\addto\captionsspanish{
  \renewcommand{\tablename}{Tabla}
  \renewcommand{\listtablename}{Índice de tablas}
}

\usepackage{amsmath}

\usepackage{float}

\usepackage{eurosym}

\usepackage{pgfgantt}

\usepackage{pdflscape}

\usepackage{wrapfig}

% ********************************************************************
% Re-usable information
% ********************************************************************
\newcommand{\myTitle}{Cuantificación de la incertidumbre de las predicciones de modelos de aprendizaje automático en problemas de estimación del perfil biológico\xspace}
\newcommand{\myDegree}{Grado en Ingeniería Informática\xspace}
\newcommand{\myName}{David González Durán\xspace}
\newcommand{\myProf}{Pablo Mesejo Santiago\xspace}
\newcommand{\myOtherProf}{Javier Venema Rodríguez\xspace}
\newcommand{\myFaculty}{Escuela Técnica Superior de Ingenierías Informática y de Telecomunicación\xspace}
\newcommand{\myFacultyShort}{E.T.S. de Ingenierías Informática y de Telecomunicación\xspace}
\newcommand{\myDepartment}{Departamento de Ciencias de la Computación e Inteligencia Artificial\xspace}
\newcommand{\myUni}{\protect{Universidad de Granada}\xspace}
\newcommand{\myLocation}{Granada\xspace}
\newcommand{\myTime}{\today\xspace}
\newcommand{\myVersion}{Version 0.1\xspace}


\hypersetup{
   pdfauthor   = {\myName (davgondur2910@correo.ugr.es)},
   pdftitle    = {\myTitle},
   pdfsubject  = {},
   pdfkeywords = {palabra_clave1, palabra_clave2, palabra_clave3, ...},
   pdfcreator  = {LaTeX con el paquete ....},
   pdfproducer = {pdflatex}
}

%\hyphenation{}


%\usepackage{doxygen/doxygen}
%\usepackage{pdfpages}
\usepackage{url}
\usepackage{colortbl,longtable}
\usepackage[stable]{footmisc}
\usepackage{index}

\usepackage{rotating}

\usepackage{placeins}

\usepackage{caption}


\makeindex
% \usepackage[style=long, cols=2,border=plain,toc=true,number=none]{glossary}
\usepackage[acronym]{glossaries}
\makeglossaries

\usepackage{mdframed}

\definecolor{LightGrayBox}{HTML}{EAEAEA}
% Definición del estilo que ya tienes
\mdfdefinestyle{StatisticsStyle}{%
   skipabove=0.5cm,
   skipbelow=0.5cm,
   % topline=false,
   roundcorner=10pt,
   frametitlefont={\color{black}\normalfont\bfseries},
   frametitlerule=true,
   frametitlealignment =\raggedright\noindent,
   frametitlebackgroundcolor=LightGrayBox,
   backgroundcolor=LightGrayBox,
   linecolor=LightGrayBox,
   fontcolor=black, 
}
\newmdenv[style=StatisticsStyle]{Statistics} 


% --- Contador para Statistics ---
\newcounter{statistics}[chapter]
\renewcommand{\thestatistics}{\thechapter.\arabic{statistics}}

% --- Nuevo entorno que añade numeración y referencia ---
\newenvironment{StatisticsRef}[2][]{%
    \refstepcounter{statistics} % permite \label
    \begin{Statistics}[frametitle={Análisis estadístico \thestatistics: #2}]%
    \ifx&#1&\else\label{#1}\fi
}{%
    \end{Statistics}%
}





% Definición de comandos que me son útiles:
%\renewcommand{\indexname}{Índice alfabético}
%\renewcommand{\glossaryname}{Glosario}



\pagestyle{fancy}
\fancyhf{}
\fancyhead[LO]{\leftmark}
\fancyhead[RE]{\rightmark}
\fancyhead[RO,LE]{\textbf{\thepage}}
\renewcommand{\chaptermark}[1]{\markboth{\textbf{#1}}{}}
\renewcommand{\sectionmark}[1]{\markright{\textbf{\thesection. #1}}}

\setlength{\headheight}{1.5\headheight}

\newcommand{\HRule}{\rule{\linewidth}{0.5mm}}
%Definimos los tipos teorema, ejemplo y definición podremos usar estos tipos
%simplemente poniendo \begin{teorema} \end{teorema} ...
\newtheorem{teorema}{Teorema}[chapter]
\newtheorem{ejemplo}{Ejemplo}[chapter]
\newtheorem{definicion}{Definición}[chapter]

\definecolor{gray97}{gray}{.97}
\definecolor{gray75}{gray}{.75}
\definecolor{gray45}{gray}{.45}
\definecolor{gray30}{gray}{.94}

\lstset{ frame=Ltb,
     framerule=0.5pt,
     aboveskip=0.5cm,
     framextopmargin=3pt,
     framexbottommargin=3pt,
     framexleftmargin=0.1cm,
     framesep=0pt,
     rulesep=.4pt,
     backgroundcolor=\color{gray97},
     rulesepcolor=\color{black},
     %
     stringstyle=\ttfamily,
     showstringspaces = false,
     basicstyle=\scriptsize\ttfamily,
     commentstyle=\color{gray45},
     keywordstyle=\bfseries,
     %
     numbers=left,
     numbersep=6pt,
     numberstyle=\tiny,
     numberfirstline = false,
     breaklines=true,
   }
% minimizar fragmentado de listados
\lstnewenvironment{listing}[1][]
   {\lstset{#1}\pagebreak[0]}{\pagebreak[0]}

\lstdefinestyle{CodigoC}
   {
	basicstyle=\scriptsize,
	frame=single,
	language=C,
	numbers=left
   }
\lstdefinestyle{CodigoC++}
   {
	basicstyle=\small,
	frame=single,
	backgroundcolor=\color{gray30},
	language=C++,
	numbers=left
   }

 
\lstdefinestyle{Consola}
   {basicstyle=\scriptsize\bf\ttfamily,
    backgroundcolor=\color{gray30},
    frame=single,
    numbers=none
   }


\newcommand{\bigrule}{\titlerule[0.5mm]}


%Para conseguir que en las páginas en blanco no ponga cabecerass
\makeatletter
\def\clearpage{%
  \ifvmode
    \ifnum \@dbltopnum =\m@ne
      \ifdim \pagetotal <\topskip
        \hbox{}
      \fi
    \fi
  \fi
  \newpage
  \thispagestyle{empty}
  \write\m@ne{}
  \vbox{}
  \penalty -\@Mi
}
\makeatother



\newacronym{AF}{AF}{Antropología Forense}
\newacronym{ID}{ID}{Identificación Humana}
\newacronym{PB}{PB}{Perfil Biológico}
\newacronym{IA}{IA}{Inteligencia Artificial}
\newacronym{ML}{ML}{Machine Learning}
\newacronym{CP}{CP}{Conformal Prediction}
\newacronym{CNN}{CNN}{Convolutional Neural Network}
\newacronym{SSH}{SSH}{Secure Shell}
\newacronym{DL}{DL}{Deep Learning}
\newacronym{DNN}{DNN}{Deep Neural Network}
\newacronym{FC}{FC}{Fully Connected}
\newacronym{MLP}{MLP}{Multilayer Perceptron}
\newacronym{UQ}{UQ}{Uncetainty Quantification}

\newacronym{ICP}{ICP}{Inductive Conformal Prediction}
\newacronym{QR}{QR}{Ruantile Regression}
\newacronym{CQR}{CQR}{Conformalized Quantile Regression}

\newacronym{LAC}{LAC}{Least-Ambiguous set-valued Classifiers}
\newacronym{MCM}{MCM}{Mondrian Confidence Machine}
\newacronym{APS}{APS}{Adaptive Prediction Sets}
\newacronym{RAPS}{RAPS}{Regularized Adaptive Prediction Sets}
\newacronym{SAPS}{SAPS}{Sorted Adaptive Prediction Sets}

\newacronym{NC}{NC}{No Conformidad}

% \newacronym{MAE}{MAE}{Mean Absolute Error}
% \newacronym{MSE}{MSE}{Mean Squared Error}
% \newacronym{EC}{EC}{Empirical Coverage}

\newacronym{OOD}{OOD}{Out-Of-Distribution}

\renewcommand*{\acronymfont}[1]{\textcolor{black}{#1}}

%\acrshort{}

\usepackage{pdfpages}
\begin{document}

   \frontmatter
   \begin{titlepage}
\thispagestyle{empty}

\begin{center}
\includegraphics[width=0.9\textwidth]{portada/imagenes/logo_ugr.jpg}\\[1.4cm]

\textsc{ \Large TRABAJO FIN DE GRADO\\[0.2cm]}
\textsc{ GRADO EN INGENIERÍA INFORMÁTICA}\\[1cm]

{\LARGE \bfseries Cuantificación de la incertidumbre de las predicciones de modelos de aprendizaje automático en problemas de estimación del perfil biológico\\}
\noindent\\[3.5ex]
\end{center}

\vspace{0.5cm}

\begin{center}
\textbf{Autor}\\ {David González Durán}\\[2.5ex]
\textbf{Director}\\ {Pablo Mesejo Santiago}\\[2.5ex]
\textbf{Mentor}\\{Javier Venema Rodríguez}\\[2cm]
\includegraphics[width=0.3\textwidth]{portada/imagenes/etsiit_logo.png}\\[0.1cm]
\textsc{Escuela Técnica Superior de Ingenierías Informática y de Telecomunicación}\\
\textsc{---}\\
Granada, 3 de septiembre de 2025
\end{center}

\end{titlepage}

   %\chapter*{}
\phantomsection
%\thispagestyle{empty}
%\cleardoublepage

%\thispagestyle{empty}

%\input{portada/portada_2}
%----------------------------------------------------------------------------------------------------------------------%
% PÁGINA DE RESUMEN EN ESPAÑOL

\cleardoublepage
\thispagestyle{empty}

\begin{center}
    {\large\bfseries Cuantificación de la incertidumbre de las predicciones de modelos de aprendizaje automático en problemas de estimación del perfil biológico}
\end{center}
\begin{center}
    David González Durán
\end{center}

\vspace{0.7cm}

\noindent\textbf{Palabras clave:} palabra\_clave1, palabra\_clave2, palabra\_clave3, ... 

\vspace{0.7cm}

\noindent\textbf{Resumen}

Lorem ipsum dolor sit amet, consectetur adipiscing elit, sed do eiusmod tempor incididunt ut labore et dolore magna aliqua. Ut enim ad minim veniam, quis nostrud exercitation ullamco laboris nisi ut aliquip ex ea commodo consequat. Duis aute irure dolor in reprehenderit in voluptate velit esse cillum dolore eu fugiat nulla pariatur. Excepteur sint occaecat cupidatat non proident, sunt in culpa qui officia deserunt mollit anim id est laborum.

%----------------------------------------------------------------------------------------------------------------------%
% PÁGINA DE RESUMEN EN INGLÉS

\cleardoublepage
\thispagestyle{empty}

\begin{center}
    {\large\bfseries Quantification of the uncertainty in machine learning model predictions for biological profile estimation problems}
\end{center}
\begin{center}
    David González Durán 
\end{center}

\vspace{0.7cm}

\noindent\textbf{Keywords:} Keyword1, Keyword2, Keyword3, ...

\vspace{0.7cm}

\noindent\textbf{Abstract}

Lorem ipsum dolor sit amet, consectetur adipiscing elit, sed do eiusmod tempor incididunt ut labore et dolore magna aliqua. Ut enim ad minim veniam, quis nostrud exercitation ullamco laboris nisi ut aliquip ex ea commodo consequat. Duis aute irure dolor in reprehenderit in voluptate velit esse cillum dolore eu fugiat nulla pariatur. Excepteur sint occaecat cupidatat non proident, sunt in culpa qui officia deserunt mollit anim id est laborum.


%----------------------------------------------------------------------------------------------------------------------%
% AUTORIZACIÓN

\cleardoublepage
\thispagestyle{empty}

\noindent\rule{\textwidth}{2pt}\par\vspace{4.5ex}

Yo, \textbf{David González Durán}, alumno de la titulación TITULACIÓN de la \textbf{Escuela Técnica Superior
de Ingenierías Informática y de Telecomunicación de la Universidad de Granada}, con DNI 32071015E, autorizo la
ubicación de la siguiente copia de mi Trabajo Fin de Grado en la biblioteca del centro para que pueda ser
consultada por las personas que lo deseen.

\vspace{6cm}

\noindent Fdo: David González Durán

\vspace{2cm}

\begin{flushright}
    Granada, a X de mes de 202.
\end{flushright}

%----------------------------------------------------------------------------------------------------------------------%
% INFORME DE LOS TUTORES

\cleardoublepage
\thispagestyle{empty}

\noindent\rule{\textwidth}{2pt}\par\vspace{4.5ex}

D. \textbf{Pablo Mesejo Santiago}, Profesor del Área de XXXX del Departamento de Ciencias de la Computación e Inteligencia Artificial de la Universidad de Granada.

\vspace{0.5cm}

D. \textbf{Javier Vénema Rodríguez}, Esdudiante de Doctorado del programa de de Tecnologías de la Información y de la Comunicación e investigador en Inteligencia Artificial en Panacea Cooperative Research.

\vspace{0.5cm}

\textbf{Informan:}

\vspace{0.5cm}

Que el presente trabajo, titulado \textit{\textbf{Cuantificación de la incertidumbre de las predicciones de modelos de aprendizaje automático en problemas de estimación del perfil biológico}},
ha sido realizado bajo su supervisión por \textbf{David González Durán}, y autorizamos la defensa de dicho trabajo ante el tribunal
que corresponda.

\vspace{0.5cm}

Y para que conste, expiden y firman el presente informe en Granada a X de mes de 2025.

\vspace{1cm}

\textbf{Los directores:}

\vspace{5cm}

\noindent \textbf{Pablo Mesejo Santiago \hfill Javier Vénema Rodríguez}

%----------------------------------------------------------------------------------------------------------------------%
% AGRADECIMIENTOS

\cleardoublepage
\chapter*{Agradecimientos}
\thispagestyle{empty}

\vspace{1cm}

Poner aquí agradecimientos...

   \tableofcontents
   \listoffigures
   \listoftables
   \printglossary[type=\acronymtype, title=Lista de Abreviaturas]
   
   \mainmatter
   
\chapter{Introducción}

% --------------------------------------------------------------------------------------------------------------

\section{Descripción del problema}

La antropología es la ciencia que estudia la humanidad en todas sus dimensiones: biológica, cultural, lingüística o arqueológica \cite{AAA2022AnthropologyDefinition}, a lo largo del tiempo y en distintas partes del mundo. La antropología biológica o física se centra en el estudio de la anatomía, el crecimiento, la adaptación y la evolución del cuerpo humano \cite{nawrocki2006}. Dentro de este campo, la \textbf{antropología forense (AF)} es el subcampo especializado que aplica métodos y técnicas antropológicas para resolver cuestiones médico-legales \cite{nawrocki2006}, empleando conocimientos de antropología física, aunque a veces también de la arqueología, para la correcta recuperación y análisis de la evidencia forense.Aunque tradicionalmente asociada al estudio de restos humanos esqueletizados o en descomposición, la AF también contribuye a la estimación del perfil biológico en individuos vivos, especialmente en contextos legales.

Tradicionalmente, los antropólogos forenses han tenido cinco principales objetivos en su trabajo \cite{byers2023}:

\begin{enumerate}
    
    \item Determinar el \textbf{perfil biológico} de un individuo (es decir, sexo, edad, estatura y ascendencia), ya sea en restos esqueletizados donde los tejidos blandos se han deteriorado hasta el punto de que estas características no pueden determinarse mediante inspección visual, o en personas vivas mediante técnicas no invasivas como análisis radiográficos o morfológicos. 

    \item Identificar la naturaleza de lesiones traumáticas (como heridas de bala, puñaladas o fracturas) en huesos humanos, así como sus causantes, con el objetivo de recopilar información sobre la causa y circunstancias de la muerte.

    \item Estimar el intervalo \textit{post mortem}, es decir, el tiempo transcurrido desde la muerte, gracias a su conocimiento sobre los procesos de descomposición corporal.
    
    \item Asistir en la localización, recuperación y conservación de los restos (superficiales o enterrados) aplicando técnicas arqueológicas, garantizando la recolección de toda la evidencia forense relevante.

    \item Proporcionar información clave para la \textbf{identificación} de los fallecidos, basándose en las características distintivas de los esqueletos.

\end{enumerate}

Además de estos roles, en la actualidad los antropólogos desempeñan otros trabajos que no están relacionados con el ámbito criminalístico. Entre ellos, uno de sus campos de acción más relevantes es la \textbf{identificación de víctimas en contextos de catástrofes masivas} \cite{deBoer2019, prinz2007, beauthier2009}, como accidentes aéreos, ataques terroristas o desastres naturales, donde los restos suelen estar mutilados o desfigurados.

Su labor también es fundamental en la \textbf{recuperación e identificación de violaciones sistemáticas de derechos humanos}, como exterminios, persecuciones políticas y represiones dictatoriales \cite{skinner2003}. Casos como la Guerra Civil Española y la Dictadura Franquista \cite{sanchisgimeno2024, baeta2015}, así como las múltiples dictaduras en el Cono Sur de América \cite{ataliva2024}, han requerido la intervención de equipos forenses para esclarecer la verdad histórica y restituir la identidad de las víctimas a sus familiares, contribuyendo al proceso de memoria, justicia y reparación para las familias afectadas. Esta vinculación con la justicia trasciende lo nacional: la ciencia forense es clave en la \textbf{investigación de crímenes de guerra contra poblaciones civiles}. Organizaciones como Médicos por los Derechos Humanos y la ONU financian equipos especializados que documentan estos crímenes, proporcionando pruebas esenciales para tribunales internacionales \cite{tanaka2020}.

Y por último, también son fundamentales para \textbf{estimar la edad de personas vivas en casos legales}, especialmente cuando no existen registros confiables. Esto ocurre, por ejemplo, en casos de solicitudes de asilo, adopciones internacionales o procesos judiciales donde es necesario determinar si una persona es menor o mayor de edad, lo cual puede tener importantes implicaciones legales. Según el tipo de procedimiento, se puede requerir tanto la estimación de la edad mínima como la edad más probable del individuo, con el fin de priorizar la protección de los menores, evitando que queden expuestos a violaciones de sus derechos.


\subsection{Identificación humana y estimación del perfil biológico}

Como hemos visto, la \textbf{identificación humana (ID)} es una de las principales tareas que aborda la AF. Consiste en la determinación y verificación de la identidad de una persona en base a \cite{thompson2006}: evidencias circunstanciales (hora y lugar del descubrimiento del cuerpo, efectos personales, confirmación visual por parte de familiares y amigos); y evidencias físicas, obtenidas a través de examinación externa de características como el sexo, color de piel, tatuajes, o huellas dactilares, o, cuando estas no estén disponibles, mediante examinación interna con técnicas médico-científicas, donde se aplican técnicas de antropología y genética forense.

Cabe destacar que, aunque los análisis dactilares y genéticos superan en precisión identificativa a los métodos antropológicos, su aplicabilidad enfrenta limitaciones técnicas significativas que condicionan su uso en ciertos contextos forenses \cite{beauthier2009}. Las huellas dactilares requieren de: tejido blando preservado, lo que es común en cadáveres frescos, pero se pierde con la descomposición o la carbonización; y una base de datos que incluya la huella del individuo en vida (registros \textit{ante mortem}). Por otro lado, en cuanto al análisis genético, este puede verse comprometido por una mala conservación del ADN que puede deberse a su degradación o contaminación. La concentración presente en un cadáver se reduce drásticamente en los primeros 8 meses \textit{post mortem} \cite{higgins2015}, y factores como las altas temperaturas, la exposición a humedad ambiental o la presencia de aguas subterraneas y entornos ricos en oxígeno, que fomentan la presencia microbiana, perjudican la conservación del ADN \cite{latham2018}. Y, aún extraída una secuencia válida de ADN, se necesita de muestras con las que compararla, a ser posible de familiares de primer grado, para establecer una identificación concluyente. 

Por tanto, la AF contribuye al problema de identificación humana en dos escenarios \cite{swganth2010}:  

\begin{enumerate}

    \item Cuando los otros métodos no son viables, dado que las pruebas no se puedan recoger o no sean válidas, o no haya registros con los que compararlas.
    
    \item Como apoyo a otras técnicas de identificación. Por ejemplo, las técnicas de estimación del perfil biológico pueden reducir el grupo de posibles coincidencias en bases de datos genéticos, facilitando el cotejo de secuencias genéticas y reduciendo el coste del proceso.  

\end{enumerate}

La \textbf{estimación del perfil biológico (PB)} es, por tanto, un proceso fundamental de la AF, en el cual se determinan características biológicas clave de un individuo \cite{byers2023}: 

\begin{itemize}

    \item \textbf{sexo}, mediante el análisis morfológico y métrico de rasgos sexuales en el esqueleto, 
    especialmente en la pelvis y el cráneo;
    
    \item \textbf{edad}, estimada a partir de cambios morfológicos y de desarrollo en el esqueleto, pudiendo referirse tanto a la \textbf{edad al momento de la muerte} en restos óseos, como a la \textbf{edad cronológica}%
    \footnote{
        La edad cronológica es la edad real de una persona desde su nacimiento, mientras que la edad biológica o fisiológica refleja la condición fisiológica del cuerpo \cite{marcante2025}.
    }
    en personas vivas en contextos forenses o humanitarios;
    
    \item \textbf{estatura}, mediante la estimación de la talla a partir de longitudes óseas, particularmente de los huesos largos; y
    
    \item \textbf{ascendencia} o \textbf{afinidad poblacional}, analizando variaciones craneométricas y morfológicas asociadas a poblaciones o grupos geográficos (actualmente en revisión \cite{ross2021a, ross2021b, flouri2022}).

\end{itemize}

En los problemas de ID, cuando estas características biológicas coinciden con los registros \textit{ante mortem}, se fortalece la hipótesis de identificación; en cambio, si existen una o más discrepancias ---especialmente de alguna característica firme como múltiples epífisis no fusionadas, que no pueden ocurrir en un adulto mayor---, el individuo es excluido como posible coincidencia \cite{byers2023}. En la Figura \ref{fig:SFI_pipeline} podemos observar que la estimación del PB es uno de los primeros pasos en el proceso de ID forense. 

\begin{figure}[h]
    \centering
    \includegraphics[width=\textwidth]{capitulos/cap_01/imagenes/SFI_pipeline.png}
    \caption{Procedimiento secuencial para la identificación forense basada en el esqueleto humano (\textit{skeleton-based forensic identification}) \cite{mesejo2020}.} 
    \label{fig:SFI_pipeline}
\end{figure}

La estimación del PB en restos humanos es una tarea compleja, especialmente cuando se estima la edad en el momento de la muerte, ya que hay diferentes métodos a aplicar dependiendo de la fase de desarrollo del individuo. Las variaciones en la morfología de los huesos son bien conocidas, pero estas no siempre ocurren al mismo tiempo en diferentes individuos, ya que no están expuestos a las mismos condiciones genéticas y del entorno.

Además, como se ha mencionado anteriormente, la estimación de edad también se realiza sobre personas vivas en casos legales donde la edad es un factor determinante \cite{schmeling2016}, por ejemplo, con menores migrantes no acompañados. En estos casos no se tiene acceso a los huesos de la persona de forma directa, por lo que el análisis se realiza sobre imágenes médicas.

% ------------------------------------------------------------------------------------------------------------
% ------------------------------------------------------------------------------------------------------------
% ------------------------------------------------------------------------------------------------------------

\section{Motivación}

Los métodos de estimación del PB se basan en la evaluación visual y en el análisis morfométrico de rasgos esqueléticos, que requieren de conocimiento especializado. Sin embargo, su aplicación puede presentar ambigüedades en su formulación que den lugar a interepretaciones variables ---muchas veces fruto de sesgos cognitivos \cite{nakhaeizadeh2014, cooper2019}--- y están sujetos a posibles errores de medición \cite{langley2018}. Además, la gran variabilidad genética y ambiental entre individuos, que afecta la morfología del esqueleto y genera diferencias significativas entre poblaciones de distintas regiones \cite{ubelaker2017}, hace que muchos de estos métodos ---basados en muestras de referencia limitadas o no representativas de la diversidad humana global--- pierdan precisión. Esto puede introducir sesgos al estimar el PB de individuos de grupos poco estudiados o con características atípicas.

Frente a estas limitaciones, recientes avances en inteligencia artificial (IA) y \textit{machine learning} (ML) han demostrado el potencial de mejorar la exactitud y objetividad de estimación del PB, tanto para la estimación de sexo \cite{curate2017, darmawan2015, pinto2016} como de edad \cite{kim2017, larson2018, lee2017}.

Sin embargo, aún mejorando la exactitud de las predicciones, los modelos siguen mostrando carencias respecto a la cuantificación de incertidumbre, pues no todas las predicciones tienen el mismo nivel de confianza o fiabilidad. Ya en \cite{ferrante2009} se introducía no solo la necesidad de identificar el método adecuado para estimar la edad a partir de los elementos disponibles, sino también de evaluar su confiabilidad y realizar un estudio del error arrojado por las predicciones del método. Estos generalmente se han basado en la estadística frecuentista%
\footnote{
    La estadística frecuentista es la corriente que se desarrolla a partir de los conceptos de probabilidad y que se centra en el cálculo de probabilidades y el contraste de hipótesis.
}
\cite{verma2020, stepanovsky2024, heinrich2024}. Un ejemplo de este tipo de análisis se ilustra en la Figura \ref{fig:regression_lentibia_stature}, donde se examina la distribución probabilística del error residual arrojado por el modelo de regresión propuesto en \cite{verma2020}.

\begin{figure}[h]
    \centering
    \includegraphics[width=0.7\textwidth]{capitulos/cap_01/imagenes/regression_line_lentibia_stature.png}
    \caption[
        Línea de regresión del modelo de regresión propuesto en \cite{verma2020} que predice la estatura a partir de la longitud de la tibia.
    ]{
        Línea de regresión del modelo de regresión propuesto en \cite{verma2020} que predice la estatura a partir de la longitud de la tibia. En rojo, la línea de regresión; en verde, la línea de los intervalos de confianza del 95\%; y en naranja, la línea de los intervalos de predicción al 95\% de confianza.
    } 
    \label{fig:regression_lentibia_stature}
\end{figure}

Aunque existen métricas para evaluar el error cuando se dispone de \textit{ground truth}, la mayoría de los modelos actuales se limitan a ofrecer predicciones puntuales en regresión \cite{park2024, imaizumi2021, stepanovsky2024} o etiquetas únicas en clasificación \cite{venema2022, park2024}, sin cuantificar la incertidumbre asociada a cada predicción.

Con lo anterior se expone la motivación de la aplicación de ML a la AF, así como de la necesidad de cuantificar la incertidumbre en las predicciones, para ofrecer garantías de confiabilidad estadística que aspiren a sustentar la validez legal en contextos judiciales. Algunos datos que magnifican la necesidad de técnicas de AF confiables actualmente son:

% 1/3 de los muertos del 11S sin identificar

\begin{itemize}

    \item En los últimos años, ha aumentado significativamente el número de cadáveres hallados en el territorio español, como podemos apreciar en la Figura \ref{fig:evolucion_hallazgosID_cadaveres} \cite{interior2025desaparecidos}. En 2024 se ha alcanzado una cifra record, ---en gran parte debido a las inundaciones de la DANA Valencia---, de 531 cadáveres en 2024, de los cuales se pudo identificar a 323.

    \begin{figure}[h]
        \centering
        \includegraphics[width=\textwidth]{capitulos/cap_01/imagenes/hallazgos_cadaveres.png}
        \caption[
            Evolución de hallazgos/identificación de cadáveres en España (2010-2024) \cite{interior2025desaparecidos}.
        ]{
            Evolución de hallazgos/identificación de cadáveres en España (2010-2024) \cite{interior2025desaparecidos}. 
        } 
        \label{fig:evolucion_hallazgosID_cadaveres}
    \end{figure}

    \item En 2020, de las 2.457 fosas totales documentadas de la Guerra Civil y el franquismo, aún 1.221 seguían sin ser intervenidas y se estimaba que ``con una intervención oficial del Estado podrían recuperarse unos 20 a 25.000 individuos'' e identificar ``entre 5 y 7.000 de ellos'', estimándose  necesario contar con unos 40-50 profesionales de la antropología forense \cite{etxeberria2020}. 

    % \item De acuerdo con UNICEF \cite{unicef2013}, en 2012 cerca de 230 millones de niños menores de cinco años no contaban con un registro oficial de nacimiento. Las regiones con las tasas más bajas de registro incluyen África subsahariana (44\%) y el sur de Asia (39\%). Esta situación se agrava aún más, ya que muchos niños registrados no poseen un certificado de nacimiento, y los documentos existentes suelen perderse durante procesos de migración.

    \item En España, se ha registrado en la última década (2013-2023) un aumento significativo en la llegada de Menores Extranjeros No Acompañados \cite{fge2024,fge2019,fge2016,fge2013}, que ha disparado consigo el número de diligencias abiertas para la determinación de su edad, como se ve reflejado en la Figura \ref{fig:evolucion_DPDE}.

    \begin{figure}[h]
        \centering
        \includegraphics[width=\textwidth]{capitulos/cap_01/imagenes/dpde_España.png}
        \caption{
            Evolución del número de diligencias preprocesales de determinación de edad abiertas en España (2011–2023). Elaboración propia a partir de \cite{fge2013,fge2016,fge2019, fge2024}.
        } 
        \label{fig:evolucion_DPDE}
    \end{figure}

    \item La relevancia de la ciencia forense en la identificación de víctimas y la protección de la dignidad humana ha convertido su aplicación en un pilar fundamental de los derechos humanos y la justicia internacional, naciendo así la  \textbf{acción forense humanitaria} \cite{cordner2017}. Esta disciplina emplea la ciencia forense con un propósito exclusivamente humanitario, con los objetivos de: identificar a las personas fallecidas, gestionar dignamente sus restos y aliviar el sufrimiento de sus familias en situaciones de conflicto, migración y desastres naturales \cite{tidballbinz2021}. 

\end{itemize}

% ------------------------------------------------------------------------------------------------------------

\section{Objetivos}

La \textbf{predicción conformal} emerge como un marco teórico robusto para generar intervalos de predicción con garantías estadísticas sólidas, independientemente de la distribución subyacente de los datos. A diferencia de los enfoques tradicionales, este método no solo ofrece predicciones puntuales sino que cuantifica la incertidumbre asociada a cada estimación mediante intervalos o conjuntos de predicción que reflejan la confiabilidad de la predicción en cada caso particular.

Este Trabajo de Fin de Grado tiene un doble objetivo: 

\begin{itemize}

    \item desde un prisma teórico, defender la cuantificación de incertidumbre como herramienta esencial en ML, ofrecer un panorama de métodos destacados, analizando sus ventajas y limitaciones, y centrarnos en la predicción conformal y sus técnicas más populares.

    \item aplicar la predicción conformal a un contexto práctico como es el problema de estimación del PB, centrándose en la estimación de edad y de sexo a partir de datos biológicos e imágenes médicas.

\end{itemize}

De esta forma, podremos incorporar la incertidumbre propia del problema a resolver y del modelo entrenado para él, para, en aquellos casos más confusos, devolver conjuntos de predicciones con más de una etiqueta predicha (p.ej., \{masculino, femenino\}) en problemas de clasificación, o intervalos de predicción más amplios (p.ej., edad$\in$[16,20]) en problemas de regresión, en ambos casos para un nivel de confianza determinado.

Por tanto, podemos desgranar los objetivos en:

% \todo{Pablo: Esto no creo que quede muy claro. Generalmente, se proponen uno o dos objetivos principales, y luego se presentan una serie de objetivos parciales, cuya consecución asegura el cumplimiento de los objetivos principales. Pero aquí no me queda claro si estos son los objetivos parciales... entiendo que sí. De ser así, no dudes en indicarlo con claridad, diciendo sin ambages que estos son objetivos parciales.}

\begin{itemize}

    \item Estudiar de forma exhaustiva la bibliografía sobre predicción conformal y sus diversas variantes, así como de la estimación de sexo y edad, centrando nuestra atención en el estado del arte.

    % \todo{David: Entonces falta apartado para la estimación de sexo en el estado del arte}

    \item Implementar, entrenar y validar modelos de regresión ---en problemas de estimación de edad--- y clasificación ---tanto en problema de estimación de sexo como edad legal--- a los que aplicar la inferencia conformal.

    \item Comparar los intervalos y conjuntos de predicciones generados para evaluar su calibración empírica, robustez ante datos ambiguos y utilidad forense, contrastándolos con métodos tradicionales (p.ej., intervalos de confianza clásicos).  

    \item Realiza una primera aproximación a un marco interpretable y con garantías estadísticas para la estimación del perfil biológico (véase un ejemplo práctico en la Figura \ref{fig:example_intervalic_estimation}), donde la incertidumbre cuantificada pueda integrarse en informes periciales bajo estándares jurídicos.

\end{itemize}

% \todo{
%     Pablo: Algo que yo creo que podría ser de utilidad al lector es mostrar un esquema genérico visual de lo que piensas hacer a nivel práctico, por ejemplo, a la hora de estimar la edad empleando una red neuronal. Yo incluiría un esquema de aprendizaje automático, como la Fig. 3 de https://link.springer.com/article/10.1007/s00521-022-07981-0, pero añadiendo elementos visuales que indiquen y subrayen la idea de que se emplean intervalos de predicción (con un conjunto de calibración, por ejemplo) para cuantificar la incertidumbre en las predicciones proporcionadas.    
%     Dicho de otro modo: queremos una figura que, de un vistazo, permita entender cómo vas a combinar estimación de la edad (por ejemplo) e intervalos de predicción (y estos últimos de dónde salen). Aunque luego los detalles se presenten más adelante, un diagrama en la introducción creo que ayudaría a aterrizar las ideas principales.
% }

\begin{figure}[h]
    \centering
    \includegraphics[width=\textwidth]{capitulos/cap_01/imagenes/example_intervalic_estimation.png}
    \caption[
        Diagrama de modelo de regresión que usa predicción conformal, el cual, además de proporcionar una estimación puntual del valor esperado, entrega un intervalo de predicción con un nivel de confianza del 95\%.
    ]{ 
        Diagrama de modelo de regresión que usa predicción conformal, el cual, además de proporcionar una estimación puntual del valor esperado, entrega un intervalo de predicción con un nivel de confianza del 95\%.
        Esta salida se lee de la siguiente manera: ``la edad esperada del individuo es de 16.1 años y, con una confianza del 95\%, la edad real del individuo está entre los 14.3 y los 17.6 años''.
    } 
    \label{fig:example_intervalic_estimation}
\end{figure}


En resumen, este trabajo pretende explorar la integración de marcos probabilísticos en la práctica forense que capturen la incertidumbre de los problemas, y facilitar el uso de la inferencia conformal en ellos. Este enfoque proporciona estimaciones calibradas de incertidumbre, con garantías estadísticas de contener el valor real en un conjunto o intervalo de predicción, útiles para la toma de decisiones fundamentadas en contextos prácticos donde la interpretabilidad y robustez son críticas.

% ------------------------------------------------------------------------------------------------------------

\section{Planificación temporal del proyecto}

Partimos de que el Proyecto de Fin de Grado tiene asignado 12 créditos ECTS, lo que equivale a 300 horas de trabajo. Estas 300 horas se distribuyen a lo largo del segundo cuatrimestre del curso 2024/2025, constando de 66 días lectivos, lo que resulta en una carga de trabajo de aproximadamente 4.54 horas al día. La planificación inicial del proyecto se presenta en el diagrama de Gantt de la Figura \ref{fig:inicial_gantt}.

\begin{sidewaysfigure}
    \centering
    \begin{ganttchart}[
        expand chart=\textwidth,
        hgrid,
        x unit=1.15mm,
        y unit chart=7mm,
        y unit title=12mm,
        time slot format=isodate,
        %
        bar/.append style={draw=black!50, fill=black!20},
        bar label font=\small,
        bar top shift = 0.2,
        bar height = 0.6,
        %
        group/.append style={draw=black, fill=green!50},
        %
        milestone/.append style={fill=orange, rounded corners=3pt},
        % group label font=\small\bfseries,    % texto de grupos más pequeño y en negrita
        % milestone label font=\footnotesize
    ]{2025-02-17}{2025-06-30}
        \gantttitlecalendar{year, month=shortname}\\
        \ganttbar{Definición de propuesta y objetivos}{2025-02-17}{2025-02-19} \\
        \ganttbar{Implementación del núcleo}{2025-02-20}{2025-03-09} \\
        \ganttbar{Implementación de algoritmos de CP}{2025-03-10}{2025-04-13} \\
        \ganttbar{Experimentación}{2025-04-14}{2025-06-08} \\
        \ganttbar{Análisis de resultados}{2025-06-09}{2025-06-22} \\
        \ganttbar{Redacción de la memoria}{2025-02-20}{2025-06-22}\\
        \ganttbar{Contingencia}{2025-06-23}{2025-06-30}
    \end{ganttchart}
    \caption{Diagrama de Gantt inicial del proyecto}
    \label{fig:inicial_gantt}
\end{sidewaysfigure}


No obstante, debido a ajustes en la orientación conceptual del trabajo, modificaciones en el enfoque metodológico y determinadas circunstancias personales de salud, fue necesario realizar cambios sobre la planificación inicial, posponiéndose finalmente la entrega del TFG al mes de septiembre.

\begin{sidewaysfigure}
    \centering
    \begin{ganttchart}[
        expand chart=\textwidth,
        hgrid,
        x unit=1.15mm,
        y unit chart=7mm,
        y unit title=12mm,
        time slot format=isodate,
        %
        bar/.append style={draw=black!50, fill=black!20},
        bar label font=\small,      % texto de tareas más pequeño
        bar top shift = 0.2,
        bar height = 0.6,
        %
        group/.append style={draw=black, fill=black!100},
        % group label font=\small\bfseries,    % texto de grupos más pequeño y en negrita
        %
        % milestone label font=\footnotesize
        milestone/.append style={draw=black, fill=black},
        milestone left shift =-1,
        milestone right shift =2,
        milestone inline label node/.append style={left=1ex,text opacity=1},
        flip/.style={milestone inline label node/.append style={right=2ex}},
    ]{2025-02-17}{2025-08-31}
        \gantttitlecalendar{year, month=shortname}\\
        \ganttbar{Definición de propuesta y objetivos}{2025-02-24}{2025-02-26} \\
        %
        \ganttbar{Implementación del núcleo}{2025-03-15}{2025-05-16}
        \ganttbar[inline]{Revisión}{2025-06-19}{2025-06-28} \\
        %
        \ganttbar{Implementación de técnicas de CP}{2025-02-27}{2025-03-14}
        \ganttbar[inline]{Regres.}{2025-05-28}{2025-07-06} 
        \ganttbar[inline]{Clasif.}{2025-07-07}{2025-08-01} \\
        %
        \ganttbar{Experimentación}{2025-07-14}{2025-08-03}
        \ganttbar[inline]{}{2025-06-15}{2025-07-06} \\
        %
        \ganttbar{Análisis de resultados}{2025-07-18}{2025-08-15}\\
        %
        \ganttbar{Redacción de la memoria}{2025-02-27}{2025-08-22} 
        \ganttmilestone[inline]{Cap.1}{2025-05-02} 
        \ganttmilestone[inline]{Cap.2}{2025-06-03} 
        \ganttmilestone[inline]{Cap.3-4}{2025-07-06}
        \ganttmilestone[inline]{Cap.5-6}{2025-08-22}
    \end{ganttchart}
    \caption{Diagrama de Gantt final del proyecto}
    \label{fig:final_gantt}
\end{sidewaysfigure}

La organización temporal del trabajo se desarrolló finalmente de la siguiente manera (véase la Figura \ref{fig:final_gantt}): 

\begin{itemize}

    \item Febrero: se redactó la propuesta del Trabajo de Fin de Grado, la cual fue aceptada en el plazo de una semana.
    
    \item Marzo: la primera parte del mes se destinó a la implementación de algunas técnicas de CP en clasificación(LAC, APS y RAPS) sobre el conjunto de datos CIFAR-10. A mediados de mes se concedió acceso al clúster SLURM, lo que permitió comenzar la organización del proyecto, la implementación del núcleo del código y la configuración del entorno de trabajo mediante un proceso iterativo de prueba y error orientado a obtener una solución más cómoda y flexible.
    
    \item Abril: se llevó a cabo la búsqueda, revisión, organización y análisis de bibliografía, con especial atención a la literatura en AF. A mediados de mes se obtuvo acceso al conjunto de datos definitivo que sería utilizado en el proyecto. Fue necesario adaptar el núcleo de código previamente desarrollado, dedicándose varias semanas a la experimentación y mejora del modelo de red neuronal convolucional (CNN). Se aprovechó para organizar el código y subirlo a un repositorio. 

    \item Mayo: a principios de mes se completó una primera versión del capítulo inicial (\textit{Introducción}). Durante la primera mitad del mes se continuó con la optimización del entrenamiento e inferencia del modelo, en paralelo con la redacción del capítulo de Fundamentos teóricos.

    \item Junio: a comienzos de mes se obtuvo la primera versión completa del capítulo segundo (\textit{Fundamentos teóricos}). En este momento se decidió centrar los esfuerzos en la parte de regresión, lo que condujo a la implementación de las técnicas de CP aplicadas a este contexto. Durante el proceso se identificó la necesidad de reformar nuevamente el núcleo del código, con el fin de reducir repeticiones y dotarlo de mayor flexibilidad, permitiendo mediante argumentos entrenar toda la red o únicamente la cabeza, seleccionar la técnica de CP, realizar inferencia, entre otras opciones. De forma paralela, se avanzó en la redacción del capítulo tercero (\textit{Estado del arte}) y del capítulo cuarto (\textit{Materiales y métodos}).

    \item Julio: a principios de mes se dispuso de una versión preliminar de la memoria que incluía la parte de regresión hasta la presentación de resultados, aunque sin discusión. Tras esto, el trabajo se orientó hacia la implementación de técnicas de CP para clasificación, en problemas derivados del enfoque de regresión. Asimismo, se desarrolló un sistema de almacenamiento más organizado de los resultados, de manera que pudieran ser cargados posteriormente en un cuaderno Jupyter para su análisis.

    \item Agosto: se discutieron los resultados obtenidos tanto en regresión como en clasificación, se completaron apartados pendientes de la memoria y se llevaron a cabo ajustes finales. Finalmente, se redactó la conclusión y el \textit{abstract}.
    
\end{itemize}

Finalmente, el trabajo se desarrolló entre el 24 de febrero y el 22 de agosto, considerando únicamente los días laborables, con excepción de los periodos del 4 al 8 de junio y del 3 al 10 de agosto, debido a vacaciones. Durante los días laborables se dedicó una media de 6 horas diarias, mientras que los sábados se trabajó aproximadamente 3 horas diarias.\footnote{Estimación realizada de manera aproximada.}

Teniendo en cuenta esta distribución, se puede estimar el total de horas trabajadas de la siguiente manera:

\begin{itemize}
    \item Días laborables efectivos: $122\textnormal{ días} \times 6\textnormal{h/día} = 732\textnormal{h}$
    \item Sábados: $23\textnormal{ días} \times 3\textnormal{h/día} = 69\textnormal{h}$
\end{itemize}

Por tanto, el total aproximado de horas dedicadas al proyecto asciende a 801 horas, lo que supera por mucho la cifra teórica de 300 horas. Este tiempo de más se puede atribuir a varias causas principales: 

\begin{itemize}

    \item Complejidad del proyecto: Nunca antes me había enfrentado a un trabajo de esta complejidad, por lo que la planificación inicial no podía anticipar todas las dificultades técnicas y conceptuales.
    
    \item Adaptación y organización del código: La necesidad de modificar constantemente el núcleo del código para hacerlo más modular y flexible ha supuesto tiempo adicional no contemplado en la estimación inicial.
    
    \item Iteraciones y mejoras del modelo: Se llevaron a cabo múltiples iteraciones de entrenamiento y ajuste de hiperparámetros de los modelos CNN, así como experimentos para mejorar la eficiencia y la estabilidad de las predicciones, lo que incrementó considerablemente la carga de trabajo.
    
    \item Búsqueda, análisis y revisión bibliográfica: La investigación en literatura especializada, especialmente en antropología forense, materia en la que soy nuevo, requirió un tiempo prolongado de lectura, análisis y síntesis para incorporarlo a la memoria.
    
    \item Circunstancias externas y aprendizaje: Ajustes en la planificación por cuestiones de salud, aprendizaje de nuevas herramientas (PyTorch, SLURM, Jupyter), problemas con el software (drivers de CUDA) y adaptaciones metodológicas también contribuyeron a la ampliación del tiempo dedicado.

\end{itemize}

% ------------------------------------------------------------------------------------------------------------

\section{Presupuesto del proyecto}

En este apartado hacemos un ejercicio de cuándo hubiese costado desarrollar este trabajo 

Se ha dispuesto de todos los materiales necesarios para la realización del proyecto de manera gratis. Aún así, en este apartado se hace una estimación del coste de desarrollar el trabajo ... 


Este trabajo ha sido realizado con dos equipos independientes: 

\begin{itemize}
    \item \textbf{Ordenador portátil personal}: Asus Zephyrus G14, modelo GA401 de 2021, empleado principalmente para la redacción y compilación de este documento en \LaTeX, así como el anális exploratorio de datos y el tratamiento de resultados. El sistema operativo empleado es Linux Mint (versión 22.1).
    El portátil fue adquirido por 1100\euro{} en marzo de 2022. Para la estimación de la amortización se ha considerado una vida útil de 4 años, criterio habitual en contabilidad empresarial y acorde con la obsolescencia tecnológica de este tipo de dispositivos:

    \begin{align*}
    \text{Coste imputado} 
    &= \frac{\text{Precio de adquisición}}{\text{Vida útil}} \times 
    \begin{array}{c}
        \text{Tiempo de uso} \\ 
        \text{en el proyecto}
    \end{array} \\[1ex]
    &= \frac{1100 \text{ \euro}}{48 \text{ meses}} \times 6 \text{ meses} = 137.5 \text{ \euro}
    \end{align*}

    \item \textbf{Clúster de computación DaSCI}: proporcionado por el Instituto de Ciencia de Datos e Inteligencia Artificial de la Universidad de Granada, al que se accede mediante conexión SSH. 
    
    El clúster cuenta con nodos diversos: con y sin GPU, con entre 40 y 60 CPUs y aproximadamente 122 GB de memoria RAM cada uno. Los nodos con GPU presentan diferentes modelos de gráfica: NVIDIA TITAN XP y NVIDIA Titan RTX, con CUDA 11.7 o superior. Todos los nodos cuentan con sistemas Linux compatibles con Pytorch.
    La implementación del código será adaptada para ejecutarse en cualquiera de los nodos con GPU. 

    El desarrollo del trabajo ---tanto experimentación como puesta en marcha--- se ha cobrado un total de 682.72 horas de GPU para el entrenamiento e inferencia de los modelos ML%
    \footnote{
        Estas cifras de tiempo han sido obtenidas del clúster Slurm, donde se han registrado todas las horas consumidas por cada trabajo. 
    }.  
    \todo{Aquí me estoy adelantando en el trabajo, ya que las horas totales las he obtenido tras terminar el trabajo, pero no sé de qué manera podría haberla estimados antes de comenzado un trabajo que es de investigación y donde se pierde más tiempo probando y errando que ejecutando el código definitivo.}

%     \begin{lstlisting}[language=bash]
% $ sacct -S 2024-02-01 -E now --name=train_models --format=JobName,ElapsedRaw -P -X > tiempos.csv
% $ awk -F'|' 'NR>1 {sum[$1]+=$2} END {for (n in sum) printf "%s: %.2f horas\n", n, sum[n]/3600}' tiempos.csv
%     \end{lstlisting}
    
    Para simplificar la estimación del coste, se considera un único nodo con la GPU más económica, NVIDIA Titan XP, ya que los requisitos computacionales del proyecto no justificaban el uso de hardware más potente.  
    La estimación del coste de uso de la GPU se basa en precios de referencia de servicios \textit{cloud} equivalentes. La NVIDIA Titan XP no es una GPU de centro de datos, pero se aproxima ---en rendimiento para operaciones F32 y memoria--- a una NVIDIA Tesla P100, con un precio de referencia entre 1.2-1.5 \euro por hora. Asumiendo un valor medio de 1.35 \euro/h, el coste total imputado del uso de la GPU en el proyecto sería:

    \begin{align*}
        \textnormal{Coste GPU} &= \textnormal{Tiempo GPU} \times \textnormal{Precio por tiempo} \\[1ex]
        &= 682.72\text{ h} \times 1.35 \text{ \euro/h} = 921.7 \text{ \euro}
    \end{align*}
\end{itemize}

En cuanto al software, se ha recurrido exclusivamente a herramientas de código abierto y libre distribución:

\begin{itemize}
    \item \textbf{Linux Mint}: es el sistema operativo del ordenador personal. 
    \item \textbf{Visual Studio Code}: empleado como entorno de desarrollo integrado (IDE) y como interfaz de conexión al clúster mediante SSH.
    \item \textbf{TeX Live}: distribución de \LaTeX, utilizada para la redacción, compilación y maquetación del documento.
    \item \textbf{Python}: lenguaje de programación utilizado para la implementación de los algoritmos.
    \item \textbf{Jupyter Notebook}: para experimentación interactiva y análisis exploratorio de datos y resultados.
    \item \textbf{PyTorch}: framework de referencia en el desarrollo de modelos de aprendizaje profundo.
    \item \textbf{Matplotlib y Seaborn}: bibliotecas de visualización para la representación gráfica de resultados, métricas y distribuciones.
\end{itemize}

Finalmente, en cuanto a recursos humanos, el coste asociado a la mano de obra se divide en dos categorías: investigación (desarrollo del trabajo) y mantenimiento del clúster.

En cuanto a la investigación, aunque el trabajo ha sido realizado por el autor, se puede estimar un coste hipotético basado en horas dedicadas. ...
\todo{Qué número de horas y precio debería considerar? El número de horas teóricas (300 horas) o reales (801)?}

Respecto al mantenimiento del clúster, al tratarse de un recurso institucional y dado que su coste operativo ya estaría incluido en servicios comerciales equivalentes, no se imputa adicionalmente al proyecto.

El coste total estimado, considerando ..., asciende a ...
\todo{Completar}
   \chapter{Fundamentos teóricos}

Este capítulo tiene el propósito de presentar y describir los fundamentos teóricos que sustentan los métodos 
utilizados en el trabajo, además de justificar su importtancia para abordar los problemas planteados.

% ------------------------------------------------------------------------------------------------------------
% MACHINE LEARNING -------------------------------------------------------------------------------------------
% ------------------------------------------------------------------------------------------------------------

\section{Machine Learning}

Frente a la idea de intentar crear un programa que simulara directamente el comportamiento inteligente de una 
``mente adulta'', Alan Turing ya vaticinó un enfoque alternativo \cite{turing1950}: que las máquinas pudieran 
aprender como lo hace un niño, mediante un ``proceso educativo'' con el cual se logra alcanzar progresivamente 
una ``mente adulta'', obteniendo así comportamientos inteligentes complejos.

% Athur Samuel años 50 surge machine learning
%He popularized the term "machine learning" in 1959.[4]
%Samuel, Arthur L. (1959). "Some Studies in Machine Learning Using the Game of Checkers". IBM Journal of Research and Development. 44: 206–226.

En las décadas de 1960, 1970 y 1980, en un contexto marcado por las limitaciones computacionales y el 
escepticismo académico surgió el aprendizaje automático (\textit{machine learning}, (ML)) como 
una rama marginal de la IA, centrada en el desarrollo de modelos y algoritmos que permitiesen a las 
computadoras imitar la forma en la que los humanos aprenden, realizar tareas autónomas y mejorar su 
rendimiento a través de la experiencia y exposición a más datos. De esta forma, estos modelos podrían realizar 
predicciones o tomar decisiones sin ser programadas para cada caso.

Los años 90 y 2000 marcaron un punto de inflexión para el ML, gracias a los avances teóricos, el mayor poder 
computacional y la disponibilidad de grandes volúmenes de datos. De 2010 en adelante, la evolución del ML ha 
sido exponencial, marcada por la consolidación del \textit{deep learning}, la escalabilidad masiva y su 
integración en numerosas aplicaciones: de visión por computador, reconocimiento de lenguaje natural, robótica, 
diagnóstico médico y forense, finanzas o recomendación de contenidos, entre otros. De esta forma, el ML se ha 
convertido en un campo tan amplio y exitoso que ahora ``eclipsa'' al resto de campos de la IA 
\cite{domingos2015}.

El ML diferencia tres tipos de aprendizaje en base a tres tipos de retroalimentación \cite{rusell2021}: 

\begin{itemize}
    
    \item \textbf{Aprendizaje supervisado}, en el que el agente (refiriéndose con este al modelo de ML y su 
    algoritmo de aprendizaje) observa ejemplos de pares entrada-salida y aprende la función que mejor mapea 
    las entradas (inputs) a las salidas (outputs) correspondientes. El objetivo es generalizar este 
    aprendizaje para hacer predicciones precisas sobre datos nuevos y no vistos \cite{bishop2006}.

    \item \textbf{Aprendizaje por refuerzo}, en el que los datos de entrenamiento no contienen salida 
    objetivo, sino que contiene posibles resultados junto con medidas de calidad de dicho resultado, es decir, 
    una función de evaluación del estado. En este tipo de aprendizaje, el agente toma decisiones en un entorno 
    y recibe recompensas o penalizaciones por las acciones que realiza, ajustando su comportamiento mediante 
    prueba y error, maximizando la recompensa acumulada en el tiempo \cite{alpaydin2010}.

    \item \textbf{Aprendizaje no supervisado}, en el que el agente tampoco dispone de valores de salida, solo 
    de entrada \cite{bishop2006}, y los objetivos pueden ser muy variados, centrándose en descubrir patrones, 
    estructuras o relaciones ocultas en los datos. A diferencia de los otros enfoques, aquí no hay una 
    ``respuesta correcta'' predefinida, sino que el modelo debe inferir conocimiento directamente desde la 
    distribución de los datos.

\end{itemize}

Este trabajo se centrará en el aprendizaje supervisado, pues es este tipo de aprendizaje el empleado en los 
problemas de clasificación y regresión que aplicaremos en el ámbito de la antropología forense.


% Retos en el aprendizaje supervisado


% No importa el formato de entrada



% ------------------------------------------------------------------------------------------------------------

% En el \textbf{aprendizaje supervisado} \cite{bishop2006}, el agente (refiriéndose con este al 
% modelo de ML y su algoritmo de aprendizaje) observa ejemplos de pares entrada-salida y aprende
% la función que mejor mapea las entradas (inputs) a las salidas (outputs) correspondientes. El 
% objetivo es generalizar este aprendizaje para hacer predicciones precisas sobre datos nuevos y 
% no vistos.

% Hay dos principales tipos de problemas de aprendizaje supervisado: 

% \begin{itemize}
%     \item la \textit{clasificación} para cuando el valor de salida es una etiqueta categórica, y
%     \item la \textit{regresión} cuando el valor de salida es un valor continuo.
% \end{itemize}

% % ----------------------------------------------------------------------------------------------------------

% En cambio, en el \textbf{aprendizaje por refuerzo}, los datos de entrenamiento no contienen 
% salida objetivo, sino que contiene posibles resultados junto con medidas de calidad de dicho 
% resultado, es decir, una función de evaluación del estado. En este tipo de aprendizaje, el 
% agente toma decisiones en un entorno y recibe recompensas o penalizaciones por las acciones 
% que realiza, ajustando su comportamiento mediante prueba y error, maximizando la recompensa 
% acumulada en el tiempo \cite{alpaydin2010}.  

% De esta forma, el algoritmo no aprende a dar una acción/salida a partir de un input, sino que 
% desarrolla una política que determina la mejor acción a tomar en cada estado del entorno, con 
% el objetivo de maximizar la recompensa acumulada a largo plazo.

% Esta aproximación es clave en aplicaciones como juegos, robótica, optimización de recursos y 
% sistemas conversacionales (ChatAI), donde las decisiones secuenciales y la interacción 
% dinámica con el entorno son fundamentales.

% % ----------------------------------------------------------------------------------------------------------

% Y, por último, en el \textbf{aprendizaje no supervisado}, el agente tampoco dispone de valores 
% de salida, solo de entrada \cite{bishop2006}, y los objetivos pueden ser muy variados, 
% centrándose en descubrir patrones, estructuras o relaciones ocultas en los datos. 

% A diferencia de los otros enfoques, aquí no hay una "respuesta correcta" predefinida, sino que 
% el modelo debe inferir conocimiento directamente desde la distribución de los datos.

% Algunos problemas clásicos de este tipo de aprendizaje son: 

% \begin{itemize}

%     \item El \textit{clustering} o agrupamiento, donde el objetivo es encontrar clústeres o 
%     agrupaciones del input. Esto puede ser útil, por ejemplo, para una empresa que quiera 
%     segmentar sus clientes, o para identificar patrones en datos genéticos sin etiquetar.
    
%     \item La \textit{detección de anomalías} (\textit{outlier detection} en inglés), que 
%     consiste en encontrar instancias atípicas o inusuales en los datos. Sus aplicaciones 
%     incluyen: identificación de fraudes en transacciones bancarias, fallos en equipos 
%     industriales o identificación de ciberataques.

%     \item La \textit{reducción de dimensionalidad}, que trata de reducir el número de 
%     variables manteniendo la mayor información posible. Esto es útil para visualizar datos 
%     complejos o mejorar la eficiencia de algoritmos (p.ej., transformaciones PCA y t-SNE).

%     \item El \textit{aprendizaje de representaciones} (como los \textit{autoencoders}), 
%     donde el modelo busca capturar características latentes de los datos de manera eficiente. 
%     La compresión de archivos o la reducción de ruido en imágenes son algunos ejemplos de 
%     aplicaciones. 

% \end{itemize}



% ------------------------------------------------------------------------------------------------------------

\subsection{Problemas de regresión}

Como se ha mencionado antes, la regresión es un tipo de problema clásico en el aprendizaje supervisado, y 
consiste en predecir el valor de una o más \textbf{variables continuas} objetivo a partir de unos datos de 
entrada \cite{bishop2006}, utilizando un modelo entrenado con ejemplos ya con valores conocidos.

Matemáticamente, este proceso implica modelar la relación entre la variable dependiente $Y$ y las variables
independientes $X$, de modo que se pueda predecir o explicar el comportamiento de $Y$ en función de los 
valores de $X$. El modelo aprende una función de predicción $f$ que, dado un nuevo ejemplo $i$ con 
características $X_i$, genera una estimación $\hat{Y_i}$:

$$
f(X_i) = f(X_{i0}, X_{i1}, \dots, X_{in}) = \hat{Y_i} = Y_i + \varepsilon_i
$$

donde 

\begin{itemize}
    \item $X_{i0},X_{i1}, \dots, X_{in}$ son las características o atributos del ejemplo $i$,
    \item $Y_i$ es el valor real de la variable objetivo para ese ejemplo,
    \item $\hat{Y_i}$ es la predicción generada por el modelo, y
    \item $\varepsilon_i$ representa el error o residuo \footnote{... a pesar de que en la literatura estos 
    términos se distinguen, ...}, es decir, la diferencia entre la predicción y el valor real.
    Este término captura factores aleatorios o imprecisiones que el modelo no logra explicar perfectamente.
\end{itemize}

El análisis y la evaluación estadística del error son fundamentales para valorar la utilidad práctica del 
modelo y optimizar su capacidad predictiva mediante técnicas de ajuste y validación. Existen numerosas 
métricas para evaluar el rendimiento en problemas de regresión, pero tres destacan especialmente por ser 
\textit{model-agnostic}, es decir, aplicables a cualquier modelo de regresión independientemente del algoritmo 
subyacente. Estas son:

\begin{itemize}
    \item El \textbf{error absoluto medio (\textit{mean absolute error}, MAE)} mide el promedio de las 
    diferencias absolutas entre los valores reales ($Y_i$) y los valores predichos ($\hat{Y_i}$) por el 
    modelo.

    $$
    MAE = \frac{1}{n} \sum_{i=1}^n{|Y_i - \hat{Y_i}|}
    $$

    donde $n$ es el número de ejemplos/instancias con las que se cuenta en los datos a evaluar.

    La interpretación más inmediata de esta métrica es que representa cuánto se desvía en promedio la 
    predicción del valor real sin considerar la dirección del error (positivo o negativo) y, por tanto, cuanto 
    más se acerque a cero el valor, mejor es el ajuste del modelo.

    Existe una variante denominada \textbf{error absoluto mediano (\textit{median absolute error}, MedAE)}, 
    que realiza la mediana de las diferencias absolutas, en vez de la media, aumentando la robustez frente a 
    valores atípicos con errores extremos.

    \item El \textbf{error cuadrático medio (\textit{mean squared error}, MSE)} mide el promedio de los 
    errores al cuadrado entre valores reales ($Y_i$) y los valores predichos ($\hat{Y_i}$) por el modelo.
    
    $$
    MSE = \frac{1}{n} \sum_{i=1}^n{(Y_i - \hat{Y_i})^2}
    $$

    Al igual que el MAE, cuantifica qué tan cerca están las predicciones de los valores reales, pero penaliza
    más los errores grandes, y es más sensible por tanto a valores atípicos.

    Como veremos más tarde, esta métrica es muy útil en optimización mediante gradiente descendente, usado a 
    la hora de entrenar modelos de regresión basados en redes neuronales.

    Y también tiene una variante, la \textbf{raíz del error cuadrático medio (\textit{root mean square error}, 
    RMSE)}, que se obtiene extrayendo la raíz cuadrada del MSE:

    $$
    RMSE = \sqrt{\frac{1}{n} \sum_{i=1}^{n}{(Y_i-\hat{Y_i})^2} }
    $$

    Esta métrica conserva las mismas unidades que la variable objetivo, lo que facilita su interpretación 
    práctica. Es comparable con el MAE en cuanto a escala, aunque sigue penalizando más los errores grandes.

    \item El \textbf{coeficiente de determinación}, o más conocido como \textbf{R² o bondad de ajuste}, mide 
    la proporción de la variabilidad de la variable dependiente ($Y$) que es explicada por el modelo.

    $$
    R^2 = 1 - \frac{\sum_{i=1}^n (Y_i - \hat{Y}_i)^2}{\sum_{i=1}^n (Y_i - \bar{Y})^2}
    $$

    donde 

    \begin{itemize}

        \item $Y_i$ es el valor real de la variable dependiente para la instancia $i$,
        
        \item $\hat{Y_i}$ es la predicción generada por el modelo, y
        
        \item $\bar{Y}$ es el promedio de los valores reales de la variable dependiente a lo largo de todas 
        las instancias del conjunto de datos.
    
    \end{itemize}
    

    El valor de esta métrica varía entre $-\infty$ y 1, y su interpretación es la siguiente:

    \begin{itemize}

        \item $R^2 \le 0$ significa que el modelo no explica ninguna variabilidad y que las predicciones del 
        modelo no son mejores que simplemente predecir la media de los valores reales.
        
        \item $R^2 \in \left(0,1\right)$ indica que el modelo está explicando una fracción de la variabilidad 
        de los datos, y cuanto más cercano sea a 1, mejor será el ajuste del modelo.
        
        \item $R^2 = 1$ indica un ajuste perfecto y, por tanto, el modelo explica toda la variabilidad de los 
        datos. 
    
    \end{itemize}

    A diferencia de las anteriores, es una métrica relativa y adimensional, es decir, no depende de las 
    unidades de la variable objetivo y evalúa qué tan bien se ajusta el modelo en comparación con un modelo 
    base que siempre predice la media de los valores reales.
    
\end{itemize}

% Elementos visuales

No obstante, el uso exclusivo de métricas numéricas resulta en un análisis pobre, ya que estas no permiten 
identificar patrones ocultos, detectar relaciones no lineales ni distinguir entre errores positivos o 
negativos. Por esta razón, se recomienda completar el análisis con representaciones gráficas, tales como:

\begin{itemize}

    \item La \textbf{gráfica de puntos de valores reales vs. predichos}, que permite visualizar la relación 
    entre las predicciones del modelo y los valores reales. Idealmente, los puntos deberían alinearse 
    alrededor de la recta $Y=\hat{Y}$. Desviaciones sistemáticas indican sesgos o problemas de ajuste. Un 
    ejemplo de esta gráfica lo encontramos en la Figura \ref{fig:scatter_pred_vs_act_AE}.

    \begin{figure}[h]
        \centering
        \includegraphics[width=0.5\textwidth]{capitulos/cap_02/imagenes/scatterplot_pred_vs_act_AE.png}
        \caption[
            Gráfica de puntos de valores de edad reales vs. predichos obtenidos por el modelo propuesto 
            en \cite{heinrich2024}.
        ]{
            Gráfica de puntos de valores de edad reales vs. predichos obtenidos por el modelo propuesto 
            en \cite{heinrich2024}. Se observan valores más dispersos en edades avanzadas.
        } 
        \label{fig:scatter_pred_vs_act_AE}
    \end{figure}

    \item También existe una versión más refinada de presentar esta información, especialmente útil en casos 
    en los que muchos datos sobrecargan la gráfica, en \textbf{la gráfica de cajas (en inglés 
    \textit{boxplot}) de valores reales vs. predichos}. Estos proporcionan una visión clara de la distribución 
    de los datos, con mediana, cuartiles y valores atípicos, ya sea agrupando por valores reales o por valores 
    predichos.

    La Figura \ref{fig:boxplot_pred_vs_act_AE} muestra la distribución de las edades predichas en función de 
    distintos grupos de edad real, y viceversa, lo que facilita la identificación de errores en el desempeño 
    del modelo.

    \begin{figure}[h]
        \centering
    
        \begin{subfigure}[b]{0.45\textwidth}
            \centering
            \includegraphics[width=\textwidth]{capitulos/cap_02/imagenes/boxplot_pred_vs_act_AE_1.png}
            \caption{Gráfica de cajas de edades reales en función de la estimada}
            \label{fig:boxplot_pred_vs_act_AE_a}
        \end{subfigure}
        \hfill
        \begin{subfigure}[b]{0.45\textwidth}
            \centering
            \includegraphics[width=\textwidth]{capitulos/cap_02/imagenes/boxplot_pred_vs_act_AE_2.png}
            \caption{Gráfica de cajas de edades estimadas en función de la real}
            \label{fig:boxplot_pred_vs_act_AE_b}
        \end{subfigure}
    
        \caption[
            Gráfica de cajas de valores de edad reales vs. predichos obtenidos por el modelo propuesto en 
            \cite{stepanovsky2024}.
        ]{
            Gráfica de cajas de valores de edad reales vs. predichos obtenidos por el modelo propuesto en 
            \cite{stepanovsky2024}. Se observa en \ref{sub@fig:boxplot_pred_vs_act_AE_b} que se sobreestima 
            la edad en personas jóvenes y se subestima 
            en personas de edad avanzada.
        }
        \label{fig:boxplot_pred_vs_act_AE}
    \end{figure}

    \item El \textbf{histograma de residuos} muestra la distribución de los errores (\(Y_i - \hat{Y}_i\)) del 
    modelo. Una distribución simétrica y centrada en cero sugiere un buen ajuste, mientras que una 
    distribución sesgada o asimétrica podría indicar que el modelo está subajustado o que hay algún patrón no 
    capturado por el modelo. 

    Un ejemplo de este tipo de gráfica lo encontramos en la Figura \ref{fig:prob_dist_AEerror}, donde se 
    analizaba el error obtenido con el modelo propuesto en \cite{stepanovsky2024}.

    Histograma de errores residuales del modelo de regresión propuesto en \cite{verma2020} que predice la 
    estatura a partir de la longitud de la tibia.

    \begin{figure}[h]
        \centering
        \includegraphics[width=0.8\textwidth]{capitulos/cap_02/imagenes/prob_distribution_AEerror.png}
        \caption[
            Histograma de errores residuales del modelo de estimación de edad propuesto en 
            \cite{stepanovsky2024}.
        ]{
            Histograma de errores residuales del modelo de estimación de edad propuesto en 
            \cite{stepanovsky2024}. Se evidencia una mayor probabilidad de errores negativos 
            (infraestimaciones) en comparación con los positivos Además, se destaca que el 57\% de las 
            predicciones presentan un error inferior al $\textnormal{MAE}$, y que el 90\% se encuentra dentro 
            de un margen de error menor a $2\textnormal{MAE}$.
        } 
        \label{fig:prob_dist_AEerror}
    \end{figure}


    \item Y una versión más completa que este último es la \textbf{gráfica de cajas de la distribución del 
    error en base a los valores reales o predichos}, que permite analizar cómo varía el error del modelo a lo 
    largo de diferentes rangos de valores, ya sean reales o predichos. Estas visualizaciones nos permiten 
    detectar fácilmente las fortalezas y debilidades en las predicciones del modelo, así como diagnosticar 
    sesgo o insuficiencia de datos en ciertas categorías.
    
    Un ejemplo ilustrativo de esta gráfica se presenta en la Figura \ref{fig:boxplot_error_vs_act_AE}, donde 
    se observa la variación del error del modelo propuesto en \cite{heinrich2024} a través de distintos rangos 
    de edad real. En particular, se evidencia una tendencia a cometer errores mayores en los grupos de edad 
    más avanzada.

    \begin{figure}[h]
        \centering
        \includegraphics[width=0.95\textwidth]{capitulos/cap_02/imagenes/boxplot_error_vs_act_AE.png}
        \caption{
            Gráfica de cajas de la distribución del error de estimación de edad en función de la edad 
            real, obtenida en el modelo propuesto en \cite{heinrich2024}.
        } 
        \label{fig:boxplot_error_vs_act_AE}
    \end{figure}

\end{itemize}






% ------------------------------------------------------------------------------------------------------------

\subsection{Problemas de clasificación}

En cambio, en los problemas de clasificación, los valores de salida son categóricos, denominados más 
comúnmente como \textbf{clases}, y a cada valor individual asignado a una instancia de datos se le conoce como 
\textbf{etiqueta} (\textit{label} en inglés).

Existen multitud de variante de clasificación, que pueden diferenciarse según diversos criterios:

\begin{itemize}
    \item En base a la cardinalidad de las clases de salida: \textbf{clasificación binaria o multiclase}, 
    según si existen dos clases posibles o más de dos, respectivamente.

    \item En base al número de etiquetas asignadas a cada instancia: \textbf{clasificación con etiqueta única 
    o multietiqueta}, según si cada instancia pertenece a una sola clase o a varias de forma simultánea.

    \item En base a la certeza de la asignación de clases: \textbf{clasificación con etiqueta precisa o 
    difusa}, donde en el primer caso la asignación a una clase es determinista, y en el segundo caso se 
    permite una pertenencia parcial a varias clases, con distinto grados de afinidad.
    
\end{itemize}

No obstante, la mayoría de los problemas estudiados en la literatura de ML, y concretamente en antropología 
forense, corresponden a clasificación binaria o multiclase, con etiquetas únicas y asignación precisa 
\cite{bishop2006}, que serán el foco de este trabajo. La cardinalidad de las clases tiene implicanciones 
significativas en el diseño del modelo y la evaluación de su desempeño:

\begin{itemize}

    \item \textbf{Clasificación binaria}, que es aquella en la que existen únicamente dos clases posibles para 
    la variable objetivo, siendo común en problemas donde se desea discriminar entre dos estados mutuamente 
    excluyentes (p.ej., ``positivo'' vs. ``negativo'', ``spam'' vs. ``no spam'', ``fraude'' vs. ``no 
    fraude'').
    
    Se suele denominar a una de las clases como ``positiva'' y a otra como ``negativa'' para facilitar la 
    interpretación de métricas como la precisión, la sensibilidad o la especifidad, si bien no tiene por qué 
    existir una connotación valorativa entre ambas clases.
    
    \item \textbf{Clasificación multiclase}: en este caso, la variable objetivo puede tomar más de dos valores 
    posibles, pertenecientes a un conjunto finito. Un ejemplo de problema clásico es el de clasificar dígitos
    manuscritos (0-9).

\end{itemize}

En este tipo de problemas, el error ocurre cuando no se acierta al predecir la clase del ejemplo.

% Tanto en clasificación binaria como multiclase, un problema común es el desequilibrio de clases, donde una 
% clase tiene muchos más ejemplos que otra. Esto afecta el entrenamiento del modelo, ya que puede sesgarse 
% hacia la clase mayoritaria, ya que el modelo


Una vez definido los tipo de problemas de clasificación, es fundamental establecer cómo medir la efectividad 
del modelo predictivo. A continuación, se detallan los principales criterios y elementos gráficos utilizados 
para evaluar y comparar modelos de clasificación:

\begin{itemize}
    
    \item La \textbf{matriz de confusión} es una herramienta fundamental que permite visualizar el rendimiento 
    de modelos de clasificación, tanto binarios como multiclase. Esta muestra una tabla con tantas columnas y 
    filas como clases haya. En un eje, se representan las clases reales (etiquetas verdaderas), y en el otro 
    eje, las clases predichas por el modelo. Cada celda de la matriz indica la cantidad de ejemplos que 
    pertenecen a una clase real específica y que han sido clasificados como una clase predicha específica 
    (véase la Figura \ref{fig:conf_matrix_binary}).

    Idealmente, los valores se concentrarían en la diagonal principal, lo que indicaría que las predicciones 
    coinciden con los valores reales.

    \begin{figure}[h]
        \centering
        \includegraphics[width=0.6\textwidth]{capitulos/cap_02/imagenes/confusion_matrix_binary.png}
        \caption{
            Matriz de confusión para la estimación de sexo según el modelo \textit{random forest} 
            propuesto en \cite{bidmos2023}.
        } 
        \label{fig:conf_matrix_binary}
    \end{figure}

    Esta visualización admite muchas variantes, por ejemplo, como la vista en la Figura 
    \ref{fig:conf_matrix_binary_relative}.
    
    \begin{figure}[h]
        \centering
    
        \begin{subfigure}[b]{0.3\textwidth}
            \centering
            \includegraphics[width=\textwidth]{capitulos/cap_02/imagenes/confusion_matrix_binary_1.png}
            \caption{Sin información de sexo}
            \label{fig:conf_matrix_general}
        \end{subfigure}
        \hfill
        \begin{subfigure}[b]{0.3\textwidth}
            \centering
            \includegraphics[width=\textwidth]{capitulos/cap_02/imagenes/confusion_matrix_binary_2.png}
            \caption{Sexo femenino}
            \label{fig:conf_matrix_female}
        \end{subfigure}
        \hfill
        \begin{subfigure}[b]{0.3\textwidth}
            \centering
            \includegraphics[width=\textwidth]{capitulos/cap_02/imagenes/confusion_matrix_binary_3.png}
            \caption{Sexo masculino}
            \label{fig:conf_matrix_male}
        \end{subfigure}
    
        \caption[
            Matrices de confusión para la estimación de mayoría/minoría de edad según el modelo de 
            \cite{porto2020}.
        ]{
            Matrices de confusión para la estimación de mayoría/minoría de edad según el modelo de 
            \cite{porto2020}.
            Se representan los valores de cada celda en términos porcentuales de los ejemplos reales que hay 
            de cada clase ($< 18$ y $\ge 18$), lo que permite comparar la matriz de confusión general de todos 
            los ejemplos (\ref{sub@fig:conf_matrix_general}) con la de ejemplos se sexo femenino 
            (\ref{sub@fig:conf_matrix_female}) y sexo masculino (\ref{sub@fig:conf_matrix_male}), permitiendo 
            identificar posibles sesgos en el modelo respecto al género, y así realizar una evaluación más 
            precisa del rendimiento del modelo en diferentes subgrupos de la población.
        }
        \label{fig:conf_matrix_binary_relative}
    \end{figure}

    Prácticamente todas las métricas y visualizaciones parten de la información ofrecida en esta matriz. 
    

    \item La \textbf{exactitud (\textit{accuracy})} es la proporción de predicciones correctas sobre el total.
    En clasificación binaria esto sería:
    
    $$
    \textnormal{Accuracy} = \frac{TP+TN}{TP+TN+FP+FN}
    $$

    En el caso de clasificación multiclase, esta se generaliza como:

    $$
    \textnormal{Accuracy} = 
        \frac{\textnormal{Numero de predicciones correctas}}{\textnormal{Número total de ejemplos}} =
        \frac{1}{N} \sum_{i=1}^N1(\hat{y_i}=y_i)
    $$

    donde $\hat{y_i}$ es la etiqueta predicha, $y_i$ la etiqueta verdadera para el ejemplo $i$, $N$ es el 
    número de ejemplos, y 1 es la función indicadora, que vale 1 si la predicción es correcta y 0 si no 
    lo es. 

    Los valores de esta métrica varían entre 0 y 1, donde 0 es el peor desempeño posible (todas las 
    predicciones son incorrectas) y 1 es el mejor desempeño posible (todas las predicciones son correctas).

    Es la medida más intuitiva, si bien puede dar una falsa impresión de buen desempeño si las clases 
    mayoritarias dominan la métrica. Es por esto que el análisis debe completarse con otras métricas 
    informativas. 


    \item La \textbf{precisión (\textit{precision})} indica qué proporción de las predicciones positivas 
    corresponde a casos realmente positivos. En clasificación multiclase, se interpreta como la proporción de 
    ejemplos correctamente clasificados entre todos los que fueron asignados a una clase determinada. 

    $$
    \textnormal{Precision} = \frac{TP}{TP+FP}
    $$

    Una alta precisión significa pocos falsos positivos. Esto puede interesar por ejemplo

    La \textbf{exhaustividad (\textit{recall})} indica qué proporción de los casos positivos fueron 
    correctamente detectados.

    $$
    \textnormal{Recall} = \frac{TP}{TP+FN} 
    $$

    Un alto \textit{recall} significa pocos falsos negativos. 

    
    Estas dos métricas complementarias se pueden calcular por cada clase, y hay varias formas de combinar sus
    valores:

    \begin{itemize}
        \item Macro:
        \item Micro:
        \item Weighted: 
    \end{itemize}
    

    \item El \textbf{F1-Score} 
    \todo{Por completar}

    $$
    \textnormal{F1-Score} = 2 \cdot \frac{\textnormal{Precision} \cdot \textnormal{Recall}}{\textnormal{Precision} + \textnormal{Recall}}
    $$

    Al igual que con \textit{precision} y \textit{recall}, también se puede calcular por cada clase. 


\end{itemize}

% ------------------------------------------------------------------------------------------------------------

\subsection{Selección de modelos y optimización}

El objetivo del ML es establecer una hipótesis que se ajuste de forma óptima a los ejemplos futuros. Para 
ello, suponemos que los ejemplos futuros mostrarán un comportamiento similar a los pasados. Bajo este 
supuesto, el ajuste óptimo de un modelo es, por tanto, la hipótesis que minimiza la tasa de error del 
problema \cite{rusell2021}. 

Pero medir el error del modelo sobre los mismos datos empleados en el entrenamiento suele sesgar el resultado, 
ya que el modelo puede estar sobreajustado (\textit{overfitting}) a los datos de entrenamiento, capturando no 
solo el patrón subyacente, sino también el ruido o las peculiaridades específicas de ese conjunto de datos.
Para evitar esto, es fundamental evaluar el modelo en un conjunto de datos de prueba independiente, que simule 
cómo se comportaría con ejemplos futuros no vistos durante el entrenamiento. Por este motivo, es común dividir 
los datos disponibles en dos conjuntos distintos: el \textbf{conjunto de entrenamiento (\textit{training 
set})} y \textbf{conjunto test (\textit{test set})}.

Aun así, incluso con esta división de conjuntos, puede persistir el riesgo de sobreajuste si se realizan 
múltiples ajustes y selecciones de hiperparámetros basados en el rendimiento en el conjunto test.
Esto se debe a que, indirectamente, el modelo podría estar ``aprendiendo'' características específicas del 
conjunto de prueba, comprometiendo su capacidad de generalización. Para abordar este problema, se introduce 
un tercer subconjunto: el \textbf{conjunto de validación}. Este conjunto se utiliza para evaluar y ajustar 
los hiperparámetros del modelo durante el desarrollo, reservando el conjunto test únicamente para la 
evaluación final.

Además, técnicas como la \textbf{validación cruzada (\textit{cross-validation})} son ampliamente utilizadas 
para maximizar el uso de los datos disponibles, especialmente en conjuntos pequeños. En lugar de una única 
división entrenamiento-validación, este método:

\begin{enumerate}
    \item Divide los datos en $k$ particiones (\textit{folds}) (véase la Figura \ref{fig:CVDiagram}).
    \item En cada iteración, usa $k-1$ particiones para entrenamiento y la restante para validación, rotando
    sistemáticamente la partición de validación hasta que cada una de las $k$ particiones haya sido utilizada 
    exactamente una vez como conjunto de validación. 
    \item Promedia los resultados de todas las iteraciones para obtener una métrica robusta.
\end{enumerate}

El modelo final se entrena con todos los datos de entrenamiento (incluyendo los usados en validación durante 
el ajuste). Si bien esta técnica proporciona estimaciones más confiables, su costo computacional es 
significativo, ya que requiere entrenar el modelo $k+1$ veces ($k$ iteraciones de validación más el 
entrenamiento final), lo que puede ser prohibitivo para modelos complejos, como redes neuronales profundas.

\begin{figure}[h]
    \centering
    \includegraphics[width=0.95\textwidth]{capitulos/cap_02/imagenes/CVDiagram.png}
    \caption{
        Diagrama de división del dataset para la validación cruzada. 
        Recuperado de \cite{lau2023crossvalidation}.
    } 
    \label{fig:CVDiagram}
\end{figure}


% ------------------------------------------------------------------------------------------------------------
% ENSEMBLE LEARNING ------------------------------------------------------------------------------------------
% ------------------------------------------------------------------------------------------------------------

% \section{Ensemble Learning}

% El \textbf{aprendizaje por conjuntos (\textit{ensemble learning})} es una familia de técnicas de ML que 
% combina múltiples modelos de aprendizaje automático ---denominados \textbf{modelos base}--- para obtener un 
% modelo final con un mejor desempeño que los modelos base por separado.

% \subsection{Bagging}


% \subsection{Boosting}


% \subsection{Stacking}


% ------------------------------------------------------------------------------------------------------------
% DEEP LEARNING ----------------------------------------------------------------------------------------------
% ------------------------------------------------------------------------------------------------------------

\section{Deep Learning}

El \textbf{aprendizaje profundo (\textit{deep learning}, DL)} es una familia de técnicas de ML que utiliza
redes neuronales de muchas capas. 
Las redes neuronales tienen su origen en el intento de modelar las redes de neuronas del cerebro humano 
\cite{mcculloch1943}. Se requirió de numerosas contribuciones teóricas ---como el perceptrón 
\cite{rosenblatt1958} o el algoritmo de \textit{backpropagation} \cite{rumelhart1986,werbos1994}, entre 
otras---, disponibilidad de datos estandarizados y un gran aumento en la capacidad computacional para poder 
escalar estar redes y obtener resultados 
sorprendentes en tareas complejas.

Las \textbf{redes neuronales profundas (\textit{deep neural networks}, DNNs)} destacan por su capacidad para 
aprender representaciones jerárquicas: cada capa extrae características progresivamente más abstractas 
\cite{lecun2015}, desde líneas en imágenes hasta formas geométricas complejas, objetos completos e incluso 
escenas compuestas.
Esta propiedad las hace excepcionalmente versátiles, ya que procesan datos de muy diversa naturaleza ---datos 
tabulares, imágenes, audio, texto o señales temporales---, dados que ellas mismas aprenden los procesos de 
extracción de características de estos, hasta ahora realizados ``a mano'' (mediante procesos diseñados por la 
ingeniería de características) \cite{rusell2021} \footnote{Este enfoque se denomina aprendizaje extremo a 
extremo (\textit{end-to-end learning}), en el cual tanto la extracción de características como la 
clasificación son parte de un modelo integral que se entrena de manera conjunta, optimizando todos los 
componentes del sistema en un mismo proceso \cite{rusell2021}.}. Gracias a ello, las DNNs han alcanzado 
rendimientos sobresalientes en dominios como visión por computador (clasificación de imágenes, detección de 
objetos, segmentación) o procesamiento de lenguaje natural (traducción, generación de texto) 
\cite{redhat2024DeepLearningDefinition}.
No obstante, su eficacia depende críticamente de grandes volúmenes de datos y recursos computacionales, lo que 
ha impulsado técnicas como el \textit{transfer learning} y modelos eficientes para democratizar su uso.


\subsection{El perceptrón multicapa}

El \textbf{perceptrón multicapa (\textit{multilayer perceptron}, MLP)} forma la base del \textit{deep 
learning}. Su diseño ---con capas ocultas, funciones de activación no lineales y entrenamiento  mediante 
\textit{backpropagation}--- sentó las bases conceptuales para arquitecturas más complejas, como las redes 
neuronales convolucionales o los \textit{transformers} \cite{murphy2022}. El MLP sigue siendo un referente 
teórico y la expresión más simple de cómo el aprendizaje jerárquico puede capturar patrones en los datos. 

Cada nodo en la red es denominado \textbf{unidad o neurona artifical}. Siguiendo el diseño propuesto en 
\cite{mcculloch1943,rosenblatt1958}, cada unidad recibe señales de entrada ---que o bien son las 
características de los datos o bien las salidas de las unidades de la anterior capa---, realiza una suma 
ponderada de estas con los pesos entrenables de cada conexión ---más un término independiente o sesgo, también 
entrenable---, aplica una función no lineal sobre esta para producir una salida que propaga a las unidades de 
la siguiente capa (véase la Figura \ref{fig:neuron_MLP}).

Matemáticamente, la operación de una unidad artifical se expresaría como:

$$
y = f \left( \sum_{i=1}^n{w_ix_i+b} \right)
$$

donde $x_i$ son las entradas, $w_i$ son los pesos entrenables ($w_0$ el sesgo)\footnote{
    El sesgo se considera un peso, puesto que, en la implementación, son un peso más conectado a una unidad
    de sesgo con valor constante unitario (1).
}, y $f$ es la función de 
activación.

\begin{figure}[h]
    \centering
    \includegraphics[width=0.95\textwidth]{capitulos/cap_02/imagenes/Neuron_perceptron.png}
    \caption{
        Esquema visual del funcionamiento de una unidad artificial. Adaptado de 
        \cite{codeworld2022understandingMLDL}.
    } 
    \label{fig:neuron_MLP}
\end{figure}


Esta \textbf{función de activación} a la salida de la unidad es un componente esencial que introduce no 
linealidad en el modelo, permitiendo a la red aprender relaciones complejas en los datos\footnote{
    Sin ella, el MLP se reduciría a una simple combinación lineal de las entradas, incapaz de
    representar jerarquías de características \cite{murphy2022}.
}. Existe multitud de funciones de activación, 
como la sigmoide, la tangente hiperbólica o ReLu ---y sus múltiples variantes---, cada una con sus ventajas 
y limitaciones
\footnote{
    Si bien, actualmente, ReLU y sus variantes (\textit{Leaky} ReLU, \textit{Parametric ReLU} o 
    \textit{Swish}) se han convertido en el estándar \textit{de facto} para las capas ocultas en DNNs,
    por su eficiencia computacional, y su eficacia empírica \cite{vargas2021}.
}.

La arquitectura de un MLP conecta estas unidades formando una red neuronal retroalimentada\footnote{Una red 
neuronal retroalimentada (\textit{feed-forward neural network}) es aquella en la que las conexiones entre las 
unidades no forman un ciclo y, por tanto, la información solo se mueve en una dirección: adelante.},
que consta de tres partes (véase la Figura \ref{fig:neural_network}):

\begin{itemize}

    \item \textbf{Capa de entrada}, en las que el número de unidades debe coincidir con el formato de entrada 
    de los datos, por ejemplo: en un problema con datos tabulares, debería haber una unidad por cada 
    característica.
    
    \item \textbf{Capas ocultas}, donde se realizan las transformaciones no lineales de los datos. Es en estas 
    donde el diseño puede variar en número de unidades y tipo de capas según la complejidad del problema y los 
    datos.
    
    \item \textbf{Capa de salida}, que proporciona el resultado del modelo. Su forma depende del problema a 
    resolver: 
    
    \begin{itemize}
        
        \item en problemas de regresión, esta capa tendrá tantas unidades como variables a predecir ---sin 
        función de activación, ya que esto limitaría el rango de valores posibles---;
        
        \item en problemas de clasificación, esta capa tendrá una sola unidad ---generalmente, con activación 
        sigmoide--- en clasificación binaria, o múltiples unidades ---con activación 
        \textit{softmax}\footnote{
            La activación \textit{softmax} no se aplica sobre la salida de una única unidad, sino que se 
            aplica sobre un vector de salidas de múltiples unidades, transformándolas en una distribución de 
            probabilidad, donde cada valor representa la probabilidad de pertenecer a una clase distinta y la 
            suma de todas las salidas es igual a 1.
        }--- en clasificación multiclase (véase la Figura \ref{fig:activation_func_classification}).
    \end{itemize}

\end{itemize}

\begin{figure}[h]
    \centering
    \includegraphics[width=\textwidth]{capitulos/cap_02/imagenes/ActivationFuncClassification.png}
    \caption{
        Diagrama de obtención de probabilidad en problemas de clasificación. 
        Adaptado de \cite{furnieles2022sigmoidandsoftmax}.
    } 
    \label{fig:activation_func_classification}
\end{figure}

%Quitar X y Out, y subir lo de abajo

\begin{figure}[h]
    \centering
    \includegraphics[width=0.95\textwidth]{capitulos/cap_02/imagenes/neural_network.png}
    \caption{
        Arquitectura simplificada de un MLP. 
        Recuperado de \cite{bre2017}.
    } 
    \label{fig:neural_network}
\end{figure}

% ------------------------------------------------------------------------------------------------------------

\subsection{Entrenamiento y validación de la red}

En el caso de las redes neuronales, el conjunto de datos suele dividirse en tres 
subconjuntos: entrenamiento, validación y prueba. A diferencia de métodos más tradicionales, no se utiliza 
validación cruzada, ya que entrenar redes profundas conlleva un elevado coste computacional.

Una vez hemos definido la arquitectura a emplear para resolver un problema, y definido los datos disponibles 
debemos entrenar la red con los datos de ejemplo. Este proceso implica ajustar los pesos del modelo para 
minimizar el error en las predicciones. 

El método de entrenamiento estándar en redes neuronales es el \textbf{algoritmo de retropropagación 
(\textit{backpropagation})}, que funciona en dos fases clave \cite{szeliski2010}:

\begin{itemize}

    \item \textbf{Propagación hacia adelante (\textit{forward pass})}: Los datos de entrada se procesan a 
    través de las capas de la red, generando una predicción. 

    \item \textbf{Propagación del error hacia atrás (\textit{backward pass})}: El error entre la 
    predicción y el valor real se calcula y se propaga hacia atrás en la red, ajustando los pesos mediante el 
    descenso de gradiente.
    
    Sin entrar en demasiado detalle, esto consiste en calcular el gradiente de la función de pérdida con           % Esto de "sin entrar en demasiado detalle" qué tal? 
    respecto a cada peso de la red, indicando cómo cada parámetro contribuye al error total. 
    A mayor aporte al error de un peso, más se ajustará ese peso. Así, el algoritmo priorizará modificar 
    significativamente los parámetros que más afectan al rendimiento de la red.
    
\end{itemize}

Este proceso explicado de manera vaga, tiene infinidad de detalles y variantes que influyen en su eficiencia y
eficacia:

\begin{itemize}

    \item El error obtenido entre la predicción y el valor real se calcula mediante la \textbf{función de 
    pérdida (\textit{loss function})}. Esta función cuantifica el error del modelo durante el entrenamiento, 
    midiendo la discrepancia entre las predicciones generadas y los valores o clases reales (\textit{ground 
    truth}).

    No se debe confundir con las métrica de evaluación de un modelo: aunque en algunos casos se pueden usar 
    métricas como funciones de pérdida y viceversa, las métricas destacan por ser fáciles de interpretar 
    y suele utilizarse más de una. En cambio, debe existir una únifa función de pérdida durante el 
    entrenamiento de una red neuronal, que debe cumplir tres requisitos clave:

    \begin{enumerate}

        \item Reflejar el objetivo del aprendizaje: Debe capturar adecuadamente qué significa ``éxito'' para 
        el modelo (p.ej., minimizar el error en regresión o maximizar la probabilidad de clasificación 
        correcta).

        \item Ser diferenciable: Es esencial para aplicar técnicas de descenso por gradiente, ya que el 
        optimizador necesita calcular derivadas.

        \item Ser eficiente computacionalmente: Dado que se evalúa en cada iteración del entrenamiento, su 
        cálculo debe ser rápido incluso con grandes volúmenes de datos.

    \end{enumerate}

    Mientras las métricas ayudan a entender el modelo, la función de pérdida es la que lo entrena.

    En problemas de regresión se emplean funciones de pérdida como el error cuadrático medio (\textit{mean 
    squared error}, MSE), que mide la diferencia promedio al cuadrado entre las predicciones y los valores 
    reales, o el error absoluto medio (\textit{mean absolute error}, MAE), que calcula la diferencia promedio 
    en valor absoluto\footnote{
        Aunque esta no es derivable en $x=0$, se define la derivada en ese punto como 0.
    }.

    En clasificación, las funciones de pérdida más comunes son la entropía cruzada (\textit{cross-entropy 
    loss}) para problemas de clasificación binaria y multiclase, que penaliza fuertemente las predicciones 
    incorrectas y ayuda a optimizar las probabilidades predichas para cada clase.


    \item Existen multitud de \textbf{algoritmos de optimización de parámetros}, como SGD, Adam o RMSProp. 
    Estos algoritmos determinan cómo actualizar los pesos del modelo durante el entrenamiento para minimizar 
    la función de pérdida. 
    Están basados en el descenso de gradiente, que ajusta los pesos en dirección opuesta al gradiente de 
    la función de pérdida respecto a los pesos, multiplicado por un factor escalar llamado \textbf{tasa 
    de aprendizaje (\textit{learning rate})}. Este hiperparámetro controla la magnitud de los pasos de 
    actualización: un valor demasiado alto puede hacer que el entrenamiento diverja, mientras que uno 
    demasiado bajo ralentiza la convergencia o estanca el modelo en mínimos locales.

    Existen estrategias avanzadas para ajustar el \textit{learning rate} de manera más eficiente durante el 
    entrenamiento, como la búsqueda de un \textit{learning rate} de punto de partida 


    \item Si bien existen métodos de entrenamiento de redes ejemplo a ejemplo ---como el Gradiente Descendente 
    Estocástico (SGD) puro\cite{bottou2010}---, estas se suelen entrenar por lotes 
    (\textit{minibatches})\footnote{
        Se denomina \textit{batch} al \textit{dataset} completo, y \textit{minibatch} a los subconjuntos de
        este cuyo tamaño está determinado por el hiperparámetro \textit{batch size}.
    } debido a ventajas clave, como el aprovechamiento de la paralelización de operaciones en GPU y una mayor 
    estabilidad en la función de pérdida al promediarse el error entre varios ejemplos. 
    Aún así, establecer un tamaño de lote óptimo no es una tarea trivial que requiere de encontrar un 
    equilibrio entre generalización y velocidad: los lotes grandes aceleran el entrenamiento pero pueden 
    reducir la generalización del modelo, mientras que los lotes pequeños puede presentar una gran varianza 
    que introduzca ruido en el modelo \cite{keskar2017}, si bien esto puede ayudar a escapar de mínimos 
    locales, y puede paliarse con un bajo \textit{learning rate} (aunque esto aumentaría todavía más los 
    tiempos de entrenamiento).
    

    \item Tras el uso de \textit{minibatches} en el entrenamiento, surge el concepto de \textbf{época 
    (\textit{epoch})}, que hace referencia a un ciclo completo de presentación de todos los datos de 
    entrenamiento a la red neuronal \cite{rusell2021}. Durante una época, los \textit{minibatches} se procesan 
    secuencialmente, actualizando los pesos del modelo en cada iteración (o \textit{step}) con el gradiente 
    calculado sobre un lote. Por ejemplo, si un conjunto de entrenamiento tiene 4096 ejemplos y el tamaño de 
    lote es 32, una época constará de 128 iteraciones (4096/32).

    El número de épocas es un hiperparámetro crítico: demasiadas pueden llevar a sobreajuste 
    (\textit{overfitting}), donde el modelo memoriza los datos de entrenamiento pero no generaliza bien; 
    demasiado pocas pueden resultar en infraajuste (\textit{underfitting}), donde el modelo no captura los 
    patrones subyacentes. Además, la combinación de tamaño de lote y épocas influye en la dinámica de 
    optimización, ya que lotes más pequeños requieren más pasos por época, introduciendo más ruido pero 
    potencialmente mejorando la exploración del espacio de pesos.

    En la práctica, se suele establecer un número muy alto de épocas, y monitorizar el error en un conjunto de
    validación para determinar cuándo detener el entrenamiento, evitando así el sobreajuste cuando el error de
    validación comienza a aumentar. A esta técnica se le denomina \textbf{\textit{early stopping}} 
    \cite{goodfellow2016}.
    
\end{itemize}


% ------------------------------------------------------------------------------------------------------------

\subsection{Redes Neuronales Convolucionales}

Como ya se venía anticipando, la arquitectura MLP es especialmente adecuada para trabajar con datos 
estructurados o tabulares, donde la información se organiza en una matriz en la que cada columna representa 
una característica concreta (como sexo, altura o peso). 
Sin embargo, su diseño presenta limitaciones clave: al manejar vectores de entrada de tamaño fijo y carecer 
de mecanismos para aprovechar relaciones espaciales o secuenciales, no es óptima para datos no estructurados, 
como imágenes o texto, donde cada elemento individual (un píxel o una palabra) carece de significado por sí 
mismo \cite{murphy2022}.

Por ejemplo, los patrones aprendidos en una posición de una imagen podrían no ser reconocidos en otra 
ubicación, ya que las entradas tienen un recorrido distinto dentro de la red. Por tanto, el modelo carecería       % DUDA: Esto de que las entradas tienen un recorrido distinto en la red se entiende?
de \textbf{invarianza traslacional}, puesto que los pesos no se comparten entre distintas posiciones, a lo que 
se suma una marcada ineficiencia por el elevado número de parámetros requeridos \cite{szeliski2010}.

Precisamente para estos casos, otras arquitecturas profundas resultan más apropiadas.
Las \textbf{redes neuronales convolucionales (\textit{Convolutional Neural Network} en inglés, CNNs)} son un 
tipo de DNN que, aprovechando las ventajas de las operaciones convolucionales, explotan los principios de 
localidad y correlación espacial. Esto les permite procesar imágenes (en 1D, 2D o 3D) de manera eficiente, 
interpretando patrones visuales jerárquicos que un MLP no podría capturar, y con significativamente menos 
parámetros.


\subsubsection{Capas convolucionales}

Como se ha introducido antes, el operador de \textbf{convolución} es la base de las CNNs. Este operador 
matemático aplica un \textbf{filtro} (también denominado \textit{kernel})\footnote{
    Aunque, como veremos después, a la hora de hablar de capas convolucionales, no son lo mismo.
} a regiones locales de una imagen de 
entrada, realizando un producto punto\footnote{El producto punto o producto escalar de dos vectores, se 
define como la suma de los productos componente a componente. 

$$
\mathbf{u} \cdot \mathbf{v} = \mathbf{u}_1 \cdot \mathbf{v}_1 + \mathbf{u}_2 \cdot \mathbf{v}_2 + ... + 
\mathbf{u}_n \cdot \mathbf{v}_n
$$
} 
entre los valores del filtro y los píxeles correspondientes de la imagen, y sustituyendo el valor del pixel 
central por el resultado del producto (véase la Figura \ref{fig:conv_op}).

\begin{figure}[h]
    \centering
    \includegraphics[width=0.95\textwidth]{capitulos/cap_02/imagenes/convolution_operation.jpg}
    \caption[Esquema gráfico de la aplicación de un filtro convolucional sobre una región de una imagen.]{
        Esquema gráfico de la aplicación de un filtro convolucional sobre una región de una imagen.
        Adaptado de \cite{nvidia2025convolutionoperation}.
    } 
    \label{fig:conv_op}
\end{figure}

Este proceso se repite al desplazar el filtro por toda la imagen mediante una \textbf{ventana deslizante}, 
generando un \textbf{mapa de activación}, que permite destacar líneas, curvas o texturas simples. Este mapa de
activación preserva la información de la localización de las características, si bien estas pueden ser 
detectadas en cualquier parte de la imagen. Esta propiedad se conoce como \textbf{equivarianza}. 

Las CNNs aprovechan la convolución mediante \textbf{capas convolucionales}. Cada capa convolucional está
compuesta por un conjunto de filtros convolucionales, donde cada uno a su vez tiene tantos \textit{kernels} 
como canales de entrada de la imagen haya en la capa (si es la primera capa convolucional, habrá 1 canal en 
imágenes de escala de grises, o 3 en imágenes RGB). El número de filtros en cada capa, su tamaño y la forma
en que se deslizan sobre la entrada\footnote{                                                                      
    Definidos mediante los parámetros de \textit{stride} y \textit{padding}, que controlan el desplazamiento       
    del filtro y la cantidad de relleno alrededor de la entrada, respectivamente.
}
\todo{
    ¿Se entiende aquí a qué se refiere con ``forma en la que se desliza sobre la entrada'' (la ventana 
    deslizante)?
}
\todo{
    ¿Explico más en profundidad los parámetros de stride y padding, o añado una imagen que deje clara la
    intuición detrás?
}
se determinan durante el diseño de la red, mientras que los valores de los \textit{kernels} son parámetros 
entrenables.

Cada filtro convolucional realiza la operación convolucional sobre cada canal con el \textit{kernel} que le 
corresponde. Después, se suman los mapas de activación de cada canal (pixel a pixel) añadiendo un sesgo 
(un mismo valor a todos los píxeles\footnote{
    Es por ello que no rompe la propiedad de equivarianza.
}), 
generando lo que denominamos como \textbf{mapa de características} (ya que idealmente extrae características 
relevantes). Los mapas de características generados con cada uno de los filtros son los nuevos canales, que
conforman la salida de la capa convolucional. Esta salida puede ser posteriormente procesada por otras capas,
permitiendo a la red aprender representaciones jerárquicas cada vez más abstractas de los datos de entrada:
las primeras capas convolucionales detectarán bordes, cambios de color o texturas básicas; a medida que 
avanzamos en las capas de la red, las combinaciones de estas características simples permite identificar 
formas más complejas, como objetos e incluso composiciones.

Sin embargo, hemos pasado por alto algo fundamental: ¿cómo reunimos la información de dos regiones distantes 
de una imagen en un mismo sitio? Una primera aproximación intuitiva nos diría que los filtros convolucionales 
deben ser progresivamente más grandes, para capturar patrones de mayor tamaño y contexto. No obstante, esto
incrementaría considerablemente el número de parámetros y, por tanto, aumentaría el coste computacional y 
aumentaría el riesgo de sobreajuste del modelo (ya que un modelo con más parámetros puede memorizar mejor los
datos de entrenamiento). Es por esto que, en aquellos problemas en los que no es necesario preservar la 
información de localización de las características, ---como en los que nos enfocamos en este trabajo: 
clasificación y regresión---, y, por tanto, el modelo sea invariante a la ubicación, se emplean técnicas de 
submuestreo (\textit{downsampling}) \cite{murphy2022}, como usar \textit{stride} mayor de 1 en los filtros
de las capas convolucionales o realizar \textit{pooling}\footnote{
    Nos centraremos en el último dado su amplio uso y fácil comprensión, además de su demostrada efectividad 
    empírica.
}.



\subsubsection{Capas de pooling}

Las \textbf{capas de agrupación (\textit{pooling layers})} tienen como objetivo principal comprimir la 
información de la imagen, reduciendo sus dimensiones (alto y ancho) mientras se preservan los datos más 
relevantes para la tarea. Esta reducción del tamaño espacial de los mapas de características disminuye el 
número de parámetros y operaciones en las fases posteriores, lo que reduce el coste computacional. Además, 
tiene un beneficio adicional: ayuda a prevenir el sobreajuste, ya que al limitar la cantidad de parámetros, 
el modelo evita memorizar ruido o detalles irrelevantes de los datos de entrenamiento, favoreciendo así el 
aprendizaje de patrones generalizables.

Hay diversos métodos de \textit{pooling}, entre los que destacan:

\begin{itemize}

    \item \textbf{Max pooling}, que calcula el máximo valor de regiones del mapa de características, y lo
    usa para crear un mapa de características reducido (véase la Figura \ref{fig:max_pooling}).

    \item \textbf{Average pooling}, que reemplaza el valor máximo del \textit{max pooling} por el cálculo de
    la media entre los valores de la región. 

\end{itemize}

La región de aplicación del \textit{pooling}, al igual que en la convolución, viene determinada por ciertos 
parámetros, definidos por el diseñador, como el tamaño de filtro (que suele ser de 2x2), el \textit{stride} 
y el \textit{padding}, si bien también existen variantes  adaptativas (\textit{adaptive}), que ajustan
automáticamente su cobertura para producir una salida con dimensiones específicas, independientemente del 
tamaño de la imagen de entrada. Esta funcionalidad es especialmente útil cuando se necesita adaptar los mapas
de características para conectarlos a una capa \textit{fully-connected}. 

\begin{figure}[h]
    \centering
    \includegraphics[width=0.7\textwidth]{capitulos/cap_02/imagenes/max_pooling.png}
    \caption{
        Esquema gráfico de \textit{max pooling} con un filtro 2x2 y \textit{stride} de 1.
        Recuperado de la Figura 14.12 de \cite{murphy2022}.
    } 
    \label{fig:max_pooling}
\end{figure}



\subsubsection{Capas \textit{Fully-Connected}}

Como hemos visto hasta ahora, en las CNNs, las primeras capas están diseñadas para extraer características
espaciales a través de filtros convolucionales y de \textit{pooling}. Sin embargo, una vez que se ha reducido 
la dimensionalidad y se han obtenido representaciones abstractas de alto nivel, es necesario realizar una 
predicción (en problemas de clasificación y regresión). 
Aquí es donde las \textbf{capas completamente conectadas (\textit{fully-connected}, FC)} juegan un papel 
crucial. Se utilizan en las últimas etapas de la red convolucional para combinar todas las características 
extraídas y producir una predicción final. Es decir, actúan como el clasificador/regresor\footnote{
    Si bien, independientemente de la tarea ---regresión o clasificación---, a esta parte de la red se le 
    denomina clasificador
} que toma todas las señales 
procesadas por las capas anteriores y predice la clase a la que pertenece la imagen o el valor objetivo. 

La arquitectura de esta capa sigue la estructura del MLP, con neuronas organizadas en una o más capas densas, 
donde cada neurona está conectada con todas las salidas de la capa anterior. Para que esto sea posible, 
primero se aplica una operación de \textbf{\textit{flattening}} que transforma el mapa de características 
multidimensional en un vector unidimensional. A partir de ahí, el procesamiento es equivalente al de una red 
neuronal tradicional: cada neurona calcula una combinación lineal de sus entradas seguida de una función de 
activación no lineal.


\subsubsection{Diseño de la CNN para problemas de clasificación y regresión}

Un patrón común de diseño de CNNs para la resolución de problemas de clasificación y regresión consta de dos
componentes principales:

\begin{itemize}

    \item el \textit{backbone} o extractor de características, que alterna capas convolucionales con capas de
    \textit{pooling}, cuya función es extraer representaciones jerárquicas y cada vez más abstractas de los 
    datos de entrada; y

    \item el \textit{classifier}, generalmente implementado mediante una o más capas totalmente conectadas, 
    toma estas representaciones para realizar la tarea específica de salida, ya sea clasificación o regresión.

\end{itemize}


\begin{figure}[h]
    \centering
    \includegraphics[width=0.95\textwidth]{capitulos/cap_02/imagenes/CNN_complete.png}
    \caption[
        Esquema gráfico de la arquitectura conocida como ``AlexNet'', diseñada para resolver un problema
        de clasificación con 1000 clases.
        Recuperado de la Figura 5.39 de \cite{szeliski2010}.
    ]{
        Esquema gráfico de la arquitectura conocida como ``AlexNet'', diseñada para resolver un problema
        de clasificación con 1000 clases.
        Recuperado de la Figura 5.39 de \cite{szeliski2010}.
        Esta arquitectura presenta una serie de capas convolucionales con funciones de activación no lineales 
        ReLU, max pooling, algunas capas totalmente conectadas y una capa final \textit{softmax}, la cual se 
        alimenta a una función de pérdida de entropía cruzada multiclase.
    } 
    \label{fig:CNN_complete}
\end{figure}


\subsubsection{Regularización y normalización}

Como en otras arquitecturas de redes neuronales, existen numerosas técnicas de regularización para evitar
el sobreajuste. Veamos algunas de las técnicas empleadas en CNNs:

\begin{itemize}

    \item \textbf{\textit{Data augmentation}} \cite{chen2019,zhang2021}: Consiste en añadir o modificar 
    dinámicamente ejemplos a partir de los que se tienen originalmente, de forma que se entrene la red con 
    un conjunto de datos más diverso y robusto, evitando el sobreajuste y mejorando la generalización.
    
    Algunas alteraciones realizadas pueden ser cambios en el nivel de brillo y contraste, rotaciones, 
    traslaciones, escalados o volteos de imágenes, entre otras. No existe configuración óptima, y su 
    configuración depende mucho del problema y las imágenes disponibles.

    Esta técnica sirve especialmente para problemas como clasificación o regresión, donde las clases o valores 
    predichos no suelen variar bajo pequeñas perturbaciones locales. 
    
    \item \textbf{\textit{Dropout}} \cite{srivastava2014}: Técnica que, durante el entrenamiento, ``apaga'' 
    (pone a cero) aleatoriamente un porcentaje de neuronas en cada iteración, evitando así que la red 
    dependa demasiado de determinadas unidades individuales (véase la Figura \ref{fig:net_with_dropout}). 
    En CNNs suele aplicarse a capas \textit{fully-connected}, 
    aunque existen variantes como \textit{Spatial Dropout} \cite{tompson2015} que elimina canales completos 
    en capas convolucionales, forzando una distribución más robusta de características.

    \begin{figure}[h]
        \centering

        \begin{subfigure}[b]{0.45\textwidth}
            \centering
            \includegraphics[width=\textwidth]{capitulos/cap_02/imagenes/net_without_dropout.png}
            \caption{Dropout desactivado}
            \label{fig:net_deactivate_dropout}
        \end{subfigure}
        \hfill
        \begin{subfigure}[b]{0.45\textwidth}
            \centering
            \includegraphics[width=\textwidth]{capitulos/cap_02/imagenes/net_with_dropout.png}
            \caption{Dropout activado ($p=0.5$)}
            \label{fig:net_activate_dropout}
        \end{subfigure}

        \caption[
            Diagrama del funcionamiento de neuronas con \textit{dropout}.
            Recuperado de la Figura 5.29 de \cite{szeliski2010}.
        ]{
            Diagrama del funcionamiento de neuronas con \textit{dropout}.
            Recuperado de la Figura 5.29 de \cite{szeliski2010}.
            Cuando se evalúa el modelo, todas las unidades funcionan correctamente 
            (\ref{sub@fig:net_deactivate_dropout}). Durante el entrenamiento, algunas son ``apagadas'' 
            (\ref{sub@fig:net_activate_dropout}). 
        }
        \label{fig:net_with_dropout}
    \end{figure}
    
    \item \textbf{\textit{Batch normalization}} \cite{ioffe2015}: Esta se introduce como una capa nueva a 
    añadir en el diseño de las redes, con nuevos parámetros entrenables: \textit{scale} y \textit{shift}. 
    Normaliza los valores de cada canal (media cero y desviación 1), y los reescala y desplaza en base a los
    valores de \textit{scale} y \textit{shift}. 
    Esto suaviza significativamente el espacio de valores de optimización \cite{santurkar2019} y reduce la 
    sensibilidad a la tasa de aprendizaje \cite{arora2018}, permitiendo establecer valores más altos.
    En CNNs se aplica típicamente después de las capas convolucionales y antes de la función de activación
    
\end{itemize}


\subsubsection{Conexiones residuales}

Uno de los principales problemas que no permite aumentar mucho la profundidasd de las redes convolucionales 
es el desvanecimiento de gradiente (\textit{vanishing gradient problem}), que consiste en la disminución 
exponencial de los gradientes durante el proceso de \textit{backpropagation} a medida que se retrocede hacia 
las capas iniciales de la red. Algunas de las soluciones a este problema han sido: utilizar funciones de 
activación ReLU, ya que evita gradientes pequeños para valores positivos; inicializar adecuadamente los pesos
de la red; o \textit{batch normalization}, que estabiliza la distribución de las activaciones. Sin embargo,
las conexiones residuales han sido la contribución más significativa para resolver este problema.

Las \textbf{redes residuales (\textit{residual nets}, ResNet)} 

%https://www.jeremyjordan.me/nn-learning-rate/



% ------------------------------------------------------------------------------------------------------------

\subsection{Transfer Learning}

El \textbf{aprendizaje por transferencia (\textit{transfer learning})} es una técnica que consiste en 
aprovechar el conocimiento aprendido por un modelo entrenado en una tarea como punto de partida para
mejorar el rendimiento y acelerar el entrenamiento en una nueva tarea relacionada \cite{rusell2021}.

En redes neuronales, el aprendizaje consiste en ajustar pesos, y en el caso del \textit{transfer learning}, 
estos pesos se inicializan con valores previamente optimizados para una tarea fuente, en lugar de comenzar con 
valores aleatorios (véase la Figura \ref{fig:fine-tuning}).

Se conoce como \textbf{fine-tuning} a la técnica de inicialización de los pesos de aquellas partes del modelo 
(como capas convolucionales) con los pesos previamente aprendidos, y que continúa el entrenamiento con los 
datos específicos de la nueva tarea. En este contexto, se denomina \textit{head} a las capas finales del 
modelo que se sustituyen para adaptarse a la nueva tarea. 

\begin{figure}[h]
    \centering
    \includegraphics[width=0.95\textwidth]{capitulos/cap_02/imagenes/fine-tunning.png}
    \caption[
        Diagrama de \textit{fine-tuning} de un modelo en una nueva tarea. 
        Recuperado de la Figura 19.2 de \cite{murphy2022}.
    ]{
        Diagrama de \textit{fine-tuning} de un modelo en una nueva tarea. 
        Recuperado de la Figura 19.2 de \cite{murphy2022}.
        La capa final de salida es entrenada desde cero para la nueva tarea. El resto de capas son 
        inicializadas con los pesos previos. 
    } 
    \label{fig:fine-tuning}
\end{figure}

Por ejemplo, en \cite{venema2022} se utilizan dos modelos de CNN preentrenados en clasificación con ImageNet 
(que contiene imágenes de 1000 clases): VGG16 y ResNet50. Estos modelos se ajustan (\textit{fine-tuning}) 
para estimar el sexo de una persona a partir de radiografías de húmero. Aunque ambas tareas parecen muy 
diferentes, las primeras capas de la red, especializadas en detectar características generales como bordes y 
texturas, pueden ser útiles en los dos casos, lo que permite una transferencia efectiva del conocimiento.

El \textit{fine-tuning} puede aplicarse de forma gradual: primero se entrena solo el \textit{head} 
(manteniendo el resto del modelo congelado) y luego, si es necesario, se afinan también algunas capas 
preentrenadas para mejorar el rendimiento en la tarea específica.


% ------------------------------------------------------------------------------------------------------------
% INCERTIDUMBRE ----------------------------------------------------------------------------------------------
% ------------------------------------------------------------------------------------------------------------

\section{Incertidumbre}

La metrología\footnote{
    Ciencia de las mediciones y sus aplicaciones \cite{jcgm200:2012}.
}, y la estadística comparten un papel fundamental en el análisis del error y la incertidumbre en campos como 
el ML. Mientras la metrología establece los fundamentos conceptuales de error e incertidumbre, la estadística 
proporciona métodos para cuantificar, modelar y reducir estos factores durante el desarrollo y validación de 
modelos.

El Comité Conjunto de Guías en Metrología (\textit{Joint Committe for Guides in Metrology}, JCGM)\footnote{
    Este Comité está formado por numerosas organizaciones internacionales de metrología y normalización: 
    BIPM, IEC, IFCC, ISO, IUPAC, IUPAP, OIML e ILAC. Su objetivo principal es mantener y promover las 
    guías internacionales clave en metrología, como la Guía para la Expresión de la Incertidumbre en la 
    Medición (\textit{Guide to the Expression of Uncertainty in Measurement}, GUM) \cite{jcgm100:2008} y el 
    Vocabulario Internacional de Metrología (\textit{Vocabulaire international de métrologie}, VIM) 
    \cite{jcgm200:2012}.
}
\todo{
    ¿Sobra la concreción de las siglas (principalmente en francés)? Es que en el resto del documento siempre 
    he introducido los términos de los que procedía unas siglas. 
}
define el \textbf{error} como una ``medición imperfecta'' de la magnitud observada, que puede estar causada
por efectos aleatorios (componente aleatoria del error) y por efectos sistemáticos (componente sistemática del 
error, más conocida como \textbf{sesgo}).
Por otro lado, define a la \textbf{incertidumbre} como ``parámetro, asociado con el resultado de una medición, 
que caracteriza la dispersión de los valores que podrían atribuirse razonablemente al \textbf{mensurado}, que
es como se denomina a la magnitud a ser medida. [...] El parámetro puede ser, por ejemplo, una desviación 
estándar, o la anchura de un intervalo con un nivel de confianza establecido'' \cite{jcgm100:2008}.

Partiendo de estas definiciones generales, veamos las diferencias entre los dos enfoques principales en la 
evaluación de mediciones: el enfoque basado en el error y el enfoque basado en la incertidumbre.

El \textbf{enfoque basado en el error} o enfoque tradicional parte de la premisa de que existe un valor 
verdadero. En consecuencia, el propósito de la medición es aproximarse lo más posible a dicho valor, 
minimizando las distintas componentes del error \cite{jcgm100:2008}:

\begin{itemize}
    \item para el error aleatorio, esto se logra aumentando el número de observaciones, ya que su distribución 
    tiene una media igual a cero; y

    \item para el error sistemático, es necesario identificarlo y cuantificar su magnitud, lo que permite 
    aplicar factores de corrección que compensen su efecto.

\end{itemize}

Se asume que el resultado de la medición ha sido corregido por todos los efectos sistemáticos identificados
como significativos, de modelo que la esperanza matemática de esta componente sea igual a cero.

Sin embargo, en la práctica no existen reglas claras para distinguir las componentes del error ni cómo estas 
se combinan en el error total
 que permitan diferenciar claramente las componentes del error ni
cómo estas se combinan en el error total. En general, solo es posible estimar un límite superior del valor 
absoluto del error total estimado, al que se denomina de forma inapropiada ``incertidumbre''. 

Frente a enfoque anterior, se presenta el \textbf{enfoque basado en la incertidumbre} \cite{jcgm100:2008}, 
cuyo propósito no es hallar el mejor valor posible, sino establecer un intervalo de valores razonables para el 
mensurando, el cual puede refinarse con información adicional. Así, la medición misma se convierte en una 
herramienta para determinar el error del instrumento. 

% ------------------------------------------------------------------------------------------------------------

\subsection{Intervalos de valores razonables}

Veamos qué tipos de intervalos de valores nos permiten cuantificar la variabilidad de los resultados y, por 
tanto, la incertidumbre de la medición realizada. 

\begin{itemize}

    \item El \textbf{intervalo de confianza (IC)} es una herramienta común de la estadística 
    frecuentista\footnote{
        La estadística frecuentista ... ¿Anexo explicando las diferencias entre frecuentista y bayesiana?
        (para agosto)
    },
    que permite estimar un rango de valores tal que podamos confiar en que contiene al valor verdadero de 
    un parámetro poblacional desconocido $\theta$ (p.ej., la media) \cite{berrendero2025}.

    Los métodos del cálculo del IC dependen de la distribución del estimador (p.ej., la distribución de la
    media muestral) y los parámetros conocidos. 

    Es importante aclarar un malentendido común: un intervalo de confianza con nivel 95\% para un parámetro 
    $\theta$ no significa que exista un 95\% de probabilidad de que $\theta$ esté dentro del intervalo 
    calculado a partir de una muestra específica. En realidad, el 95\% se refiere a la frecuencia con la que 
    , si muestreásemos muchas veces los datos, los intervalos construidos a partir de esas muestras incluirían 
    al valor verdadero de $\theta$ en aproximadamente el 95\% \cite{murphy2022}.


    \item El \textbf{intervalo de credibilidad o región creíble (RC)} es, de hecho, la que determina que el 
    parámetro $\theta$ está contenido en el rango de sus valores con una probabilidad determinada por la 
    confianza. Este intervalo es la aproximación bayesiana equivalente al intervalo de confianza, y, como 
    este, requiere conocer la distribución a priori de los datos.

    La diferencia radica en que, a diferencia del intervalo de confianza, que parte de que $\theta$ es un 
    parámetro fijo desconocido y los datos son tratados como aleatorios, el enfoque bayesiano fija los datos
    (ya que son conocidos) y el parámetro $\theta$ lo trata como aleatorio (ya que es desconocido) 
    \cite{murphy2022}.

    Esta interpretación resulta más intuitiva y directa en comparación con la interpretación frecuentista del 
    intervalo de confianza. En particular, una región creíble del 95\% sí puede interpretarse como que hay un
    95\% de probabilidad de que el parámetro $\theta$ se encuentre dentro de ese intervalo, dado el conjunto
    de datos observado y la distribución a priori asumida.


    \item El \textbf{intervalo de predicción (\textit{prediction interval})} es radicalmente diferente a los 
    intervalos previos, pues trata de predecir un valor futuro de una observación, no determinar un parámetro 
    poblacional. Existen numerosos métodos, con mayores y menores garantías estadísticas, con y sin 
    necesidad de conocer la distribución de los datos. El enfoque más prometedor es la predicción conformal, 
    que han demostrado ser eficaz en contextos donde los supuestos clásicos (normalidad, homocedasticidad) no 
    se cumplen \cite{romano2019}, y es actualmente el enfoque más robusto para la construcción de intervalos 
    de predicción en aplicaciones modernas de ML 
    \cite{romano2019, luo2025, sadinle2019, romano2020, angelopoulos2020}.

\end{itemize}

Como podemos esperar, a más estrecho sea el intervalo que manejemos, más se puede confiar en las predicciones,
pero no todos los tipos de intervalos revelan la misma información sobre incertidumbre. 

% ------------------------------------------------------------------------------------------------------------

\subsection{Cuantificación de la incertidumbre en \textit{machine learning}}

El enfoque basado en incertidumbre se puede extrapolar a modelos de ML. En problemas de regresión, la analogía
es directa: es deseable que los modelos no solo proporcionen una predicción puntual, sino también un intervalo 
que indique el grado de incertidumbre asociado a cada predicción, conocido como \textbf{intervalo de 
predicción (\textit{prediction interval})}. En caso de los problemas de clasificación, el concepto equivalente 
al de intervalo de predicción se denomina \textbf{conjunto de predicción (\textit{prediction set})}.

\todo{¿Cómo conecto estos dos párrafos?}

Podríamos incluso desglosar los tipos de incertidumbre según su origen, pero esto requeriría de conocimiento
específico de cada problema para el que se aplique. Sí está más aceptada la clasificación de incertidumbre en
aleatoria y epistémica \cite{hullermeier2021}:

\begin{itemize}
    
    \item La incertidumbre aleatoria es la relativa al dato individual. 
    Esta incertidumbre se debe a la variabilidad inherente del fenómeno observado y no puede reducirse, aunque 
    se disponga de más datos. Por ejemplo, en un entorno médico, puede reflejar la variabilidad entre 
    pacientes con condiciones similares.

    \item La incertidumbre epistémica es la causada por falta de conocimiento o precisión del modelo.
    Se relaciona con aspectos como la escasez de datos, la calidad de la información disponible o la capacidad 
    limitada del modelo para generalizar. A diferencia de la incertidumbre aleatoria, la epistémica es 
    reducible: puede disminuirse con más datos, mejores modelos o mayor comprensión del problema.
    
\end{itemize}

A estos, se le puede añadir un tercero: el \textit{drift}, que procede de cambios en la distribución de los 
datos a lo largo del tiempo, ya sea en la distribución de las variables de entrada, en la distribución de las
variables de salida, o en la relación entre las dos previas. Por ejemplo: una imagen de entrada a un modelo de
clasificación que no corresponde a ninguna clase con la que se haya entrenado anteriormente, o  


% Relacionándolo con el apartado anterior, los intervalos de confianza solo informan de la incertidumbre 
% aleatoria, puesto que asumen que el modelo está bien especificado y que el error residual es la única fuente 
% de variabilidad. 

% Los intervalos de credibilidad sí integran la incertidumbre en los parámetros del modelo y en la estructura 
% del modelo mismo, capturando así tanto la incertidumbre aleatoria como la incertidumbre epistémica.

% Frente a estos, los intervalos de predicción obtenidos mediante predicción conformal sí presentan información
% de los tres componentes de incertidumbre. 




\begin{figure}[h]
    \centering
    \includegraphics[width=\textwidth]{capitulos/cap_02/imagenes/adversarial_example.png}
    \caption[
        Ejemplo adverario mal clasificado por un modelo ML entrenado con datos textuales.
        Adaptado de la Figura 2 de \cite{hullermeier2021}, original de \cite{sato2018}.
    ]{
        Ejemplo adverario mal clasificado por un modelo ML entrenado con datos textuales.
        Adaptado de la Figura 2 de \cite{hullermeier2021}, original de \cite{sato2018}.
        Se observa que el cambio de una sola palabra ---y aparentemente sin mucha relevancia--- (destacada en 
        negrita) basta para cambiar la predicción de ``sentimiento negativo'' a ``sentimiento positivo''.
    } 
    \label{fig:adversarial_example}
\end{figure}

% ------------------------------------------------------------------------------------------------------------

\subsection{Calibración de modelos}

% Qué es la calibración de modelos

% Modelos de clasificación: Platt Scaling y Temperature Scaling

% Modelos de regresión: Quantile Regression

% Preludio de la Conformal Prediction





% ------------------------------------------------------------------------------------------------------------
% CONFORMAL PREDICTION ---------------------------------------------------------------------------------------
% ------------------------------------------------------------------------------------------------------------

\section{Conformal Prediction}


% ------------------------------------------------------------------------------------------------------------

\subsection{Conformal Prediction en problemas de regresión}



% ------------------------------------------------------------------------------------------------------------

\subsection{Conformal Prediction en problemas de clasificación}

...

\subsubsection{Least-Ambiguous Set-Valued Classifiers}

...

\subsubsection{Adaptive Prediction Sets}

El \textbf{\textit{Adaptive Prediction Sets} (APS)} \cite{romano2020}

\subsubsection{Regularized Adaptive Prediction Sets}

\textbf{\textit{Regularized Adaptive Prediction Sets} (RAPS)} \cite{angelopoulos2020} es una variante del método APS, 
que 

añade una penalización a conjuntos de predicción demasiado grandes, realizando esto a través 
de la suma de un componente 





   \chapter{Estado del arte}


\section{Estimación de la edad en antropología forense}





\section{Estimación de la edad en antropología forense usando Machine Learning}





\section{Cuantificación de incertidumbre en antropología forense}








   \chapter{Materiales y métodos} 

% ------------------------------------------------------------------------------------------------------------
% ------------------------------------------------------------------------------------------------------------

\section{Conjunto de datos disponibles}

Disponemos de un conjunto de datos compuesto por radiografías panorámicas maxilofaciales de individuos de 
diversos países y continentes (véase en la tabla \ref{tab:instituciones_fuente_dataset}), obtenidas con
distintos modelos de máquinas de rayos X
\footnote{
    Los modelos empleados fueron: \textit{Planmeca Promax Digital Panoramic}; \textit{Sirona ORTHOPHOS-XG}, 
    \textit{ORTHOPHOS-DS}, y \textit{SIDEXIS}. Las constantes radiológicas usadas fueron de 66 a a 70 kV, 7 a 
    11 mA, y 15 s.
}.
Este conjunto de datos ha sido sido proporcionados por Panacea Cooperative Research, empresa \textit{spin-off} 
de la Universidad de Granada.  

\begin{table}[h]
\begin{tabular}{@{}clc@{}}
\toprule
País                                                            & Instituciones                                                                                                                                                            & Nº de ejemplos \\ \midrule
\begin{tabular}[c]{@{}c@{}}Bosnia y \\ Herzegovina\end{tabular} & Universidad de Sarajevo                                                                                                                                                  & 882            \\ \hline
Botsuana                                                        & \begin{tabular}[c]{@{}l@{}}Dos clínicas dentales privadas en \\ Garobone\end{tabular}                                                                                    & 1242           \\ \hline
Chile                                                           & \begin{tabular}[c]{@{}l@{}}Dos clínicas dentales privadas en \\ Santiago y Rancagua\end{tabular}                                                                         & 1016           \\ \hline
\begin{tabular}[c]{@{}c@{}}República \\ Dominicana\end{tabular} & \begin{tabular}[c]{@{}l@{}}Tres clínicas dentales privadas en \\ Santo Domingo, La Vega y Santiago\end{tabular}                                                          & 541            \\ \hline
Japón                                                           & \begin{tabular}[c]{@{}l@{}}Department of Forensic Sciences, \\ Iwate Medical University, Iwate\end{tabular}                                                              & 1045           \\ \hline
Corea                                                           & Catholic University of Korea, Seoul                                                                                                                                      & 500            \\ \hline
Malasia                                                         & \begin{tabular}[c]{@{}l@{}}Faculty of Dentistry Universiti Teknologi \\ MARA Selangor Branch, Selangor\end{tabular}                                                      & 667            \\ \hline
Turquía                                                         & \begin{tabular}[c]{@{}l@{}}Department of Dentomaxillofacial \\ Radiology, Baskent University, Turkey\end{tabular}                                                        & 2323           \\ \hline
Uganda                                                          & \begin{tabular}[c]{@{}l@{}}Department of Dental Morphology with \\ the Université Claude Bernard Lyon 1, \\Faculté d’odontologie, Lyon\end{tabular}                      & 283            \\ \hline
Italia                                                          & \begin{tabular}[c]{@{}l@{}}Department of Surgical Sciences, \\ University of Cagliari\end{tabular}                                                                       & 173            \\ \hline
Kosovo                                                          & \begin{tabular}[c]{@{}l@{}}University Dentistry Clinical Center, \\ Pristina\end{tabular}                                                                                & 1397           \\ \hline
Líbano                                                          & Clínica dental privada en Beirut                                                                                                                                         & 690            \\ \bottomrule
\end{tabular}
\caption[
    Instituciones participantes en la recolección de datos e imágenes
]{   
    Lista de instituciones participantes en la recolección de los datos e imágenes dentales utilizados en el 
    trabajo.
}
\label{tab:instituciones_fuente_dataset}
\end{table}

Este \textit{dataset} incluye:

\begin{itemize}

    \item datos tabulares (en formato CSV), donde cada fila representa un ejemplo (un individuo), con los 
    siguientes campos: un identificador único, sexo del individuo, edad del individuo y ``sample'' 
    (clasificación según el origen geográfico de la radiografía).

    \item imágenes bidimensionales de radiografías panorámicas maxilofaciales, con una imagen asociada 
    a cada individuo y se identifica mediante su ID único. 

\end{itemize}

Se proporcionan los datos ya preprocesados, por lo que no es necesario realizar tareas adicionales de limpieza 
o transformación previa antes de su análisis.

Se ha ignorado el campo ``sample'', dado que se trata de una asignación sesgada y no representa 
necesariamente una clasificación fiable del origen poblacional de los individuos. Por tanto, este campo no 
se emplea en el análisis ni en el entrenamiento de los modelos, centrándose exclusivamente en las variables 
de edad, sexo e imagen.

En el \textit{dataset} hay un total de 10.739 ejemplos, de los que 5.756 son de individuos de sexo femenino 
y 4.983 de sexo masculino. 
Las edades mínima y máxima son 14 y 26 años, respectivamente, y la media son 19,13 años.
En la Figura \ref{fig:histogram_ages} se observa que el número de ejemplos por edad se mantiene relativamente 
constante desde los 14 hasta los 21 años, a partir de los cuales disminuye progresivamente, con una 
representación notablemente menor en los grupos de 24, 25 y 26 años.
 
\begin{figure}[h]
    \centering
    \includegraphics[width=0.7\textwidth]{capitulos/cap_04/imagenes/histogram_ages.png}
    \caption[
        Histograma de edad de los individuos del conjunto de datos disponible.
    ]{
        Histograma de edad de los individuos del conjunto de datos disponible. 
        Elaboración propia.
    } 
    \label{fig:histogram_ages}
\end{figure}

En la Figura \ref{fig:kde_and_boxplot_ages_sex} podemos comprobar cómo en términos relativos la distribución 
de edad por sexo es muy similar, compartiendo ambas prácticamente el mismo rango de edades y patrones de 
dispersión, sin observarse diferencias sustanciales en la mediana ni en la forma general de las 
distribuciones.

\begin{figure}[h]
    \centering
    \includegraphics[width=\textwidth]{capitulos/cap_04/imagenes/kdeplot_ages.png}
    \caption[
        Gráficas de densidad y de caja de edad por sexo de los individuos del conjunto de datos disponible.
    ]{
        Gráficas de densidad y de caja de edad por sexo de los individuos del conjunto de datos disponible. 
        Elaboración propia.
    } 
    \label{fig:kde_and_boxplot_ages_sex}
\end{figure}

En conclusión, el dataset presenta en general un buen balance entre clases y edades, lo que permite un 
análisis representativo de la población incluida. No obstante, será necesario examinar con mayor detalle la 
infrarepresentación de los grupos de mayor edad, especialmente a partir de los 22 años, para evaluar su 
posible impacto en el rendimiento y generalización de los modelos entrenados.

Se proporcionan los datos ya divididos en \textit{train} ---con un 80\% de los individuos--- y \textit{test}
---con el 20\% restante---, con la intención de que puedan ser utilizados para entrenar y evaluar modelos de 
predicción. En la Figura \ref{fig:kde_ages_train_test} se puede observar cómo existe una distribución 
edad-sexo similar en los datos de ambos subconjuntos, por lo que se puede asumir que la partición respeta la 
representatividad de la población original, favoreciendo una evaluación más realista del rendimiento de los 
modelos en datos no vistos.

\begin{figure}[h]
    \centering

    \begin{subfigure}[b]{0.47\textwidth}
        \centering
        \includegraphics[width=\textwidth]{capitulos/cap_04/imagenes/kde_ages_F.png}
        \caption{Distribución de edad de individuos de sexo femenino.}
        \label{fig:kde_ages_F}
    \end{subfigure}
    \hfill
    \begin{subfigure}[b]{0.47\textwidth}
        \centering
        \includegraphics[width=\textwidth]{capitulos/cap_04/imagenes/kde_ages_M.png}
        \caption{Distribución de edad de individuos de sexo masculino.}
        \label{fig:kde_ages_M}
    \end{subfigure}

    \caption[
        Distribución de edad de los individuos del conjunto de datos disponible por sexo.
    ]{
        Distribución de edad de los individuos del conjunto de datos disponible por sexo. 
        Elaboración propia.
    }
    \label{fig:kde_ages_train_test}
\end{figure}


% ------------------------------------------------------------------------------------------------------------
% ------------------------------------------------------------------------------------------------------------

\section{Problemas planteados}

Como se ha anticipado anteriormente, este trabajo se centrar en los problemas de estimación de edad y 
(estimación de mayoría/minoría de edad o clasificación de grupos etarios?).
Analicemos más en profundidad estos dos. 

% ------------------------------------------------------------------------------------------------------------

\subsection{Problema de estimación de edad}


El problema de \textbf{estimación de edad (\textit{age estimation}, AE)} consiste en predecir la edad 
cronológica de un individuo en una escala continua, lo que lo define como un problema de regresión.

En este trabajo se plantea dos variantes del problema (véase la Figura \ref{fig:regression_problems}): 
\begin{itemize}
    \item Una que solo tenga como entrada imágenes de radiografías panorámicas maxilofaciales.
    \item Otra que tenga como entrada tanto las imágenes de radiografías como el sexo del invididuo. 
\end{itemize}

\begin{figure}[h]
    \centering
    \includegraphics[width=\textwidth]{capitulos/cap_04/imagenes/regression_problems.png}
    \caption[
        Esquema visual de los modelos de regresión propuestos. 
        Elaboración propia.
    ]{
        Esquema visual de los modelos de regresión propuestos. 
        Elaboración propia.
        El primer modelo solo tiene radiografías maxilofaciales como entrada. 
        El segundo tiene tanto las radiografías maxilofaciales como el sexo de cada inviduo. 
    } 
    \label{fig:regression_problems}
\end{figure}

Se espera que la información adicional del sexo, como mínimo mantenga el desempeño del modelo, y 
potencialmente lo mejores, dado que el crecimiento y desarrollo óseo varía entre hombres y mujeres
\cite{adserias2019, scheuer2000}, lo que sugiere que incluir el sexo como variable de entrada podría ayudar 
al modelo a ajustar sus predicciones de manera más precisa. 


% ------------------------------------------------------------------------------------------------------------

\subsection{Estimación de minoría/mayoría de edad}

El problema inmediatamente derivado del anterior es la 
\textbf{estimación de minoría/mayoría de edad (\textit{assessment of the age of majority}, AAM)}.
Este se trata de un problema de clasificación binaria, ...

\todo{Por completar (JULIO)}


% ------------------------------------------------------------------------------------------------------------
% ------------------------------------------------------------------------------------------------------------

\section{Métodos propuestos}

\subsection{Arquitectura empleada}

El primer problema propuesto es el de estimación de edad. Partiremos de un planteamiento muy simple: imágenes 
bidimensionales de las radiografías panóramicas maxilofaciales ---y sexo, opcionalmente---
como entrada, y estimación de edad a la salida.

Como modelo, empleamos una CNN, dado su buen desempeño en tareas de visión por computador. Específicamente,
implementamos la arquitectura ResNeXt50 \cite{xie2017}, utilizando un modelo entrenado con el \textit{dataset} 
ImageNet
\todo{¿Debería entrar en detalle de por qué ResNeXt50? ¿Comparar sus capacidades con otras arquitecturas CNN?}
\footnote{
    El dataset Imagenet contiene 1.000 clases de objetos. Estas clases abarcan una
    amplia variedad de categorías, como animales (\textit{tiger}, \textit{koala}, \textit{zebra}, ...), 
    vehículos (\textit{ambulance}, \textit{airliner}, \textit{mountain bike}, ...), alimentos 
    (\textit{strawberry}, \textit{pizza}, \textit{bagel}, ...), entre otras. 
}
\cite{deng2009} como punto de partida. Este modelo preentrenado es accesible a través de Pytorch. 

Aunque ResNeXt50 fuera diseñado originalmente para un problema de clasificación de imágenes y entrenado con
un dominio distinto al de nuestro problema, su adaptación a una tarea de estimación de edad es sencilla: 
reemplazar su cabecera de clasificación por una de regresión. Además, el uso de peso preentrenados proporciona
una inicialización más robusta que el entrenamiento desde cero, ya que el modelo ya ha aprendido filtros 
genéricos para detectar características visuales básicas, como bordes o texturas.

% ------------------------------------------------------------------------------------------------------------

\subsection{Regresión cuantílica}

\todo{No sabía si incluir este apartado en Fundamentos teóricos, pero decidí incluirlo aquí porque es una 
técnica más específica, que técnicamente no diverge mucho de lo explicado en los Fundamentos Teóricos, 
¿Qué opina?}

La \textbf{regresión cuantílica (\textit{quantile regression}, QR)} es un tipo de regresión que, a diferencia
de la regresión puntual, predice intervalos o cuantiles específicos de la distribución de la variable 
respuesta, en lugar de solo su media. 
Esta técnica permite modelar límites inferiores y superiores (por ejemplo, el percentil 10\% y 90\%) para
capturar la incertidumbre o heterocedasticidad en los datos.

No debe confundirse con una técnica de UQ, ya que no modela explícitamente la incertidumbre epistémica ni 
proporciona garantías estadísticas de cobertura como lo hacen los métodos de predicción conformal, aunque 
puede utilizarse como parte de un enfoque para cuantificar la incertidumbre aleatoria o 
condicional al estimar intervalos de predicción directamente a partir de los datos.

Esta técnica de regresión puede implementarse en modelos de redes neuronales y modelos tipo \textit{ensemble}, 
aunque su implementación difiere significativamente. 

En redes neuronales, esta regresión requiere de:

\begin{itemize}

    \item Definir una capa de salida con múltiples neuronas, una por cada cuantil deseado. Por ejemplo, para 
    obtener una región del 90\% con predicción puntual, tendríamos los cuantiles 0.05 y 0.95 para los límites
    inferior y superior, respectivamente, y 0.5 para la predicción puntual. 

    \item Cambiar la función de pérdida por una que admite varias salidas. En general, se suele utilizar la 
    la pérdida \textit{pinball} \cite{steinwart2011}.

    La \textbf{función de pérdida \textit{pinball}} es una generalización de la función de pérdida del MAE,
    que penaliza las predicciones de manera asimétrica según el cuantil objetivo. Para un cuantil 
    $\tau \in \left( 0,1\right)$, se define como:

    \todo{¿Debería incluir la función de pérdida en el apartado de entrenamiento en el capítulo 5?}

    $$
    L_\tau(y,\hat{y}) = \left\{
        \begin{array}{rcl}
            \tau \cdot (y-\hat{y}) & \mbox{si} & y \ge \hat{y}
            \\
            (1-\tau) \cdot (\hat{y}-y) & \mbox{si} & y < \hat{y}
        \end{array}
    \right.
    $$

    En la Figura \ref{fig:pinball_loss} podemos apreciar la penalización asimétrica para errores positivos y 
    negativos. 

    \begin{figure}[h]
        \centering
        \includegraphics[width=0.5\textwidth]{capitulos/cap_04/imagenes/pinball_loss.png}
        \caption[
            Visualización de la función de pérdida \textit{pinball} para cada valor de error.
        ]{
            Visualización de la función de pérdida \textit{pinball} para cada valor de error.
            Adaptado de la Figura 1 de \cite{romano2019}.
            Esta concretamente muestra la función de pérdida para un cuantil cercano a cero, 
            ya que es más permisivo con los errores positivos que con los negativos.
        } 
        \label{fig:pinball_loss}
    \end{figure}

    Esta función de pérdida, aplicada a múltiples salidas (cada una asociada a un cuantil específico), busca 
    que las predicciones del modelo cubran la proporción deseada de los datos dentro del intervalo definido 
    por los cuantiles. Por ejemplo:

    \begin{itemize}

        \item Para $\tau = 0.05$ y $\tau = 0.95$, el modelo intentará que el 90\% de las observaciones reales 
        ($y$) caigan entre los límites predichos ($\hat{y}_{0.05}$ y $\hat{y}_{0.95}$).

        \item La mediana ($\tau = 0.5$) proporciona una predicción central robusta, equivalente a minimizar 
        el MAE. 
        
    \end{itemize}

    De esta forma, la función de pérdida se puede expresar como la media de las pérdidas para cada cuantil.

\end{itemize}

Este tipo de regresión entonces da una estimación puntual $\hat{y}$ y una estimación interválica formado por 
límites inferior y superior $\left[ \hat{y}_{lower}, \hat{y}_{upper} \right]$. Este enfoque es ampliamente
aplicable y obtiene intervalos adaptativos a la heterocedasticidad de los datos \cite{romano2019}. 
Sin embargo, no tiene garantías estadísticas de cobertura bajo distribuciones arbitrarias de errores.
Es por ello que se requiere de herramientas adicionales para garantizar la cobertura.

% ------------------------------------------------------------------------------------------------------------

\subsection{Métodos de predicción conformal para regresión}

Todos los métodos propuestos en este trabajo son \textit{split calibration}, es decir, los datos de 
entrenamiento se dividen en dos subconjuntos: entrenamiento y calibración. No hemos implementado técnicas 
\textit{cross-calibration} como \cite{barber2021} dado que requieren un mayor coste computacional.
Además, en los experimentos preliminares, \textit{split calibration} demostró ser suficiente para obtener
valores razonablemente buenos de cobertura marginal y una eficiencia adecuada en los intervalos de predicción.

% ------------------------------------------------------------------------------------------------------------

\subsubsection{\textit{Inductive Conformal Prediction} (ICP)}

La ICP \cite{papadopoulos2002} fue la primera técnica de predicción conformal desarrollada para problemas de 
regresión. 

El procedimiento sigue dos fases:

\begin{itemize}

    \item \textbf{Calibración}:
    
    \begin{itemize}
        \item Se calculan las puntuaciones no conformidad $R$ sobre los datos del conjunto de calibración como el 
        error absoluto entre el valor predicho y el real:
        $$
        R = \left\{ | y_i - \hat{f}(x_i) | \right\}_{i=1,...,n_{calib}}
        $$

        \item Se calcula un umbral de no conformidad para un nivel de confianza dado $q_{1-\alpha}$, como el 
        cuantil $(1-\alpha)(1+1/n)$ de $R$:
        $$
        q_{1-\alpha} = Quantile_{ \lceil  (1-\alpha) (1 + 1/n)  \rceil } ( R )
        $$

    \end{itemize}
    
    \item \textbf{Inferencia conformal}:
    
    \begin{itemize}

        \item Se obtiene la inferencia puntual $\hat{f}(x_{n+1})$ de un nuevo ejemplo con el modelo.  
        
        \item Se construyen los intervalos de predicción como:
        $$
        \hat{C_\alpha}(x_{n+1}) = \left[ \hat{f}(x_{n+1}) - q_{1-\alpha}, \hat{f}(x_{n+1}) + q_{1-\alpha}\right]
        $$

    \end{itemize}

\end{itemize}

Este método de CP presenta varias ventajas: 

\begin{itemize}
    \item \textbf{\textit{Model-agnostic} y \textit{domain-agnostic}}: Es independiente tanto del modelo como 
    del dominio, ya que no utiliza representaciones internas del modelo ni de las entradas. 
    
    \item \textbf{Bajo coste computacional}: Solo añade coste computacional en la calibración, con el 
    cálculo de puntuaciones de no conformidad en calibración $\left( \mathcal{O}(n_{calib}) \right)$ y
    cálculo del cuantil empírico $\left( \mathcal{O}(n_{calib} \log n_{calib}) \right)$. La inferencia
    conformal mantiene el mismo orden que el modelo base ($\mathcal{O}(1)$ por predicción). 

\end{itemize}

Sin embargo, también presenta importantes limitaciones: 

\begin{itemize}
    
    \item \textbf{Intervalo simétrico y no adaptativo}: El intervalo es simétrico, además de tener siempre el 
    mismo ancho ($2q_{1-\alpha}$), no permitiendo adaptarse a la incertidumbre específica de la predicción. 

    \item \textbf{Sensibilidad a datos ruidosos o OOD}: 
    Si el conjunto de calibración contiene \textit{outliers} o viola el supuesto de intercambiabilidad, el 
    umbral \(q_{1-\alpha}\) puede inflarse, generando intervalos excesivamente conservadores. Tampoco detecta 
    heterocedasticidad automáticamente.

\end{itemize}

% ------------------------------------------------------------------------------------------------------------

\subsubsection{\textit{Conformalized Quantile Regression} (CQR)}

Como su nombre indica, esta técnica se realiza sobre la regresión cuantílica. La CQR \cite{romano2019}
combina la flexibilidad de la regresión cuantílica para estimar directamente los cuantiles condicionales con 
la garantía de validez estadística proporcionada por la conformalización. Esto permite obtener intervalos de 
predicción que son asimétricos y adaptativos, ajustándose localmente a la variabilidad y distribución de los 
datos.

He optado por implementar la segunda definición del intervalo de predicción, presentada en el segundo teorema 
de \cite{romano2019}, que incluye la calibración de ambas colas para obtener intervalos asimétricos 
\cite{linusson2014}. Según el artículo, esta opción mejora las garantías de cobertura, aunque puede implicar 
un aumento en el ancho del intervalo.

\begin{itemize}

    \item \textbf{Calibración}:
    
    \begin{itemize}
        \item Se calculan dos arrays de puntuaciones de no conformidad sobre los datos del conjunto de 
        calibración como las diferencias entre los valores observados y los límites del intervalo predictivo:
        
        \begin{equation*}
        \begin{split}
            R_{upper} = \left\{ y_i - \hat{f}_{upper}(x_i) \right\}_{i=1,...,n_{calib}} \\
            R_{lower} = \left\{ \hat{f}_{lower}(x_i) - y_i \right\}_{i=1,...,n_{calib}}
        \end{split}
        \end{equation*}

        donde $\hat{f}_{upper}(x_i)$ y $\hat{f}_{lower}(x_i)$ representan los límites superior e inferior del 
        intervalo predictivo para la observación $x_i$, respectivamente, e $y_i$ es el valor observado real.

        \item Se calcula un umbral de no conformidad para un nivel de confianza dado $q_{1-\alpha}$, como el 
        cuantil $(1-\alpha)(1+1/n)$ de $R$:

        \begin{equation*}
        \begin{split}
            q_{upper_{1-\alpha}} &= Quantile_{ \lceil  (1-\alpha) (1 + 1/n)  \rceil } ( R_{upper} ) \\
            q_{lower_{1-\alpha}} &= Quantile_{ \lceil  (1-\alpha) (1 + 1/n)  \rceil } ( R_{lower} )
        \end{split}
        \end{equation*}

    \end{itemize}
    
    \item \textbf{Inferencia conformal}:
    
    \begin{itemize}

        \item Se obtiene la inferencia interválica $[\hat{f}_{lower}(x_{n+1}), \hat{f}_{upper}(x_{n+1})]$ de 
        un nuevo ejemplo con el modelo.
    
        \item Se construyen los intervalos de predicción como:
        $$
        \hat{C_\alpha}(x_{n+1}) = 
            \left[ 
                \hat{f}_{lower}(x_{n+1}) - q_{lower_{1-\alpha}}, 
                \hat{f}_{upper}(x_{n+1}) + q_{upper_{1-\alpha}}
            \right]
        $$

    \end{itemize}

\end{itemize}

CQR, al igual que ICP, es independiente del modelo y del dominio, ya que solo emplea las salidas y 
valores reales para realizar la calibración. También tiene el mismo orden de eficiencia computacional, puesto 
que realiza prácticamente las mismas operaciones que ICP, pero para cada límite del intervalo predicho, 
calibrando los cuantiles inferior y superior de manera independiente para mantener la cobertura deseada.

Sin embargo, CQR logra intervalos asimétricos y adaptativos, dado que la regresión cuantílica
estima directamente los cuantiles condicionales de la distribución de la variable objetivo,
permitiendo que los límites del intervalo se ajusten según la heterocedasticidad y la forma local de la 
distribución de los datos, en lugar de asumir una distribución simétrica o constante del error. 
%Esto resulta en intervalos que reflejan mejor la incertidumbre en distintas regiones del espacio de 
%entrada, mejorando la precisión y utilidad de las predicciones de intervalo.

% ------------------------------------------------------------------------------------------------------------

\subsubsection{\textit{Monte Carlo Conformalized Quantile Regression} (MCCQR)}


La MCCQR \cite{bethell2024} es una variante de CQR que añade el método de UQ \textit{Monte Carlo Dropout}, 
que permite estimar la incertidumbre epistémica al realizar múltiples pasadas con \textit{dropout} activado 
por la red neuronal durante la inferencia. Esto genera una distribución de predicciones, de la cual se puede 
obtener una media (como predicción final) y una varianza (como medida de incertidumbre).



En lugar de utilizar un número fijo de pasadas, MCCQR implementa un esquema adaptativo que detiene el proceso 
una vez que la varianza de las predicciones converge, redu ciendo significativamente el costo computacional.





Esta técnica modifica la inferencia conformal (no modifica la calibración, por lo que se puede emplear el 
mismo modelo con los mismos parámetros). Esta técnica incorpora \textit{Monte Carlo dropout adaptativo} 
para generar múltiples predicciones por cada instancia nueva, permitiendo estimar una distribución de 
salida a partir de la cual se calcula la varianza como medida de incertidumbre.

Posteriormente, esta distribución es utilizada junto con los umbrales calibrados de CQR 
para construir intervalos de predicción más robustos y menos conservadores que los producidos por métodos 
conformales tradicionales. 

\todo{Voy a borrar probablemente esta técica porque apenas modifica los resultados con CQR}





% ------------------------------------------------------------------------------------------------------------

\subsubsection{\textit{Conformal Residual Fitting} (CRF)}

\todo{Esta técnica permite muchas implementaciones, ya que trata de predecir la incertidumbre de una 
predicción mediante otro modelo ML. Estoy intentando implementar SCCP (Seedat, Nabeel et al., 2023),
pero me está llevando más tiempo del esperado. }

$$
R = \left\{ 
        \frac{| y_i - \hat{y_i} |}{\sigma(x)} 
    \right\}_{i=1,...,n_{calib}}
$$


$$
\hat{C_\alpha}(x_{new}) = 
    \left[ 
        \hat{f}(x_{new})- \hat{\sigma}(x_{new}) \cdot \delta_\alpha, 
        \hat{f}(x_{new})+ \hat{\sigma}(x_{new}) \cdot \delta_\alpha
    \right]
$$



% ------------------------------------------------------------------------------------------------------------

\subsection{Métodos de predicción conformal para clasificación}


\subsubsection{Least-Ambiguous set-valued Classifiers (LAC)}

LAC es el primer método propuesto de predicción conformal para problemas de clasificación, en 
\cite{sadinle2019}. Propone un enfoque de clasificación de conjuntos de valores (set-valued classification) 
en el que, en lugar de asignar una única etiqueta a cada instancia, se selecciona un conjunto de etiquetas 
que garanticen un nivel de confianza predeterminada por el usuario.

Para cada instancia del conjunto de calibrado se obtiene un error, que es calculado como
$$
R_i = 1 - \hat{\pi}_{Y_i}(X_i)
$$
donde
\begin{itemize}
    \item $X_i$ e $Y_i$ son la imagen y la etiqueta de la instancia $i$.
    \item $\hat{\pi}(X_i)$ es el vector de valores de certeza (scores) de las clases para la imagen $i$, 
    arrojadas por el modelo.
    \item $\hat{\pi}_{Y_i}(X_i)$ es el valor de certeza para la clase de la etiqueta verdadera.
\end{itemize}

Luego, este error también se puede ver como la suma de valores de certeza de todas las clases salvo la 
correspondiente a la etiqueta verdadera.

% Hablar sobre que este es el principal método para clasificación binaria

\todo{Por completar (JULIO)}


\subsubsection{Adaptive Prediction Sets (APS)}

APS \cite{romano2020} ... es más adaptativo, de forma que los conjuntos 
de predicciones sean más pequeños en ejemplos fáciles de clasificar, y más grandes en ejemplos difíciles, 
y de esta forma la predicción sea más informativa.

Dado el vector de valores de certeza de una determinada instancia $\hat{\pi}(X_i)$, podemos ordenar sus 
valores en orden decreciente:

$\hat{\pi}_{(1)}(x) \ge \hat{\pi}_{(2)}(x) \ge \cdot\cdot\cdot \ge \hat{\pi}_{(K)}(x)$

La filosofía de este método no habla tanto de error, sino más bien de una valoración de certeza a incluir en 
el conjunto de predicción que garantice la inclusión de la etiqueta verdadera ---para la cobertura 
requerida---.

$$
E_i=\sum_{j=1}^{k} \hat{\pi}_{(j)}(X_i) \textnormal{ donde } (k)=Y_i
$$


El enfoque APS es más permisivo que LAC, ya que reconoce que un modelo puede identificar características 
comunes entre varias clases y generar valores de certeza repartidos, y aun así seguir siendo un buen modelo. 
Lo crucial es que el valor de certeza más alto corresponda a la clase verdadera, y por eso APS ...


\todo{Por completar (JULIO)}



\subsubsection{Regularized Adaptive Prediction Sets (RAPS)}

RAPS \cite{angelopoulos2020} es una variante del método APS, que añade una penalización a conjuntos de 
predicción demasiado grandes, realizando esto a través de la suma de un componente calculado al error $R$.

Para ello, introduce dos parámetros:
$$
R_i = \sum_{j=1}^{k}{} + \lambda \ max(k-k_{reg},0)
$$

\begin{itemize}
    \item $k_{reg}$, que es el tamaño óptimo del conjunto de predicción (en el sentido de que, si todas los 
    conjuntos de predicción tuvieran ese tamaño, se alcanzaría la cobertura deseada)
    \item $\lambda$, un parámetro de regularización que penalizará más a aquellos conjuntos que superen 
    $k_{reg}$ etiquetas predichas cuanto mayor valor tenga.
\end{itemize}


\todo{Por completar (JULIO)}




%Se suele determinar tamaños de lotes que sean potencias de dos, en el rango de 16 a 256 \cite{goodfellow2016}. 

% Una estrategia recomendada consiste en:

% \begin{enumerate}

%     \item Entrenar inicialmente solo el nuevo head (la capa o capas finales añadidas para la tarea específica) 
%     durante una época con una tasa de aprendizaje (learning rate) alta, manteniendo el resto de los parámetros 
%     del modelo congelados (sin actualizar).

%     \item Luego, realizar un entrenamiento adicional de todo el modelo (incluyendo las capas preentrenadas) 
%     con una tasa de aprendizaje más baja, permitiendo un ajuste fino (fine-tuning) de todos los parámetros.

% \end{enumerate}
   \chapter{Experimentación}

% ------------------------------------------------------------------------------------------------------------

\section{Protocolo de validación experimental}

Como se ha comentado anteriormente, se han proporcionado los datos ya dividos en conjunto de entrenamiento
(\textit{train}) y de test, para evitar problemas asociados al \textit{data snooping}. El 
\textbf{\textit{data snooping}} ocurre cuando información del conjunto de test se filtra, directa o 
indirectamente, en el proceso de entrenamiento del modelo, lo que puede llevar a una sobreestimación del 
rendimiento y a modelos que generalizan pobremente en datos nuevos.




Debemos ser cuidadosos a la hora de tratar los datos en test, no debemos ... la variable de edad de los 
individuos, ya que es el target en nuestro problema de regresión, y cualquier ... puede ... Esto se conoce
como data snooping





Para valorar los resultados obtenidos en los experimentos realizados se han divido los datos de 
entrenamiento en \textit{train} y \textit{validation}. 

Se consideró la validación cruzada (\textit{cross-validation}), pero 
\textit{data split}


\begin{figure}[h]
    \centering
    \includegraphics[width=\textwidth]{capitulos/cap_04/imagenes/data_split_base.png}
    \caption[
        Diagrama de división del \textit{dataset} en \textit{train}, \textit{validation} y \textit{test}.
    ]{
        Diagrama de división del \textit{dataset} en \textit{train}, \textit{validation} y \textit{test}. 
        Elaboración propia.
    } 
    \label{fig:data_split_base}
\end{figure}

Sin embargo, aquellos métodos de predicción conformal requieren de una fracción 

\begin{figure}[h]
    \centering
    \includegraphics[width=\textwidth]{capitulos/cap_04/imagenes/data_split_conformal.png}
    \caption[
        Diagrama de división del \textit{dataset} en \textit{train}, \textit{validation}, \textit{calibration} 
        y \textit{test}.
    ]{
        Diagrama de división del \textit{dataset} en \textit{train}, \textit{validation}, \textit{calibration}
        y \textit{test}. Elaboración propia.
    } 
    \label{fig:data_split_conformal}
\end{figure}



% ------------------------------------------------------------------------------------------------------------

\section{Métricas}



% MAE
% MSE
% R² 

% Covertura empírica
% Tamaño intervalo medio

% ------------------------------------------------------------------------------------------------------------

\section{Experimentos realizados}


   \chapter{Conclusiones y trabajos futuros}

\section{Conclusiones}

\subsection{Conclusiones sobre mejor método }

Tengo que sintetizar y redactar:

CQR es el mejor método empleado para estimación de edad, pero hay mucha incertidumbre en las edades más avanzadas, por lo que es mejor emplearlo cuando hay indicios de que el individuo tiene una edad joven. 

Los métodos de CP apenas reducen el rendimiento en sus predicciones puntuales respecto a método sin CP.

Los resultados de los métodos de CP mejorarán a medida que se mejore el desempeño general del modelo.

Teorema de No Free Lunch, pero en la práctica sí hay métodos mejores para dejar de sobrecubrir tanto en algunos grupos para cubrir más en grupos infracubiertos. El objetivo es la cobertura condicional.

Al igual que con las predicciones puntuales, se sigue requierendo un estudio previo con las métricas en base a las variables específicas de cada problema, para conocer las debilidades y fortalezas del modelo (donde estima mejor y dónde peor el modelo, donde infracubre y dónde sobrecubre). 

Dentro del procedimiento de estimación del perfil biológico, una predicción con gran incertidumbre puede indicar que hay que revisar la imagen para repetir la prueba o hacer otra prueba distinta. 

El potencial de las herramientas de CP está en la flexibilidad para integrarse con otros métodos de estimación de incertidumbre e incluir información del problema específico para reducir la incertidumbre en las predicciones conformales.

¿Es posible recalibrar el modelo para mejorar su cobertura condicional? Por ejemplo, una vez calibrado el modelo con CQR se podría calibrar de nuevo pero con el cálculo de más umbrales sobre los mismos datos de calibración en base a: el decil de tamaño de intervalo, o la edad predicha (parte entera), si bien hay riesgo de poblaciones reducidas no representativas para calibrar varios umbrales ; aunque esto también se podría mitigar con número de cuantiles o grupos de edad adaptativos que se ajusten con un conjunto de datos adicional, que podría ser validación. A explorar en un anexo. 


Para clasificación, especialmente en aquellos problemas en los que las clases se pisan en el espacio de entrada (p.ej., una misma edad biológica puede corresponderse a una edad entera y la inmediatamente posterior), los métodos tradicionales de clasificación pueden verse forzados a elegir una única clase, incluso cuando existe ambigüedad o solapamiento entre ellas. La CP permite devolver conjuntos de clases en lugar de una sola etiqueta, lo que es particularmente útil en estos escenarios de ambigüedad inherente.

Hay dos maneras de mejorar la cobertura condicional: mejorando la cobertura sobre los grupos infracubiertos concretos con un método de conformal prediction o mejorando las predicciones del modelo sobre los grupos infracubiertos, de manera que el error se asemeje más al de los grupos cubiertos. 


% ------------------------------------------------------------------------------------------------------------
% ------------------------------------------------------------------------------------------------------------


\section{Trabajos futuros}

% Problemas que siguen existiendo:
% - Variabilidad poblacional
% - Restos incompletos o fragmentados
% -   

% (Conformal Risk Control, 2025)
% ecently there have been many extensions of the conformal algorithm, mainly targeting
% deviations from exchangeability [9–12] and improved conditional coverage [3, 13–16]. Most relevant to us is
% recent work on risk control in high probability [17–19] and its applications [20–26, inter alia].

% Explicabilidad de la Conformal Prediction
   
   \nocite{*}
   \printbibliography
   
   \appendix
   % \chapter{Problema de estimación de sexo}


   %\chapter{Comparación de resultados de estimación de edad y clasificación de edad}

Es interesante comparar los resultados obtenidos en los problemas de estimación de edad y clasificación de edad, ya que existe una correspondencia directa entre ambos enfoques, si bien es necesario salvar ciertas distancias conceptuales y asumir limitaciones metodológicas. Esto se debe a que, mientras el primero trata valores como 17.1 y 17.9 como datos numéricamente diferenciados ---aunque relativamente próximos---, el segundo los interpreta dentro de una misma categoría, es decir, 17 años, sin capturar la variación interna que existe dentro de la clase. A pesar de ello, dado que existe un número relativamente alto de edades posibles y la variabilidad inherente a la predicción suele estar bastante dispersa en el dominio de predicción, se puede establecer una relación entre ambas tareas.

En la Tabla \ref{tab:AE_AC_comparative} se muestran los resultados de los mejores algoritmos en cada modalidad según dos métricas clave: la cobertura empírica y el tamaño medio del conjunto (en clasificación) o la amplitud media del intervalo (en regresión). Se observa que ningún método domina claramente sobre ningún otro; es decir, no existe un algoritmo que consiga simultáneamente una mayor cobertura y un menor tamaño/amplitud media del intervalo predictivo. Esto nos lleva a pensar que los tres algoritmos están sobre una frontera de compromiso de valor similar.

Tendríamos que valorarlo por otras aspectos como adaptatividad en base a tamaño o edad cronológica

\renewcommand{\arraystretch}{1.4}
\begin{table}[htbp]
    \small 
    \centering
    \begin{tabular}{ccccccccc}
    \toprule
    \multirow{2}{*}{\textbf{Ejecución}} &  & \multicolumn{3}{c}{\textbf{\begin{tabular}[c]{@{}c@{}}Cobertura \\[-0.8ex] Empírica (\%)\end{tabular}}} &  & \multicolumn{3}{c}{\textbf{\begin{tabular}[c]{@{}c@{}}Tamaño Medio \\[-0.8ex] del Conjunto\end{tabular}}} \\ \cline{3-5} \cline{7-9} 
 &  & \textbf{CQR} & \textbf{LAC} & \textbf{SAPS} &  & \textbf{CQR} & \textbf{LAC} & \textbf{SAPS} \\ \cline{1-1} \cline{3-5} \cline{7-9} 
    Ejecución 1 &  & 95.31 & 94.66 & 94.98 &  & 6.23 & 5.79 & 6.05 \\
    Ejecución 2 &  & 94.80 & 94.24 & 95.12 &  & 6.11 & 5.76 & 6.03 \\
    Ejecución 3 &  & 95.45 & 95.21 & 95.26 &  & 6.02 & 6.04 & 6.17 \\
    Ejecución 4 &  & 94.61 & 94.89 & 94.80 &  & 5.90 & 5.86 & 5.98 \\
    Ejecución 5 &  & 94.93 & 95.17 & 95.21 &  & 5.92 & 5.77 & 6.16 \\
    Ejecución 6 &  & 94.33 & 94.80 & 95.45 &  & 5.94 & 5.80 & 6.08 \\
    Ejecución 7 &  & 95.26 & 93.91 & 94.56 &  & 6.00 & 5.69 & 6.07 \\
    Ejecución 8 &  & 95.12 & 95.59 & 95.86 &  & 6.08 & 6.03 & 6.28 \\
    Ejecución 9 &  & 94.93 & 94.70 & 95.59 &  & 6.06 & 5.86 & 6.15 \\
    Ejecución 10 &  & 94.56 & 94.14 & 94.80 &  & 5.96 & 5.88 & 6.36 \\ \cline{1-1} \cline{3-5} \cline{7-9} 
    Media &  & \textbf{94.93} & \textbf{95.45} & \textbf{95.16} &  & \textbf{6.02} & \textbf{5.85} & \textbf{6.13} \\ 
    \bottomrule
    \end{tabular}
    \caption[
        Cobertura empírica y 
    ]{   
        Cobertura empírica y 
    }
    \label{tab:AE_AC_comparative}
\end{table}



   \chapter{Intervalos de valores razonables}

En este apartado diferenciaremos los tipos de intervalos de valores nos permiten cuantificar la variabilidad de los resultados y, por tanto, la incertidumbre de la medición realizada. 

\begin{itemize}

    \item El \textbf{intervalo de confianza (IC)} es una herramienta común de la estadística frecuentista, que permite estimar un rango de valores tal que podamos confiar en que contiene al valor verdadero de un parámetro poblacional desconocido $\theta$ (p.ej., la media) \cite{berrendero2025}.

    Los métodos del cálculo del IC dependen de la distribución del estimador (p.ej., la distribución de la
    media muestral) y los parámetros conocidos. 

    Es importante aclarar un malentendido común: un intervalo de confianza con nivel 95\% para un parámetro $\theta$ no significa que exista un 95\% de probabilidad de que $\theta$ esté dentro del intervalo calculado a partir de una muestra específica. En realidad, el 95\% se refiere a la frecuencia con la que, si muestreásemos muchas veces los datos, los intervalos construidos a partir de esas muestras incluirían al valor verdadero de $\theta$ en aproximadamente el 95\% \cite{murphy2022} (véase la Figura \ref{fig:confidence_interval}).

    \begin{figure}[htbp]
        \centering
        \includegraphics[width=0.55\textwidth]{apendices/imagenes/confidence_interval.png}
        \caption[
            Ejemplo de intervalo de confianza para la media poblacional.
        ]{
            Ejemplo de intervalos de confianza para la media poblacional. La interpretación correcta del nivel de confianza (95\% en este caso) es: \textit{Si repitiéramos el proceso de muestreo y construcción de intervalos muchas veces, aproximadamente el 95\% de ellos contendrían el verdadero valor de la media poblacional}. En esta simulación, la media real conocida es 0.153, y podemos ver que la mayoría de los intervalos la capturan, mientras que unos pocos (generalmente alrededor del 5\%) no lo logran.
            En general, se suele pedir uno solo de estos intervalos, calculado con toda la muestra disponible, aunque la media poblacional podrá estar o no contenida, pero es desconocido. 
        } 
        \label{fig:confidence_interval}
    \end{figure}


    \item El \textbf{intervalo de credibilidad o región creíble (RC)} es, de hecho, la que determina que el parámetro $\theta$ está contenido en el rango de sus valores con una probabilidad determinada por el nivel de credibilidad. Este intervalo es la aproximación bayesiana equivalente al intervalo de confianza, y, como este, requiere conocer la distribución a priori de los datos.

    La diferencia radica en que, a diferencia del intervalo de confianza, que parte de que $\theta$ es un parámetro fijo desconocido y los datos son tratados como aleatorios, el enfoque bayesiano fija los datos(ya que son conocidos) y el parámetro $\theta$ lo trata como aleatorio (ya que es desconocido) \cite{murphy2022}.

    Esta interpretación resulta más intuitiva y directa en comparación con la interpretación frecuentista del intervalo de confianza. En particular, una región creíble del 95\% sí puede interpretarse como que hay un 95\% de probabilidad de que el parámetro $\theta$ se encuentre dentro de ese intervalo, dado el conjunto de datos observado y la distribución a priori asumida.

    \begin{figure}[htbp]
        \centering
        \includegraphics[width=0.75\textwidth]{apendices/imagenes/credibility_interval.png}
        \caption[
            Ejemplo de intervalo de credibilidad para la media poblacional.
        ]{
            Ejemplo de intervalo de credibilidad para la media poblacional. La interpretación correcta es: \textit{Con un 95\% de probabilidad, el valor verdadero está dentro del intervalo}. En esta simulación, la media real conocida es 0.153, y podemos observar que efectivamente este valor está contenido en el intervalo.
        } 
        \label{fig:credibility_interval}
    \end{figure}


    \item El \textbf{intervalo de predicción (\textit{prediction interval})} es radicalmente diferente a los intervalos previos. Trata de predecir un valor futuro de una observación, no determinar un parámetro poblacional. Existen numerosos métodos, con y sin necesidad de conocer la distribución de los datos. 
    
    El enfoque explorado en este trabajo es la predicción conformal, que ha demostrado ser eficaz en contextos donde los supuestos clásicos (normalidad, homocedasticidad) no se cumplen \cite{romano2019}, y es actualmente el enfoque más robusto para la construcción de intervalos de predicción en aplicaciones modernas de ML \cite{romano2019, luo2025, sadinle2019, romano2020, angelopoulos2020}. La predicción conformal tiene una interpretación frecuentista: $1-\alpha$ intervalos producidos cubren el verdadero valor (véase la Figura \ref{fig:prediction_intervals}).

    \begin{figure}[htbp]
        \centering
        \includegraphics[width=0.75\textwidth]{apendices/imagenes/prediction_intervals.png}
        \caption[
            Intervalos de predicción (95\% de confianza) construidos con CQR para estimación de edad.
        ]{
            Intervalos de predicción (95\% de confianza) construidos con CQR para estimación de edad.
        } 
        \label{fig:prediction_intervals}
    \end{figure}

\end{itemize}

% Relacionándolo con el apartado anterior, los intervalos de confianza solo informan de la incertidumbre 
% aleatoria, puesto que asumen que el modelo está bien especificado y que el error residual es la única fuente 
% de variabilidad. 

% Los intervalos de credibilidad sí integran la incertidumbre en los parámetros del modelo y en la estructura 
% del modelo mismo, capturando así tanto la incertidumbre aleatoria como la incertidumbre epistémica.

% Frente a estos, los intervalos de predicción obtenidos mediante predicción conformal sí presentan información
% de los tres componentes de incertidumbre. 

Como podemos esperar, a más estrecho sea el intervalo que manejemos, más se puede confiar en las predicciones pero no todos los tipos de intervalos revelan la misma información sobre incertidumbre. 

   \chapter*{}
   \thispagestyle{empty}

\end{document}
