\chapter{Identificación genética}

En este también se recomienda la identificación genética como técnica que ofrece más garantías de fiabilidad y 
precisión en los procesos de identificación humana, destacando la importancia de establecer una base de datos nacional 
de ADN que permita cruzar información genética. Además, enfatiza la necesidad de contar con laboratorios especializados 
en genética forense, dotados de tecnología avanzada y personal cualificado, debido a los desafíos técnicos que presenta 
la degradación del ADN en los restos exhumados y la posible contaminación de las muestras

Este trabajo se centra en la estimación del perfil biológico (edad, sexo, estatura) más que en la identificación 
humana, por lo que no abordará análisis genéticos. No obstante, el estudio osteológico mantiene su relevancia por 
varias razones:

\begin{enumerate}
    
    \item La concentración de ADN se reduce drásticamente en los primeros 8 meses post-mortem \cite{higgins2015differential}. 
    Y factores como las altas temperaturas, la exposición a humedad ambiental o la presencia de aguas subterraneas y 
    entornos ricos en oxígeno, que fomentan la presencia microbiana, perjudican la conservación del ADN 
    \cite{latham2018dna}. 

    Además, a parte de la degradación, el ADN puede sufrir contaminación, tanto por microorganismos presentes en el entorno 
    del cadáver, como por restos óseos de otros individuos.

    % Separar el ADN de dos humanos es imposible en este tipo de situaciones. 

    %Un análisis realizado a partir de 15 informes involucrando a 933 cadáveres exhumados en el Cementerio de Paterna, 
    %Valencia, determinó en 149 las identificaciones exitosas, suponiendo un 15,9 \% de individuos identificados.

    \item Se necesitan muestras con las que comparar las secuencias de ADN extraída de los individuos, a ser posible de
    familiares de primer grado. 


    \item El proceso genético para la identificación es costoso, especialmente si son muchas las muestras a cruzar. 
    Una estimación previa del perfil biológico podría facilitar la identificación, pudiendo comparar el ADN con una base
    de datos más pequeña.

\end{enumerate}
