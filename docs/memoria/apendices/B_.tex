


\subsection{Antropología forense}

La antropología es el estudio de la humanidad en todas sus dimensiones, a lo largo del tiempo y en distintas partes del 
mundo. Se estructura en cuatro campos fundamentales: biología humana, arqueología, antropología cultural y lingüística, 
que constituyen la base de la disciplina \cite{AAA2022AnthropologyDefinition}.

La antropología física o biológica se centra en el estudio de la anatomía, el crecimiento, la adaptación y la evolución 
del cuerpo humano \cite{nawrocki1996OutlineFA}.

Dentro de este campo, la \textbf{antropología forense (AF)} es el subcampo especializado que aplica métodos y técnicas 
antropológicas para resolver cuestiones médico-legales \cite{nawrocki1996OutlineFA}, empleando conocimientos de antropología 
física, aunque a veces también de la arqueológica, para la correcta recuperación y análisis de la evidencia forense.

Tradicionalmente, los antropólogos forenses han tenido cinco principales objetivos en su trabajo \cite{byers2023}:

\begin{enumerate}

    \item Determinar el \textbf{perfil biológico} de un individuo (es decir, su grupo étnico, sexo, edad y estatura) 
    cuando los tejidos blandos se han deteriorado hasta el punto de que estas características no pueden determinarse
    mediante inspección visual. 

    \item Identificar la naturaleza de lesiones traumáticas (como heridas de bala, puñaladas o fracturas) en huesos humanos, 
    así como sus causantes, con el objetivo de recopilar información sobre la causa y circunstancias de la muerte.

    \item Estimar el intervalo postmortem, es decir, el tiempo transcurrido desde la muerte, gracias a su conocimiento sobre
    los procesos de descomposición corporal.
    
    \item Asistir en la localización, recuperación y conservación de los restos (superficiales o enterrados) aplicando 
    técnicas arqueológicas, garantizando la recolección de toda la evidencia forense relevante.

    \item Proporcionar información clave para la \textbf{identificación} de los fallecidos, basándose en las características
    únicas presentes en los esqueletos. 

\end{enumerate}

Sin embargo, además de las funciones ya descritas, los antropólogos forenses desempeñan otros roles importantes en la 
sociedad moderna.

Una de ellos es \textbf{estimar la edad de personas vivas en casos legales} donde la edad es un factor determinante 
\cite{schmeling2016}, como en solicitudes de asilo, adopciones internacionales o juicios donde se debe establecer si un 
individuo es menor o mayor de edad, lo que puede tener consecuencias legales significativas.

También son consultados para  \textbf{identificar víctimas de catástrofes masivas} \cite{deBoer2019, prinz2007}, 
como accidentes aéreos, ataques terroristas, desastres naturales o cualquier otro fenómeno que cause numerosas 
muertes con restos mutilados o desfigurados.

Los antropólogos forenses desempeña un papel crucial en la \textbf{recuperación e identificación de víctimas de exterminios, 
persecuciones y represiones} \cite{skinner2003}. Casos como la Guerra Civil Española y la Dictadura Franquista 
\cite{sanchisgimeno2024, baeta2015}, así como las múltiples dictaduras en el Cono Sur de América \cite{ataliva2024}, 
han requerido la intervención de equipos forenses para esclarecer la verdad histórica y restituir la identidad de las 
víctimas a sus familiares. A través de excavaciones arqueológicas y análisis forense, estos equipos no solo reconstruyen 
los hechos del pasado, sino que también contribuyen a procesos de memoria, justicia y reparación.

La ciencia forense ha sido fundamental en la \textbf{investigación de crímenes de guerra contra poblaciones civiles}. 
Organizaciones como Médicos por los Derechos Humanos y la ONU financian equipos especializados que documentan 
estos crímenes, proporcionando pruebas esenciales para tribunales internacionales \cite{tanaka2020}.

La relevancia de la ciencia forense en la identificación de víctimas y la protección de la dignidad humana ha convertido 
su aplicación en un pilar fundamental de los derechos humanos y la justicia internacional, naciendo así la  
\textbf{acción forense humanitaria} \cite{cordner2017}, una disciplina que emplea la ciencia forense con un propósito 
exclusivamente humanitario, con los objetivos de: identificar a las personas fallecidas, gestionar dignamente sus restos y 
aliviar el sufrimiento de sus familias en situaciones de conflicto, migración y desastres naturales \cite{tidballbinz2021}. 

De los objetivos anteriormente descritos, el que posee mayor relevancia para el desarrollo de este proyecto es la 
estimación del Perfil Biológico, si bien este problema está estrechamente ligado con la Identificación 
Humana.

% --------------------------------------------------------------------------------------------------------------------------

\subsection{Identificación Humana}

La \textbf{identificación humana (ID)} consiste en la determinación y verificación de la identidad de una persona mediante el 
análisis de elementos directamente relacionados con los hechos, así como de características individuales lo más objetivas 
y perdurables posibles, que permitan un análisis detallado y posean un potencial discriminatorio común a todos los individuos 
\cite{lorente1999}. 

Esta identificación se basa en dos tipos de evidencia \cite{thompson2006}: 

\begin{itemize}

    \item evidencias circunstanciales, como la hora y lugar del descubrimiento del cuerpo, efectos personales (por ejemplo, 
    ropa, joyas o contenido en los bolsillos), y la confirmación visual de la identidad por parte de familiares o amigos; y

    \item evidencias físicas de identidad, obtenidas a través de la examinación externa de características como el sexo, 
    el color de piel, tatuajes, o huellas dactilares. Cuando estas no son suficientes, se recurre a la examinación interna 
    mediante técnicas médico-científicas, como el análisis de fracturas curadas, condiciones patológicas, registros dentales 
    o perfil genético (ADN).

\end{itemize}

% DUDA: ¿Introduzco los niveles de confirmación de identidad (pag. 16 de Forensic Human Identification An Introduction)?

En este trabajo, nos centraremos en las evidencias obtenidas a partir del examen interno, que constituyen el eje principal 
de la antropología forense, aunque también intervienen otras disciplinas que aportan información relevante, como la 
dactiloscopía y la genética forense.

Cabe destacar que, aunque los análisis dactilares y genéticos superan en precisión identificativa a los métodos 
antropológicos, su aplicabilidad enfrenta limitaciones técnicas significativas que condicionan su uso en contextos 
forenses reales.

Las huellas dactilares requieren de: tejido blando preservado, lo que es común en cadáveres frescos, pero se 
pierde con la descomposición o la carbonización; y una base de datos que incluya la huella del individuo en vida 
(registros \textit{antemortem}).

Y, por otro lado, el análisis genético puede verse comprometido por una mala conservación del ADN que puede deberse a:

\begin{itemize}
        
    \item Degradación del ADN: la concentración presente en un cadáver se reduce drásticamente en los primeros 8 
    meses post-mortem \cite{higgins2015}, y factores como las altas temperaturas, la exposición a
    humedad ambiental o la presencia de aguas subterraneas y entornos ricos en oxígeno, que fomentan la presencia 
    microbiana, perjudican la conservación del ADN \cite{latham2018}.

    \item Contaminación de las muestras: microorganiosmos presentes en el entorno del cadáver, la mezcla con restos 
    óseos de otros seres o la manipulación inadecuada pueden alterar los resultados también.

\end{itemize}

Y, aun extraída una secuencia válida de ADN, se necesita de muestras con las que compararla, a ser posible de 
familiares de primer grado, para establecer una identificación concluyente. 

%NOTA: ¿Debería incluir que las técnicas antropológicas también son sensibles a la disponibilidad de los
% restos óseos?

Por tanto, la antropología forense contribuye al problema de identificación en dos escenarios \cite{swganth2010}:  

\begin{enumerate}

    \item Cuando los otros métodos no son viables, dado que las pruebas no se puedan recoger o no sean válidas, o 
    no haya registros con los que compararlas.

    \item Como apoyo a otras técnicas de identificación. Por ejemplo, las técnicas de estimación del perfil 
    biológico puede reducir el grupo de posibles coincidencias en bases de datos genéticos, facilitando el 
    cotejo de secuencias genéticas y reduciendo el coste del proceso.  

\end{enumerate}

% --------------------------------------------------------------------------------------------------------------------------

\subsection{Estimación del Perfil Biológico}

La \textbf{estimación del perfil biológico (PB)} es el proceso mediante el cual se analizan los restos óseos para 
determinar las características biológicas clave de un individuo desconocido: sexo, edad, estatura y ascendencia.

Y, como ya hemos visto, el principal objetivo de la estimación del PB es limitar el rango de potenciales 
coincidencias durante el proceso de ID. Cuando estas características biológicas coinciden con los registros 
\textit{antemortem}, se fortalece la hipótesis de identificación; en cambio, si existencias una o más 
discrepancias (especialmente de alguna característica firme como múltiples epífisis no fusionadas, que no 
pueden ocurrir en un adulto mayor), el individuo es excluido como posible coincidencia 
\cite[cap. 18]{byers2023}. 
En la Figura \ref{fig:SFI_pipeline} podemos observar que la estimación del PB es uno de los primeros 
pasos en el proceso de ID forense. 

% \begin{figure}[h]
%     \centering
%     \includegraphics[width=\textwidth]{imagenes/01_SFI_pipeline.png}
%     \caption{Procedimiento secuencial para la identificación forense basada en el esqueleto humano 
%             (\textit{skeleton-based forensic identification}, SFI) \cite{mesejo2020}} 
%     \label{fig:SFI_pipeline}
% \end{figure}

La estimación del perfil biológico en restos humanos es una tarea compleja, especialmente cuando se estima la edad
de un cuerpo, ya que hay diferentes métodos a aplicar dependiendo de la fase de desarrollo del individuo. 
Las variaciones en la morfología de los huesos son bien conocidos, pero estos no ocurren al mismo tiempo en 
diferentes individuos, ya que no están expuestos a las mismos condiciones genéticas y del entorno.

% --------------------------------------------------------------------------------------------------------------------------

\subsection{Estimación de edad forense}

% Sin embargo, y como veníamos anticipando en el primer apartado, la estimación de edad es un proceso que también se 
% aplica en personas vivas, principalmente en contextos legales y administrativos, para reconocer su responsabilidad
% legal.

La estimación forense de la edad no se limita únicamente al ámbito de la identificación humana. 
Como se mencionó previamente, este proceso también se aplica en personas vivas, normalmente en 
contextos legales y administrativos, para determinar la edad de migrantes, donde la falta de 
registros confiables dificulta garantizar sus derechos o determinar sus responsabilidades legales.

Los huesos no son estructuras estáticas a lo largo de la vida, sino que están en constante transformación, aunque 
de manera lenta y progresiva. Estos cambios sutiles permiten estimar la \textbf{edad fisiológica} 
\cite[cap. 9]{byers2023} de una persona, es decir, identificar la fase del desarrollo o envejecimiento en la que se 
encuentra el esqueleto dentro del continuo de transformaciones que se producen desde el nacimiento hasta la vejez.

Esta edad fisiológica no siempre coincide con la \textbf{edad cronológica} \cite[cap. 9]{byers2023}, que representa 
el tiempo transcurrido desde el nacimiento hasta la muerte o el momento de la estimación, aunque ambas suelen estar 
estrechamente correlacionadas.

A diferencia de la edad cronológica, la edad fisiológica no cuenta con una verdad absoluta o \textit{ground truth}; su 
finalidad es aproximarse lo máximo posible a la edad real mediante el uso de indicadores biológicos que, idealmente, 
sean lo más independientes posible del contexto poblacional y, al mismo tiempo, ofrezcan un alto grado de exactitud.

La elección de los métodos adecuados para la estimación de la edad depende tanto del tipo de restos óseos 
disponibles como de si el individuo corresponde a un subadulto o a un adulto \cite{ubelaker2019}. 

La estimación en cuerpos subadultos se basa en el desarrollo y erupción dental, la fusión ósea (osificación)
o la maduración y el tamaño del esqueleto \cite{scheuer2000}.

% Para esqueletos subadultos, el desarollo dental proporciona el indicador de edad con mayor exactitud.
% Si estos estuvieran fragmentados o desaparecidos, entonces el tamaño y la morfología de los huesos

La valoración en adultos es más compleja, dado que el desarrollo de la dentadura se ha completado, 
así como el crecimiento del esqueleto ha cesado \cite[cap. 9]{byers2023}, por lo que los indicadores 
se basan más en características del deterioro óseo; pero la variabilidad de estas aumenta con la 
edad debido al efecto acumulativo de las influencias ambientales \cite{ubeleaker2018, scheuer2004}.

% Por ejemplo, estudios como \cite{merritt2015,wescott2015} indican que la obesidad puede causar que se sobreestime 
% la edad del cuerpo, mientras que personas con una complexión más ligera o bajo peso corporal tienden a presentar 
% una infraestimación de la edad.

Heather Garvin y Nicholas Passalacqua \cite{garvin2012} indican que la mayoría de los 
antropólogos forenses se basan principalmente en cuatro características osteológicas : 
degeneración de la sínfasis púbica \cite{alsalihi2002}, 
de la superficie auricular del ilion \cite{lovejoy1985}, 
del extremo esternal de la cuarta costilla \cite{iscan1984}, 
y análisis de los proceso de obliteraicón de las suturas craneales \cite{meindl1985},
siendo este último el menos fiable pero el único practicable en muchas ocasiones.

% Los tres primeros se basan en análisis del aspecto morfológico de áreas que originalmente están compuestas 
% por hueso liso y juvenil, pero que con la edad se tornan rugosas y presentan desarrollo osteofítico. 
% El cuarto, en cambio, se enfoca en las líneas de sutura que separan los huesos del cráneo (incluyendo el paladar), 
% las cuales se cierran con la edad hasta fusionarse y eventualmente pueden obliterarse por completo.

Sin embargo, cuando la estimación de edad se realiza a una persona viva, no se tiene acceso a sus 
huesos de forma directa. En estos casos, que en su mayoría van dirigidos a determinar la mayoría o menoría de edad
de una persona, el análisis se realiza sobre imágenes médicas, como radiografías de manos, radiografías panorámicas
de los maxilares o tomografías computerizadas de cortes finos de la epífisis mediales de las clavículas 
\cite{schmeling2016}. Se suelen combinar múltiples métodos para una mayor exactitud en la predicción. Dependiendo de 
los asuntos legales, se requerirá la estimación de la edad mínima del individuo o su edad más probable.




% La \textbf{estimación de edad forense} es un proceso crucial en contextos legales y administrativos, ya que permite a las 
% autoridades determinar la edad oficial de personas cuya documentación es inexistente o no válida. Este procedimiento resulta 
% especialmente relevante en casos como los de migrantes, menores no acompañados o personas en situación de vulnerabilidad, 
% donde la falta de registros confiables dificulta garantizar sus derechos.

% La determinación de minoría o mayoría de edad adquiere particular importancia, pues de ella dependen derechos fundamentales, 
% privilegios procesales y beneficios sociales. Mientras los menores acceden a protecciones especiales en educación, salud y 
% justicia, los mayores de edad asumen responsabilidades legales plenas. Una estimación precisa, basada en métodos científicos, 
% asegura que estas distinciones se apliquen con equidad, protegiendo tanto a los individuos como al sistema jurídico.

% Los componentes esenciales para la estimación de edad son: 

% \begin{itemize}
%     \item el historial clínico,
%     \item un examen físico,
%     \item una radiografía de las manos, 
%     \item radiografía panorámica de los maxilares, y, 
%     \item si está indicado, una tomografía computerizado de cortes finos de la epífisis mediales de las clavículas.
% \end{itemize}

% Se suelen combinar múltiples métodos para una mayor exactitud en la predicción. Dependiendo de los asuntos legales, se 
% requerirá la estimación de la edad mínima del individuo o su edad más probable. 

% La determinación de minoría o mayoría de edad adquiere particular importancia, pues de ella dependen derechos 
% fundamentales, privilegios procesales y beneficios sociales. Mientras los menores acceden a protecciones especiales en 
% educación, salud y justicia, los mayores de edad asumen responsabilidades legales plenas. Con el fin de priorizar la 
% protección de los menores ---evitando que queden expuestos a violaciones de sus derechos---, se emplea la estimación de 
% edad mínima como criterio. Por ello, en casos dudosos, se tiende a clasificar al individuo como menor, incluso si en 
% realidad podría ser mayor.

% NOTA: En términos de problema de clasificación, si la clase positiva es ``menor de edad'' y la negativa ``mayor de edad'', 
% se prefiere un falso positivo a un falso negativo (alta sensibilidad, a costa de la especifidad).

% Si al evaluar un individuo, al menos una de las características del desarrollo estudiadas (desarrollo físico, madurez 
% esquelética, desarrollo dental) no es madura, significa que este no ha alcanzado la plena madurez biológica, y la edad 
% más probable obtenida en el proceso es válida y puede ser informada. 

% Estudios \cite{imaizumi2004quantitative} señalan que el material genético se degrada más rápidamente en 
% tejidos blandos que en huesos, es por ello que se extrae el ADN de estos últimos. 

% --------------------------------------------------------------------------------------------------------------------------

\subsection{Inteligencia Artificial en la estimación del perfil biológico}

Los métodos de estimación del perfil biológico se basan en la evaluación visual y en el análisis morfométrico de rasgos
esqueléticos. Sin embargo, su aplicación demanda conocimiento especializado, pueden presentar ambigüedades en su 
formulación que den lugar a interpretaciones variables y están sujetos a posibles errores de medición \cite{langley2018},
haciendo el proceso muy costoso en términos de tiempo.

Recientes avances en inteligencia artificial (IA) y machine learning (ML) han demostrado el potencial de mejorar la 
exactitud y objetividad de estimación del perfil biológico, tanto para la estimación de sexo 
\cite{curate2017, darmawan2015, pinto2016} como de edad \cite{kim2017, larson2018, lee2017}. 


% Añadir ventajas del uso de la IA:
% - reducir tiempo de identificación y estimación de edad, mediante automatización de tareas
% - reducir la subjetividad y errores humanos (errores inter e intraobservador, ambigüedades)