\chapter*{Convenciones}

\subsection*{Caracteres en negrita}

Los términos empleados para definir un concepto por primera vez están en negrita.


\subsection*{Caracteres en cursiva}

Se empleará letra cursiva para palabras en otros idiomas, títulos de obras, palabras mencionadas como términos
(no por su significado).


\subsection*{Citas}

Las citas y referencias se realizarán en estilo IEEE.
Al hacer clic en una cita, será redirigido a la referencia correspondiente en la bibliografía. Desde allí, 
podrá regresar a la página original, ya que las referencias incluyen enlaces a los lugares donde han sido 
citadas. No obstante, dado que una misma referencia puede aparecer varias veces a lo largo del texto, se 
recomienda al lector tomar nota de la página en la que se encontraba antes de saltar a la bibliografía.


\subsection*{Introducción de conceptos}

Por regla general, la primera vez que se introduce un nuevo concepto relevante  se presentará su término en 
español y, entre paréntesis: su término en el idioma de origen (en inglés normalmente), y su acrónimo o sigla 
(cuando exista). 
Por ejemplo: ``Red Neuronal Convolucional (\textit{Convolutional Neural Network}, CNN)''.

Algunas excepciones:

\begin{itemize}
    
    \item Si se refiere a un concepto que en español no tiene una traducción estandarizada y/o se emplea 
    habitualmente en su forma original (anglicismo técnico), se usará directamente el término en inglés, 
    seguido de sus siglas si las tiene.
    Por ejemplo: ``\textit{batch size} (en lugar de `tamaño de lote')''.

    \item Si el término es ampliamente conocido en su forma abreviada (incluso en español), puede omitirse la 
    explicación extendida. 
    Por ejemplo: ``ADN (en lugar de `ácido desoxirribonucleico, DNA')''.

\end{itemize}

Todo lo anterior sin perjuicio del término que se emplee más tarde en el texto, que puede emplear cualquiera
de los términos presentados, en base a su 


\subsection*{Introducción de obras y entidades}

Las obras se introducirán en 



\subsection*{Comillas}

Se emplearán comillas (``...'') para enmarar citas literales.

\begin{itemize}
    \item \textbf{Omisiones textuales}: 
        La secuencia ``[...]'' dentro de una cita indica que se ha omitido una parte del texto original.
    \item \textbf{Adaptaciones textuales}: 
        La secuencia ``[$<$texto$>$]'' dentro de una cita indica que se ha introducido una adaptación o 
        paráfrasis del verbo original para ajustar la cita al contexto de la oración.
\end{itemize}


\subsection*{Signos decimales y millares}

Dado que este trabajo está en español, se empleará la coma (``,'') como signo decimal, y el punto (``.'') como 
separador de millares.


\subsection*{Aclaraciones, incisos e información complementaria}

Se han seguido las recomendaciones de la RAE para el uso de guion largo y paréntesis, si bien se ha optado
por añadir también notas a pie de página: 

\begin{itemize}
    \item \textbf{Raya o guion largo (``---'')}: Se emplea para enmarcar incisos aclaratorios breves
    ---especialmente cuando interrumpen el flujo de la oración--- dentro del texto principal.
    Por ejemplo: ``\textit{El error es la diferencia entre el valor verdadero ---asumiendo que existe--- y el 
    valor medido.}''

    \item \textbf{Paréntesis (``()'')}: Se utilizan para añadir información complementaria concisa y no 
    esencial para la comprensión del texto.
    Por ejemplo: ``\textit{El dataset (compuesto por 10.000 ejemplos) se divide en 5 subconjuntos:...}''

    \item \textbf{Notas a pie de página}: Se reserva para incisos o información complementaria más extensa.
    
\end{itemize}


\subsection*{Intervalo}

Los intervalos pueden expresarse de dos formas:

\begin{itemize}
    \item Con los extremos del intervalo: $[a,b]$
    \item Con el punto central y la mitad de la amplitud del intervalo: $x \pm \frac{b-a}{2}$
\end{itemize}


\subsection*{Consideraciones lingüísticas}

Los términos ``F'' y ``M'' se refieren al sexo de la persona: femenino o masculino, respectivamente. 
